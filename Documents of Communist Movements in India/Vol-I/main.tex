\documentclass[oneside]{book}
\usepackage[utf8]{inputenc}
\usepackage{import}

\title{Documents of Communist Movements in India Vol-1}
\author{Cull Max}
\date{July 2020}

%\makeindex
\usepackage[pagestyles]{titlesec}


\titleformat{\chapter}[display]
  {\normalfont\bfseries}{}{0pt}{\Large}



%\titleformat{}[display]{\normalfont\bfseries}{}{0pt}{\huge}
\newpagestyle{mystyle}
{\sethead[\thepage][][]{}{}{\thepage}}
\pagestyle{mystyle}




\begin{document}
\maketitle
\tableofcontents

\chapter{The Communists’ Contribution to the Theory and Practice of Indian Politics}
\textbf{E. M. S. Namboodiripad}\\
\import{./Chapter-1/}{section-1}

\chapter{Early Contacts of the Indian Revolutionaries with the Leaders of October Revolution}
\import{./Chapter-2/}{section-1}
\import{./Chapter-2/}{section-2}
\import{./Chapter-2/}{section-3}

\chapter{Formation of Communist Party of India in Tashkent in 1920}
\import{./Chapter-3/}{section-1}

\chapter{Manifesto to the 36th Indian National Congress, Ahmedabad, 1921}
\import{./Chapter-4/}{section-1}

\chapter{Maulana Hasrat Mohani's Resolution for Complete Independence of India}
\import{./Chapter-5/}{section-1}

\chapter[5]{Lenin's Colonial Theses\raisebox{.3\baselineskip}{\normalsize\footnotemark}}

\footnotetext{"Preliminary Draft Theses on the National and the Colonial J. V. Stalin, M. G. Rates, Y. A. Preobrazhensky, N. D. Lapinsky, and I. Nedelkov (N. Shablin), representative of the Bulgarian Communists, as well as trom a number of leaders in Bashkiria, Kirghizia, and Turkestan. Along with correct ideas, the notes contained certain grave errors. Thus, Chicherin gave a wrong interpretation to Lenin's theses on the necessity of support for national liberation movements and on agreements with the national bourgeoisie, without due regard for Lenin’s distinction between the bourgeoisie and the peasantry. With regard to this Lenin wrote : “I lay greater stress on the alliance with the peasantry (which does not quite mean the bourgeoisie)’ ’ (Central Party Archives of the Institute of Marxism- Leninism of the C. C. C. P. S. U.). Referring to the relations between the future socialist Europe and the economically underdeveloped and dependent countries, Preobrazhensky wrote : “...if it proves impossible to reach economic agreement with the leading national groups, the latter will inevitably be suppressed by force and economically important regions will be compelled to join a union of European Republics.’’ Lenin decisively objected to this remark : “...it goes too far. It cannot be pioved, and it is wrong to say that suppression by force is “inevitable”. That is radically wrong” (see Voprosy fstorii KPSS [Problems of the C.P.S.U History! 1958. No. 2, p. 16). 

A grave error was made by Stalin, who did not agree with Lenin’s proposition on the difference between federal relations among the Soviet republics based on autonomy, and federal relations among independent republics. In a letter to Lenin, dated June 12, 1920, he declared that in reality “there is no difference between these two types of federal relations, or else it is so small as to be negligible”.' Stalin continued 
to advocate this later, when, in 1922, he proposed the “autonomisation” of the independent Soviet republics. These ideas were criticised in detail by Lenin in hi s article ‘ ‘The Question of Nationalities or ‘ Autonomisation’ ’ , and in his letter to members of the Political Bureau “On the Formation of the U.S.S.R.” (see present edition, Vol. 36, and Lenin Miscellany XXXVI, pp. 496-98). }
\import{./Chapter-6/}{section-1}
\end{document}
