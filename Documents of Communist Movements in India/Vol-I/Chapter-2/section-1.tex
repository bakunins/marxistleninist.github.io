\section{Indian Revoloutionaries in Moscow*}

On the eve of my departure from Tashkent, Abani Mukherjee arrived from Moscow. I had no news of his coming. He had no business there. I had sent him to the Baku Congress on the understanding that, on return to Moscow, he would leave for Western Europe to take up his headquarters in Holland with the object of establishing contact with India through the intermediacy of sailors. So his sudden arrival was not only a surprise, but it also turned out to be an embarrassment. He readily volunteered to take over charge of the Military School during my absence. He was also an ardent advocate of developing the Communist Party of India and increasing its membership. Because of his previous record with the Tcheka, Peter came to know of his arrival instantly. He had never got over the disappointment of having had to let him go out of his bloody clutches owing to my intervention, backed up by Lenin's benevolence. Given his ambitious and stormy character, Abani Mukherjee was sure to get into some trouble before long. Who would protect him this time against Peter's vengeance? Safarov disliked him heartily. Before leaving, I saw Peter to plead with him to be more lenient. He growled at me : Why did I bring Mukherjee here? He was sure to create trouble, and in that case Peter would take him without fail. The meaning of Peter taking anybody was quite unequivocal. I was frightened and told him that Mukherjee had promised to behave properly and there were a dozen intelligent Indian revolutionaries who would keep a check on him. I was surprised that Mukherjee had left Moscow just before the Third World Congress, and that he willingly agreed to stay away. I came to know the reason as soon as I returned to Moscow.

\footnote{* Taken from Memoirs of M. N. Roy, 1964 Publication, Pages 477-485 }

When I reached there, several Indian revolutionaries had arrived from Berlin as representatives of the defunct Indian Revolutionary Committee. On my way to Moscow, I had pleaded with the leading Indian revolutionaries in Berlin to proceed to Russia, which at that time offered them the only safe asylum and promised to be a reliable base for work to promote revolution in India. At that time, they did not seem to believe that the Russian Revolution would last; and Communism did not find favour with them. So, when at last they changed their mind and turned towards the base of world revolution, I was naturally very glad. But to my great surprise, the few representatives of the Berlin Revolutionary Committee who had already reached Moscow were rather cool in their response to my friendly attitude. However, I learned from them that they had come only as a vanguard of the Revolutionary Committee, which would before long reach Moscow in full force. I hoped that on the arrival of veteran revolutionaries like Virendranath Chattopadhyaya, Bhupendranath Dutta and others, the relation would change. I eagerly looked out for the arrival of men who with their revolutionary devotion and long experience could be expected to be good comrades and willing collaborators. 

Within a short time, they all arrived to announce that the Indian Revolutionary Committee of Berlin, which alone had the authority to speak on behalf of India, had decided to shift its headquarters to Moscow, if favourable conditions were offered. Although the declaration insinuated that I had no right to speak on behalf of India, I made no secret that the plan of the Indian revolutionaries shifting their headquarters to Moscow would have my fullest support; and there could be no doubt that nowhere in the world could better conditions be obtained than in Moscow; But surely enough; the newcomers not only tried to avoid me but some of them actually took up an openly hostile attitude. 

The Indian Revolutionary Committee of Berlin was then a thing of the past. Irrespective of whatever might have been its achievements in the earlier days, during the closing years of the war it was a divided house and had practically disintegrated. Instead of working on the authority of that legend, it would have been wiser to have made a new beginning under different circumstances. But it seems that the news of the formation of the emigrant Indian Communist Party at Tashkent had frightened the old nationalist revolutionaries, who regarded the new body as a challenge to their authority. If I had had the opportunity to meet the leaders of the delegation from Berlin, I could have explained the situation to their satisfaction. I did not approve of the formation of the emigrant Communist Party, and I did not believe that it had any right to speak on behalf of the workers of India, not to mention the Indian people as a whole. 

The delegation of Indian revolutionaries from Berlin was composed of fourteen people, including Virendranath Chattopadhyaya, Bhupendranath Dutta, Virendranath Das Gupta, the Maharashtrian Knankhoje, Gulam Ambia Khan Luhani, Nalini Gupta. The driving force of the delegation however was Agnes Smedley, an American by birth. I had met her in America. Then she was an anarchist-pacifist. Working as private Secretary of Lajpatrai for some time, she seemed to have developed a great sympathy for India. Having learned that famous Indian revolutionaries were living in Berlin, at the conclusion of the War she came over there and became a very active member of the Indian group. 

But the delegation which came to Moscow was evidently not the original Indian Revolutionary Committee of Berlin. Hardayal and Chattopadhyaya had been the two dominant figures of the Berlin Committee and as such they had clashed before long. No less ardently anti-British, Hardayal however was taken prisoner in Germany and detained on the suspicion of enemy espionage. When Germany surrendered, he escaped to Stockholm and wrote a book describing his experience in Germany. Evidently, the experience had embittered him. He appeared to be an apologist of the British rule, in India and advocated Dominion Status as against complete independence. He actually wrote something which, though true, ought not to be said by a revolutionary Indian nationalist. Pointing out the fact that the fighters for Indian freedom had learned their political lesson from Britain, Hardayal made the declaration that, if India was the mother, Britain was the grandmother. That naturally scandalised all Indian nationalists. It was alleged that he had written the book with the object of getting the permission to return to Britain and subsequently to India. But evidently he did not get the permission. He stayed on in Sweden and during the last years of his life taught Indian philosophy in the old University of Upsala. That was a recognition of his learning and intellectual calibre. 

The chairman of the Berlin Committee, Mohammad Hasan Mansoor, had gone to Turkey in the earlier years of the War. He returned from there to Berlin disillusioned and disgruntled and declared himself to be a Communist. The professed conversion to communism isolated him from the old colleagues. He did not join them when they came to Moscow, but later 
on came there alone and lived quietly for a couple of years. I have already referred to my e.Kperience with him. When in 1919 I reached Berlin, Bhupendranath Dutta was the only 
original member of the war-time Indian Revolutionary Committee living there. All the others had dispersed. Virendranath Chattopadhyaya himself had gone to Stockholm to plead the 
case of India's independence in the International Socialist Conference there. 

Feeling that the Indian revolutionaries from Berlin were not very kindly disposed towards me, I left them alone so as to obviate the impression that I was trying to influence them or to stand in the way of whatever plan they might have had. But I could not help being puzzled and pained when most of them would not even speak to me. It seemed they had the entirely groundless misgiving that I might stand in their way to seeing various Russian leaders and plead their case. Curiously enough, they were very eager to see Chicherin in the first place. He was still Commissar of Foreign Affairs, but wielded no great political influence. Moreover, he had just received the British note about the activities of Indian revolutionaries in Central Asia and naturally did not think that it would be very wise to receive well-known Indian revolutionaries in Moscow. Nevertheless, as a polite man, not willing to offend anybody's feelings, he did have a short meeting with a few of the Indian revolutionaries. It seems the latter were disappointed with the meeting. Then they demanded an interview with Lenin himself. They made a great secret of the move, most probably believing that I might stand in their way. But 1 got the news from Lenin himself. He telephoned to me and asked me to come and see him. He enquired about the Indian revolutionaries who had come to Moscow, and if it was necessary for him to see them. If they had come to discuss any plan of revolutionary work in India, they should address themselves 
to the Communist International. Lenin was surprised to hear that the Indian revolutionaries were not at all well disposed towards me. Nevertheless, I suggested that he should see them 
and hear what they had to say. Lenin remarked that I was in a minority of one against fourteen. I replied that he knew that I did not claim to represent anybody but myself So, as far as I was concerned, there was no conflict between the Indian revolutionaries and myself. Lenin enquired if 1 had discussed matters with them, and was surprised to hear that they would not even speak to me. Evidently in exasperation he sat back in his chair and said : "Well, select three of them to come and see me." I told him that 1 could not do that, he would have to contact them directly. 

In the next days there was a great flutter in the Indian delegation. Lenin had agreed to grant an interview. The Indian revolutionaries had been informed that Lenin would receive 
three of their representatives chosen by themselves. There were differences as regards the choice. Everybody considered himself to be more entitled to the honour and privilege than the others. I could get all this information through Nalini Gupta, the only one who did not share the general hostile attitude towards me. He was also the only one among the Indian revolutionaries in Europe who maintained some connection with the revolutionary organisations in India by frequently travelling back and forth secretly. He had met some of my friends in India and learned from them about the mission with which I bad gone abroad in the beginning of the War. During his last visit to India shortly before he came to Moscow, he was instructed to contact me. So from the very beginning my relation with him was of mutual trust and confidence. He gave me the information that, although among the Indian revolutionaries there was a dispute about the selection of the three to see Lenin, there was a general agreement about the case which was to be presented on that occasion. A long thesis was being prepared under the guidance of Chattopadhyaya and Agnes Smedley to contradict my thesis adopted by the Second World Congress of the Communist International the year before. Luhani, a North-Bengal Muslim, who had come to Britain to study law, was a clever man and an accomplished speaker. But not being one of the senior members of the Berlin group, he was not chosen as one of the representatives to see Lenin. The thesis to be presented by the 
representatives, however, was drafted by him. The others could not prepare a well argued document. 

Agnes Smedley, backed by Chattopadhyaya, wanted to be one of the representatives to see Lenin. Her claim was opposed by all the rest of the Indians. Finally. Chatto and Dutta, as the senior-most members, were chosen by general consent. I have forgotten who was the third one; most probably it was Knankhoje, who was chosen to obviate the allegation that the delegation was purely Bengali.

Having given them a polite and patient hearing, Lenin advised the representatives of the Indian revolutionaries to see the Secretary of the Communist International, and remarked that the Soviet Government could not actively take part in any plan for promoting revolution in other countries. The Indian revolutionary representatives returned from the coveted interview thoroughly disappointed and even angry. Dutta blurted out that Indian revolutionaries could expect no help from the Bolsheviks because they were eager to make peace with British Imperialism.

However, they saw Radek, who was then General Secretary of the Communist International. When they came to his office, I was in another room in the same building. In their presence Radek spoke to me on telephone. I begged to be excused with the remark that he would presently learn why I could not come. Lenin had passed on the thesis submitted by the Indian revolutionaries to Radek. He informed his visitors that in its activities to help the national movements in colonies, the Communist International was bound by the thesis of the Second World Congress. But, he added, if the new Indian comrades disagreed with that thesis and wanted the Communist International to alter its attitude and policy, they would have an opportunity in the near future, when the Third World Congress would meet; the Indian revolutionaries could stay on and attend the World Congress, not of course as delegates with votes, but as visitors. But if they submitted their thesis, the Secretariate of the Communist International would recommend its consideration by the World Congress. 

The Indian revolutionaries were impatient. They would not waste time in Moscow. They were eager to return to active work which had been interrupted after the War. They had come to Moscow expecting to receive help so that they could go back to West Europe and resume revolutionary activities. Radek informed the Indian revolutionaries that the Second Congress 
of the Communist International had set up its Central Asiatic Bureau as the instrument to promote revolutionary activities in the countries of the East. Pending any new decision all plans of revolutionary activities in India should be prepared in consultation with the Central Asiatic Bureau of the Communist International. Radek informed the Indian revolutionaries that I was a member of that Bureau and had just come to Moscow. He advised them to get in touch with me and discuss their plans. 

The meeting with Radek was even more disappointing than that with Lenin. In order to assuage the feelings of the Indian comrades Radek promised to ask the Executive Committee of the Communist International to set up a small commission to hear the case of the Indian delegation and to investigate the whole situation. But that did not satisfy all, and soon thereafter, most of the members of the delegation left Moscow, one by one. Chattopadhyaya. Agnes Smedley, Bhupendranath Dutta, Luhani, Nalini Gupta and a few others stayed behind. 

The Commission to hear the Indian revolutionaries and to examine the Indian situation was composed of August Thalheimer, the leader of the German Communist Party, Tom Quelch of the British Communist Party and Borodin. Chatto was the obvious leader of the Indian delegation. But he was a poor speaker, and Agnes Smedley was anxious to deputise for him. But a non-Indian would not be the right person to open the Indian case, which was done by Luhani. He gave a very good performance. After he had finished, Thalheimer enquired whether the new Indian comrades had any objection to work in co-operation with me. On enquiry, I frankly said that I would be only too glad to have the co-operation of the newcomers. I further added that 1 did not claim to represent India. If the new Indian comrades would agree on a programme of work, and decide to stay in Moscow to take over the responsibility of guiding activities, I should place myself at their disposal. That brought Chatto to his feet. With great indignation he interjected : "We have nothing against you, but we cannot have anything to do with you so long as you are associated with a known spy who has been responsible for the death of many revolutionaries in India." The Commission was taken aback. Borodin suggested that Comrade Chattopadhyaya should be a little more explicit about his allegation, if he wanted it to be taken seriously. In any case, who was the British spy he had just mentioned? Chatto signalled Luhani to answer the question. The accused was Abani Mukherjee, and the allegation was that, on his way back to India from Japan in 1916, he was arrested at Singapore and imprisoned. He did not escape from prison, as he had pretended, but was released by the British police because he had given out information about the underground revolutionary movement in India. On his information, a number of people were arrested in India and sentenced to death and long terms of imprisonment. 

In reply, I informed the Commission under what circumstances I came to know Abani Mukherjee and said that his behaviour had also made me suspicious; but as long as there was no evidence to bear out the serious allegation against him, it would not be fair to penalise him, and the penalty would be the maximum, if I withdraw my protection. I would not take such a responsibility merely on vague suspicions. 

In order to put an end to the unpleasant subject, which could not be settled there and then, Thalheimer suggested that we should revert to the discussion of any political differences the 
new Indian comrades might have had with me. On behalf of the delegation, Luhani replied that they disapproved of the formation of the Indian Communist Party in Tashkent and demanded its dissolution as the condition for any co-operation with me. 1 again explained the situation which was forced on me, and pleaded that the Communist Party of India was formed on the initiative of a number of others who would certainly not agree to the dissolution of the party, even if I recommended it. But the Indian delegation was equally adamant also on this 
question. 

Chattopadyaya, on behalf of the delegation, gave an ultimatum to the Commission. If their demands were not accepted, they would forthwith leave Moscow, fully convinced that Indian revolutionaries could not count on any help there.
