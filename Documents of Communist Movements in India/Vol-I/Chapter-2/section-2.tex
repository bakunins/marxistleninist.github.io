\section{The Emigrant Section of the Communist Party OF India and the Communist International*}

The muhajir youths who, on Manabendranath Roy’s inspiration, came to Moscow in different groups, entered the ‘Communist University of the Toiling East’ immediately on its foundation. The University was founded on April 21, 1921. Many of them joined the Communist Party of India after their arrival in Moscow. The emigrant section of the Communist Party of India was granted recognition by the Communist International in 1921. There are instances of Communist Parties, formed with a very few members, becoming affiliated to the Communist International . The Communist Party of China held its first Congress with only twelve delegates, representing about fifty members. The delegate of the Communist International also was present at the Congress of the Chinese Communist Party, affiliated lo it. 


\footnote{*Excerpts from Muzaffar Ahmad’s Memoirs : "Myself and the Communist Party of India", Pages 57-70 }


In the Communist Party in India, there were — perhaps there are still now — members who would not believe that the Party was founded abroad. How could they, therefore, reconcile themselves to the fact of its becoming affiliated to the Communist International. These members of the Communist Party of India were still under the spell of nationalism. It is true that Sripad Amrit Dange, one of the first batch of members to join the Party in the 1920’s, accepted that the Party was founded abroad; but he too refused to acknowledge that the Party had been affiliated to the Communist International. He sent this opinion of his in writing to me after the publication of my book The Communist Party of India and Its Formation Abroad. I am presenting here some facts and proofs regarding the affiliation of the emigrant section of the Communist Party of India to the Communist International in 1921.\\ 

{\textbf{I}}

Some Indian nationalist revolutionaries, led by Virendranath Chattopadhyaya, came from Western Europe to Moscow at the invitation of the Communist International. They were :\\

\indent(i) Virendranath Chattopadhyaya \\
\indent(ii) Bhupendranath Datta (not a Ph.D. yet) \\
\indent(iii) Birendranath Dasgupta \\
\indent(iv) Syed Abdul Wahid \\
\indent(v) Prof. Pandurang Sadashiva Khankhoje \\
\indent(vi) Herambalal Gupta \\
\indent(vii) Ghulam Ambiya Khan Luhani \\
\indent(viii) Agnes Smedley \\
\indent(xi) Nalini Gupta \\

During World War I all except the last-named three had come to an understanding with imperialist Germany. It had been agreed that imperialist Germany would supply the Indian revolutionaries liberally with arms and money with the help of which the Indian revolutionaries would organize widespread uprisings in India. Imperialist Germany helped the Indian revolutionaries with money, but I do not know whether this help was liberal or not. However, it is a fact that Germany could not supply arms. The Indian revolutionaries were grateful to Germany for helping them with money. We learn from the autobiography of Raja Mahendra Pratap that the Indian revolutionaries received help from the German Government even after the fall of the Kaiser Government. 

Anyway, the Indian revolutionaries, led by Virendranath Chattopadhyaya, went to Moscow to negotiate with the Communist International. As a nationalist revolutionary, Chattopadhyay was not in favour of a Communist Party being formed in India at that time. What is strange is that Chattopadhyaya himself was then a member of the Anarchist Party. Before a Commission, appointed by the Communist International, Chattopadhyaya proposed ihat a Revolutionary Board be set up to carry on work in India through its agency and that the Communist Party — if it is to be formed at all — be formed only after the British had been driven out. Chattopadhyaya was a well-educated, erudite person, but, it seems, he was a bad speaker. He, therefore, submitted what he had to say before the Commission through Ghulam Ambiya Khan Luhani, a finished speaker and a good writer. 

It is with a view to making things clear for everybody that I have tried so long to give some preliminary facts. Now let me state my main point in the words of Dr Bhupendranath Datta : 

“I want to state here that one morning some days ago it appeared suddenly in a Moscow newspaper that an Indian Communist Party had been formed and become affiliated to the Communist International. Who were the members of this Communist Party ? Roy with his wife, Mukhopadhyaya with his wife, and the muhajir youths. Talking of this Party at a meeting even before the Second Commission had begun its sittings, Luhani said, 'It is a bogus party’ . Again, while reading their thesis during the second sitting of the Commission, Luhani observed, \textit{‘Let the name of this party be struck off the rolls of the Third International'} and that help to the Indian revolutionary movement be given through their projected Revolutionary Board.”* (italics mine) 

This extract proves that the emigrant section of the Communist Party of India became affiliated to the Communist International at that time. What should be noted here is that it was Ghulam Ambiya Khan Luhani who presented before a Commission, appointed by the Communist International, the demands of the nationalist revolutionaries who had come from Berlin. 

There is no reference at all to dates in Or Bhupendranath Datta’s writings. However, it can be seen that Dr Datta and his friends came to Moscow early in 1921. I thank Prof. Khankhoje, for he has at least written that they were in Moscow for three months. Taking everything into account, we find that they left the Soviet Union even before the commencement of the Third Congress of the Communist International. The Third Congress started on June 22, 1921, and concluded on July 12, 1921. 

Dr Datta further says that on the question of forming the Communist Party of India, he, Syed Abdul Wahid and Birendranath Dasgupta were not unanimous with Virendranath Chattopadhyaya, the leader of their group. Dr Datta held that the right also to form a Communist Party was exclusively theirs, and they had already written to India in this matter. Who were these upstart muhajir youths to form the Communist Party of India abroad? But. even after a lot of investigations inside India, we have not been able to find out who were the persons to whom Dr Datta here wrote letters with the purpose of forming a Communist Party. About this, neither did he say anything to us nor did he write anything in his book. Dr Datta returned lo India in 1925 and died in 1961 : he had time enough.

Agnes Smedley was an American woman and a friend of India. She also held anarchist views. She married Virendranath Chattopadhyaya. It is not necessary to give any account of Nalini Gupta here. I shall discuss him in detail later.\\

{\textbf{II}}\\

The emigrant section of the Indian Communist Party extended its activites to Germany, where the first organ of the Party was brought out on May 15, 1922. Its name was the Vanguard of the Indian Independence. Needless to say, this first fortnightly organ of our Party was published in English language. The paper was edited mainly by Manabendranath Roy and his first wife, Evelan Trent Roy sometimes contributed articles to Indian journals under the pen-name of Shanti Devi. Packets of the Vanguard of the Indian Independence were received at different addresses supplied by us, and we distributed them among different persons. We would also put copies of the paper into the letter-boxes at certain addresses in Calcutta. Further, only individual copies were sent to certain addresses. This was a comparatively safe method to ensure delivery. The paper did not bear the declaration the it was the organ of the Communist Party of India, a branch of the Communist International. When we realized that the police had become very much aware of the existence of the Vanguard of the Indian Independence and had also started seizing packets bearing certain addresses, I wrote to Manabendranath Roy, “Now it is time to change the name of the paper. It may make things a bit easier.” I do not know whether anyone else from any other province made the same suggestion, but Roy wrote in reply that he would change the name of the paper. Thereafter, the paper was named the Advance Guard. The Advance Guard also did not carry any declaration that it was the organ of the Communist Party of India, a branch of the Communist International. But we found some time later that, like its predecessor, the Advance Guard also had roused the suspicion of the police. It was then decided to revive the former name of the paper, but it was discovered that Dr Datta and others, i.e. the nationalist revolutionaries who had returned to Germany from Moscow had in the mean-time taken possession of a half of the former name and had themselves brought out one named Indian Independence with Prof. Binoy Kumar Sarkar as editor. 

The Vanguard of the Indian Indepenence and the Advance Guard between them covered a period of one year. The first issue of the second year appeared on May 15, 1923, as simply the Vanguard. There was no hide-and-seek affair this time. In clear language it was stated that the Vanguard was the organ of the Communist Party of India, a section of the Communist International. Lest anyone should entertain any doubt about 
this, I am reproducing here the block made from the photostat of the first issue of the second year (May 15, 1923) of the Vanguard. The message, sent by the Presidium of the Communist International to the Vanguard on its first anniversary, was also printed on the first page of the issue. Can there be any evidence more incontrovertible even than this? 

I had not seen this issue or the subsequent issues of the Vanguard previously, for I was arrested and imprisoned on May 17, 1923. But Sripad Amrit Dange and Sachchidananda Vishnu Ghate were then free, and they saw and read this issue of the Vanguard of May 15, 1923, and the subsequent issues. Dange was not arrested before March, 1924. After Dange’ s arrest, i. e. from 1924, Ghate became active in the Party. Yet 
both supplied wrong information on the basis of which the Right Communists of India observed the fortieth anniversary of the Party in 1966. They have taken 1925 — the year of the Kanpur Communist Conference — as the year of the foundation of the Party. 

Although the Vanguard and other papers had from the beginning been printed in Germany, the names of different cities of India also appeared in the paper. The names of Bombay, Calcutta and Madras appear in the block printed here. There is no doubt that the name of Lahore was usually printed in the paper. As far as I can remember, the name of Kanpur also appeared some time or other. We had contacts more or less with all these cities. \\

\textbf{III}\\

The historic Seventh Plenum of the Communist International was held in Moscow from November 22 to December 12, 1926. The stenographic report in Russian of this Plenum was published in two volumes in Russia. The report is entitled in Russian Puti Mirovoi revoliutsii (Paths of World Revolution). Drawing upon the account given on p. 8 of vol. I of World Revolution, Robert C. North and Xenia have written in the book, edited by them, Af. N. Roy’s Mission to China: Communist- Kuomintang Split of 1927 : 

"During the first session of the Seventh Plenum, November 22, 1926, Roy as the representative of the Communist Party of India had been elected to the Presidium of the Comintern and to the Chinese Commission.” (p. 43) 

It is written in quite clear language that it was as the representative of the Communist Party of India that Roy was elected to the Presidium of the Communist International and 
to the Chinese Commission. This Party was the Communist Party of India founded abroad. No application for affiliation of the Communist Party was sent from within India to the office of the Communist International in 1926. \\

\textbf{IV}\\

On November 30, 1927, M. N. Roy wrote a long letter on behalf of the Communist International to the Central Committees of both the Communist Party of India and the Workers’ and Peasants’ Party. The letter became known in India as ‘Assembly Letter’. It was stated in the letter that every Communist Party must become affiliated to the Communist International. But from India no application for affiliation to the Comintern was ever sent by the Communist Party. In fact, “Up till now Communist International has acted upon the affiliation of the Emigrant Section of the Communist Party of India.” 

It is now proved that the Communist Party ,of India was established on October 17, 1920,. in the city of Tashkent, capital of the present Republic of Uzbekistan. 

It is also proved beyond dispute that the emigrant section of the Communist Party of India was affiliated to the Communist International. 

In his Memoirs M. N. Roy writes : “To challenge my representativeness was pointless. I did not claim to represent anybody but myself, and held my position in the International as an individual (p. 301)”. The facts I have presented in the foregoing pages, the extracts I have reproduced from the book M. N. Roy’s Mission to China and Roy’s letter— all taken together — go to prove that Roy’s pretension was entirely false. He had represented the Mexican Communist Party in the Second Congress of the Communist International. Thereafter, he had always represented the emigrant section of the Communist Party of India. None but representatives of the Communist Parties of different countries could hold any office in the Communist International. M. N. Roy was elected to the Presidium of the Communist International as representative of the Communist Party of India in the same way as Stalin, Bukharin and Manuilsky were elected to the Presidium as representatives of the Communist Party of the Soviet Union. Even Lenin had he been alive then, would have had to get elected in the same way. 


\subsection{Who are The Founder-members of the Party?} 

I hope that I have been able to clarify a number of points for future writers of the history of the Communist Party of India. Now that these points are clarified, those of us who are called the founder-members of the Communist Party of India can no longer find, I believe, much strength behind that claim of ours. The real founder-members of the Communist Party of India are those who joined the Party in Tashkent and Moscow in 1920-21. We can never forget the fact that the Communist International granted affiliation to the emigrant section of the Communist Party of India in 1921. Some of the members of the emigrant section of the Party went through great sufferings and hardships in order to return to India and, even after undergoing imprisonment here, did not give up serving the Party. It was with us that they worked. We could have, if we wanted, regarded our Party as an affiliate of the Communist International, but the idea did not occur to us at that time. 

Towards the close of 1921, we also began to move to some extent. We became quite active in 1922. Prison life also began for us in 1923. Everybody knows that prison life was inevitable for the revolutionary workers of India. But notwithstanding all this, can we claim to be the founder-members of the Communist Party of India? As for myself, I cannot find much conviction
to make this claim; we can, however, claim, that we only paved the way to the building of the Party. 

\subsection{A few words about Muhammad Shafiq} 

1 have already said that Muhammad Shafiq was elected the first Secretary of the emigrant section of the Communist Party of India and was — by virtue of his position — the first Secretary of the Communist Party of India, for no Communist Party had been formed in India in 1920-21. Everyone will naturally ask who this Muhammad Shafiq was. If I want to say anything about him, however insignificant, I shall have to depend on secret police reports and the reports of the proceedings at the court of law. I never knew Shafiq personally. My account, therefore, is based mainly on the material I have collected from police reports and court documents. 

Muhammad Shafiq was a resident of Akora in the tehsil Nowshera of Peshawar. In 1919, he served as a clerk in the irrigation office at Peshawar. In his judgment, delivered on April 4, 1 924, in the case against Shafiq, Mr George Connor, Sessions Judge of Peshawar, observed that during the anti Rowlatt Act movement Shafiq went to Kabul in May, 1919, without giving any notice to his office. Again it was in May, 1919, that Afghanistan attacked British India; and as a result of this war (the Third Afghan War) Afghanistan won the status of a fully independent state. Shafiq must have gone to Kabul as a muhajir (self-exiled), but it should be remembered that he did not belong to the muhajirs of 1920. The Hijrat (flight from persecution) movement was yet to start when he left India. The Hijrat movement of 1920 originated from the Khilafat agitation and only the Muslims participated in it - but the movement against the Rowlatt Act was a broad-based political one, cutting across the barriers of nationality and religion. The tragic incidents, which took place in Jalianwalla
Bagh in Amritsar on April 13. 1919, arose from the anti- Rowlatt Act movement, which the British Government in India wanted to suppress ruthlessly. 

In course of his judgment the Sessions Judge observed, “His intention in going there (Kabul) was soon made apparent for he at once got into touch with Bolshevik agents who were then at Kabul.” If the Judge’s observation is correct, then Shafiq, possibly, got into touch with the Bolsheviks before reaching Kabul. 


On December 10, 1923-after his arrest-Shafiq made a statement before Khan Muhammad, Additional Magistrate of Peshawar. To make a statement like this before a Magistrate IS a sign of great weakness; but Shafiq made it. A man making such a statement is never entirely truthful; on the he fabricates a lot of things. Therefore, to get at the truth one has to read the statement between the lines. Shafiq says that it was in Kabul he met Raja Mahendra Pratap, Abdur Rab and Acharya: they had just returned from Russia. To the muhajirs who were in Kabul at the time, either Abdur Rab or both Rab and Acharya said that the Russian Government looked upon the Indian muhajirs with respect and also helped them. As soon as Shafiq heard this, he started for Russia via Mazar-i-Sharif. He was accompanied by Ahmad Hassan, and Abdul Majid and Muhammad Sadiq of Kohat. Ahmad Hassan was probably Muhammad Ali alias Khushi Muhammad. Shafiq's statement points to the existence of factionalism among the Indian muhajirs in Kabul. Shafiq belonged to Moulana Obeidullah Sindhi's group. At the time of his departure for Russia, he had seen Abdur Rab having a quarrel with Moulana Obeidullah over extremely petty personal interests. The subsequent activites of Abdur Rab showed him to be a cantankerous and factious short of person. Shafiq and the others reached Tashkent safely. It seems that they reached Tashkent some time towards the end of 1919. As it was not possible to stay idle there, they brought out a paper named Zamindar in Urdu and Persian. In the Punjab and the North-Western Frontier Province Zamindar means ‘peasant’. Only one issue of the paper came out. When making his statement before the Magistrate, Shafiq, probably, thought that the single issue of Zamindar had not certainly reached India. He, therefore, said that it was on the ideals of Islam that the paper had been based. The issue of Zamindar was not just another exhibit in the case; it was exhibit No. 2. The judgment contained many extracts from the issue, but there was no evidence of anything Islamic in them. Shafiq, perhaps, spoke of the Islamic basis of the paper in order to minimize his offence. Some three months (three, according to Shafiq’s estimate, which, however, does not tally with my calculation) after Shafiq’ s arrival in Tashkent, Abdur Rab and Prativadi Acharya came there with a party of thirty muhajirs. The mass Hijrat movement of 1920 had already started in India. 

Some days later Shafiq received the invitation to attend the Second Congress of the Communist International, which was to be held from July 19 to August 7, 1920, With the purpose of minimizing his offence, Shafiq told the Judge that they had gone to Moscow under orders from the Soviet Government. Factionalism had already started in Tashkent also. Shafiq states, "Acharya went to Moscow for the Second Congress on behalf of the revolutionary committee of Abdur Rab, and I went on behalf of our group”. The expression ‘our group’, perhaps, means the group opposed to Abdur Rab. It was in Moscow that Shafiq first made the acquaintance 
of Manabendranath Roy and Abani Mukherjee, Roy, Mukherjee and Acharya attended the Congress as delegates; Shafiq had an observer’s ticket. It is difficult for me to say whether Shafiq was really an observer or told the story of the observer’s ticket in order to minimize his offence. 

However, Roy alone had the right both to participate in the deliberations and to vote. The other Indians could only participate in the deliberations. 

One thing, however, strikes me as very strange. Why did Manabendranath make no reference to Prativadi Acharya and Shafiq in his Memoirs! How could Roy forget that both Prativadi Acharya and Muhammad Shafiq had attended the Second Congress of the Communist International? I cannot believe that any revolutionary could ever forget an incident like this in his career. Roy writes that it must have been a very strange thing that he should represent Mexico and Abani Mukherjee should represent India (although without the voting right) in the Second Congress of the Comintern. The names of Prativadi Acharya and 'Shafiq should have been specially mentioned; but Roy did not mention them. 

When the question arose of sending back home the young Indian muhajir students of the Communist University of the Toiling East, attempts were made to procure by various means passport for them. Those who failed to procure passports for themselves reached India across the nearly impassable and insurmountable Pamir and the Hindu Kush. From the documents of the Moscow Conspiracy Case in Peshawar it appears that Shafiq Muhammad was very lucky in the matter of passport. What we had so long referred to as the Peshawar Conspiracy Case (1922-23) is now found to have been described as the Moscow Conspiracy Case in the relevant records. With the help of a false British passport procured for him, Shafiq returned to India by sea from a certain port in Holland. Fida All Zahid, an approver in the Moscow Conspiracy Case, said that he had heard from the Russian instructor in the Military Academy that from Europe Shafiq went to Lahore but then for fear of being arrested he fled to Kabul in panic. Abdulla Qadir Sehrai(Khan) said that Shafiz had visited India secretly several times. He said further that with the help of a false British passport Shafiq had returned to India by sea from a port in Holland in November, 1921, but he had gone to Kabul, fearing arrest. 

It seems that the latter is the correct information. In a letter sent from Kabul to M. N. Roy on August 29, 1922, there is a reference to one ‘S’. This ‘S’ probably is Shafiq. The 
author of this letter said that ‘S' was trying to deliver him up in the hands of A.G. (the Afghan Government). Was Muhammad Ali the author of the letter? If all this be true, then it must be admitted that Shafiq had degenerated. 

What followed was that towards the end of 1922, when the Afghan Government asked the Indian revolutionaries to leave Afghanistan, Shafiq also had to leave with them. Many of the Indian revolutionaries went to Moscow, but Shafiq no longer had the face to go there. He went to 
Seestan, where he gave himself up to the British Consul, praying to be sent to India. I cannot understand why Shafiq went to Seestan to surrender. Seestan today is not a big province; a third of it is in Iran, the rest in Afghanistan. Shafiq, however, was arrested as soon as he reached the Indian border, and proceedings (the Second Communist Conspiracy Case) were started against him under sec. 121-A of the Indian Penal Code. George Connor, Sessions Judge of Peshawar, sentenced Shafiq to three years’ rigorous imprisonment on April 4, 1924. 

I shall have to write about Muhammad Shafiq once again in connection with Shaukat Usmani. 


