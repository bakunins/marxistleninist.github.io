\section{Early Contacts of the Indian Revolutionaries with the Leaders of Bolshevik Revolution in Russia*}
\textbf{1919}\\

\textsc{Indian Revolutionaries in Russia:} The following report was sent out from the Wireless Station of Bolshevik Government in the beginning of December: 

\footnote{* Excerpts from ; "Communism in India : Unpublished Documents 1919-1924 . Edited by ; \textsc{Subodh Roy}, Pages : 1-64 [The details given here are actual 
records of the Files of the Intelligence Bureau. Home (Political) Department. 
Government of India, compiled by Subodh Roy. from the Files maintained in the 
National Archives, New Delhi.) }

"On November 25th Indian Delegation handed a memorandum to Sverdloff. President of the Central Executive Committee of the Soviets, in the name of the peoples of India. This memorandum gives an exposition of the long martyrdom of India under the yoke of England, which, although it has given itself the title of a democratic country, keeps a population of 325,000,000 of the inhabitants in slavery. The Russian Revolution produced an enormous psychological impression on the Indian people. In spite of England's efforts the principle of self-determination for the nations has penetrated into India, whose events have taken such a turn that the English Government was compelled on August 20th, 1917, to formulate in Parliament two principles of their Indian policy. Indian delegates wanted to explain the situation to the English public, but they could not obtain a permit to go to England. In the U.S.A. and in France, Indian delegates were imprisoned. They were driven out from Japan, Switzerland and Denmark under the pressure of the English diplomats. 

“The memorandum further says that the liberty of the world will be in danger as long as the imperialists' and capitalists’ power of England exists, which power is founded upon the slavery of a fifth part of the population of the globe. The memorandum ends with an expression of confidence that the days of England are numbered, that the Indians will rise and drive out the foreign domination, and that free Russia will stretch out a fraternal hand to them." 

\textsc{Bolshevism}. The following note on — by a military officer who has made a special study of Russia and the Russian situation will, I hope, interest the readers of this report. In this connection I may mention that the Daily Mail correspondent at Heisingfers telegraphed as follows to the Daily Mail, London on 18th January last: 

"The Indian Centralisation Committee, which is now working at Petrograd under the Bolsheviks, is composed of the same members as the Berlin Indian Committee. It is stated by the Petrograd Journal Krassanja Gazeta in the special number devoted to British India and to formation of Indian centralisation Committee at Petrograd, that a large number of Indian Bolshevik propagandist have already been sent to India and that 
the power of Universal Bolshevism will soon be made known to the British Empire." 

\textsc{Narendra Bhattacharjya:} who under the name of C.A. Martin and M.N. Roy played a leading part in the German plots against India has been living in Mexico for some time and appears to be carrying on anti-British propaganda in Spanish. A letter recently intercepted in the American Censorship contained a pamphlet entitled \textit{La Voz de la India} (The voice of India) which bore the name of M. N. Roy as publisher. The pamphlet contained the usual calumnies of British rule in India and criticised a pro-ally pamphlet called \textit{El Despartar de la India} (The Awakening of India), also published in Mexico. 

From another source it is reported that Bhattacharjya, H. L. Gupta and the other Indians in Mexico have formed a League of Friends of India with the object of obtaining support for the Indian revolutionary movement among the South American republics. They have also addressed a letter to the diplomatic representatives in Mexico of several countries asking them through their governments to present to 
the Peace Conference the petition of the League for the release of India from British domination. \\

\textsc{Bolshevism and India}\\

A Bolshevik agent named Carl Sandberg who had come to the United States from Christinia, was recently arrested by the American authorities. A considerable quantity of Bolshevist propagandist literature was found in his possession, some of it relating to India. Among it was a copy of a book issued by the Bolshevik government entitled "India for the Indians", which consisted of a collection of extracts from Russian 
official documents relating to India. The following passages are taken from the introduction: 

"In closing there will be pointed out the role which the Russian Revolution can on its part play for the Indian Revolution on the ground of mutual struggle with world imperialism, which has assumed in England with regard to India such unusual forms of rapacious exploitation. 

"For us Russians, who are ourselves threatened with the fatal danger of becoming a colony of Western Europe or may be of American or Japanese imperialism, it is very important to obtain in the face of the oppressed, and in many ways similar to us in India, a natural ally in India, a natural ally in the cause of the struggle with a mutual enemy. 

"Then let this collection serve our Eastern friends for the present as a first modicum of all our sympathy to the much suffering Indian people, as a certain pledge that our revolutionary paths in the near future will joyfully meet not only on the ground of a struggle for mutual liberation from a foreign sovereignty but also on the broader basis of class struggle and social construction." 

The book of course is in Russian and the translation of these passages was done in America. I take no responsibility for the grammar. 

Several British and French subjects who have recently returned from Moscow state that there is an Indian Lawyer these (sic) named “Servadi” who is on intimate terms with Lenin and is running the India Department of the Bolshevik Ministry of propaganda. This obviously refers to Hassan Shahid Suhrawardy, a member of well-known Calcutta family, who obtained permission from the British Government to go to Russia from England in 1916. It is said that he has several Indian assistants working under him at Moscow but their names has not yet been ascertained. \\

\textsc{Berlin Committee}\\ 

This is at present inactive and the German Government does not pay much attention to it. The German Foreign office continues to pay for the establishment of the Committee, and will pay 400 marks a month to every Indian Nationalist residing in Germany until peace is signed or free communication with India opened again. \\

\textsc{Russian Committee in Moscow}\\ 

This, on the contrary, is showing much activity and is working to organise a new Russo-Indian Mission to Afghanistan. 

It is reported—though it could not appear to be likely-Dr. Hafiz and Umrao Singh Majithia are in Moscow; in any event much mystery is made in Berlin as to the present whereabouts, which applies also to Sen (unidentified). 

Das Gupta has recently received a letter from Dutt (Bhupendra Nath Dutt) informing him that the chief of the Moscow Committee has arrived in Switzerland. 

All the members of this Committee are reported to have become Bolsheviks, and they all, on the suggestion of the Soviet Government, desire to turn their National Committee to Communism. Das Gupta is himself affected in this way. He states that the name of the Moscow chief has not been mentioned to him, but he has grounds for the belief that he is Umrao Singh Majithia.\\ 

\textsc{Indian Agitators Abroad }\\

\textsc{Barkutallah:} 

According to a wireless telegram from Moscow Barkatullah had an interview with Lemn on 8th May. 

\textsc{Hardayal:} It is considered by some well-informed Indians in London that Hardayal' s sudden detestation of Germany and “fancy” for England is blind. They say that any one who knows his record before he became a politician knows that he was in effect a Bolshevik in the days when Bolshevism was not known. His pamphlet on the A Conquest of the Dravidians, written about 12 years ago, is instanced as a proof of this. 
He may, it is said, easily enough dislike the late German Government and the Kaiser and his entourage on account of the way he himself was treated by the German Foreign office, but he has no reason to hate the German people. Nor does it follow that he should have come to like England, any more than Germany does, though the Soviet has over-thrown Imperialism. 

There are many who think that German penetration into India has in no sense been abandoned and “Hardayal is not a fool”. He is on the contrary remarkably clever. Being a Delhi man he is able to exercise power and influence equally between Hindu and Muhammadan students; and as an Indian Nationalist he does not owe allegience to anyone— he would use Russia, or Germany, or England to gain bis object. He is believed to be in close touch with Russian Bolshevism in Stockholm; he knows the channels of communication from England and may be expected to arrange to correspond with Russians from India if he should be allowed to return there. \\

\textsc{Bolshevik Propaganda}\\ 

In paragraph 4 of my Weekly Report dated 3 1 st March, 1919 it was stated that an Indian (Hassan Shahid Suhrawardy) was running the India Department of the Bolshevik Ministry of Propaganda. This is a Department of the Bolshevik Foreign office and is said to include men and women of every race. 

Further reports have been received to the effect that Turkistan has been chosen as the main base for oriental propaganda. A special mission is sr.id to have been sent to Tashkent for this purpo.se. A former Russian Consular Officer in Persia, one Bravin has been put in charge of this mission, and has been given full powers, large sums of money, and much literature and pamphlets. 

A report dated 19th April, 1919, stated that Bravin accompanied by another Bolshevik emissary named Batavin has gone from Tashkent to Bokhara intending to proceed into Afghanistan. A quantity of propaganda specially directed against the British rule in India, was reported in March this year to have been sent into the Pamirs with a view to its being smuggled eventually into India through Chinese territory. \\

\textsc{Indian Revolutionaries Abroad} \\

\textsc{Proclamation of the Provisional Government of India:} A lithographed circular letter has recently been found on the Frontier which purports to emanate from Provisional Governinent of India. It bears the signature of Obeidulla, Wazir an Zafar Hussain, Secretary to the Provisional Government of India. Obeidulla is a Sikh convert to Islam and was the signatory of the “Silk letters.” He is the “officiating Salar of Kabul” in the “Army of God.” 

Zaffar Hussian was one of the Lahore students who fled to the Frontier in February 1915. He is a “Lieutenant Colonel” in the “Army of God.” A translation of the letter is printed below; 

“You have read the news of the Provisional Government of India in the Rowlatt Sedition Committee Report. This Government has been instituted in order to establish a better government in place of the present treacherous, usurping and tyrant Government. Your Provisional Government has been continuously struggling for the last four years. As soon as you determined to refuse to accept the oppressive law, the Provisional Government, too, succeeded in obtaining help then and there. 

The Provisional Government has entered into a compact with the invading forces. Hence you should not destroy your real interest by fighting against them, but kill the English in every possible way, don't help them with men and money, and continue to destroy rails and telegraph wires. 

Earn peace at the hands of the attacking armies and obtain sanads of honour by supplying them with provisions. 

The attacking army grants peace to every Indian irrespective of caste and creed. The life and honour of every Indian is safe. He who will stand against them will alone be killed or disgraced. 

May God guide our brethren to tread on the right path.” 

\begin{center}
Sd/- Obeidulla. \\
Wazir of the Provisional Government of India. \\

Zafar Hussain\\

Secretary, Provisional Government of India. \\

Delhi. \\
\end{center}

\textsc{Russian Committee:} The India Committee in Moscow is busy in Russian Turkestan and Bokhara. It is said that there are about 60,000 Indians residing in Turkestan. A successful Bolshevik propaganda is carried on there through the Indian Committee. Several Indians in Turkestan have already joined the Committee in addition to six Indians from Afghanistan and India who have arrived. \\

\textsc{Propaganda in the East}\\

A report dated 15th April, 1919 stated that there were many indications that the Bolshevik authorities have a special organisation for the encouragement of revolutionary movements in the orient and that they are engaged in turning out propagandist literature in Indian and other Eastern languages. It added that there was little doubt that many of the Indian revolutionaries and anarchists who formerly composed the 
Indian Committee under the German Foreign office have now taken service in Moscow. 

\textsc{A Bolshevist Muhammadan Agent:} It is reported from Helsengfors on 5th April. 1919 that Muhammad Bak Hajilachet corresponds with Bombay and is engaged in Bolshevik propaganda among the Mussulman population of India. The Training of Agitators : A report received in London on 25th April, 1919 states that very many agitators have been prepared for service in the East. A large number of these are to try to reach Tashkent and Persia. It is reported that a branch of the “League of the Eastern Freedom” is already working in Tashkent Natives are being trained as agitators. The "League of Eastern Freedom" has as its object the spread Of Bolshevism among the people of Asia. With this end in Courses” have been arranged in Moscow in Mussulman Workman's Hall (Asadoulev's house, Bolshoi Tartarski Street). Lectures are delivered on:

(1) Economics of the East, by Suetloff.

(2) .................................. 

(3) India, by C. D. Mstislavsky. 

(4) Imperialism in the East, by V. Kriajin. 

(5) ..................................

(6) Socialism in the East.'by' Troanovsky. 

(7) Revolntion and the Mussalmans, by Cysoupoff. 

(8) ..................................

(9) ..................................

(10) .................................

In addition to the above, periodical lectures on other subjects are delivered. The temporary bureau of the “League of the Eastern Freedom” is in Sivtsefvrajka Street, House 14. \\

\textbf{1920}\\

\textsc{Defensive Measures Proposed Against Bolshevism. Appointment of a Special Officer in Each Province to Deal With Bolshevik Propaganda}

Telegram P. No. M.D.O — 2616, dated 28th Nov. 1919. 
From : General Malleson, Meshed. 
To : The Chief of General Staff, Delhi. 
Priority : The following is a report from a British news writer, regarding the Bolshevik Mission. 

1. It is difficult to give the exact composition of Suric’s party as all intercourse with it is jealously guarded but the following is approximately correct : Suric : Russian Jew (other informants say he is an educated Kalmuck Muhammadan). A Russian Colonel formerly in Kurshk as Captain; speaks Persian and acts as interpreter. Russian doctor, Russian Secretary, Russian lawyer. Young German, Three Austrians, Fourteen Cossack, Maulavi Abdur Rab (also known as Abdur Auf), probably an Indian; said to have been in Kabul two years ago and gone thence to Bokhara. An India Rajah (Mahendra Pratap, Brahmin); said to be a converted Muhammadan; eats with Russians. Another Indian said to be a Madrasi Hindu. 
2. 
3.
4. 
5. 
6. The news writer states that he is convinced that the object of Suric is to induce the Amir to renew the war with India-and the arrival of Suric in Kabul will be followed speedily by a fresh outbreak of war. 

Telegram P., Nos. 116 — 8. Dated the 28th January, 1920.
From— His Excellency the Viceroy 
(Foreign and Political Department), Delhi. 
To His Majesty's Secretary of State for India, London. 

Anti-Bolshevik Measures in India : Please refer to my telegram. Home Department, No. 1022, dated the 18th October, work has now been commenced by officers specially appointed for counter-propaganda, coordination of intelligence, both internal and external, and organisational measures to keep Bolshevist emissaries and propaganda out of India. Conflicting announcements in Reuter’s Telegram, however, regarding policy about to be adopted by His Majesty's Government towards Bolshevists hamper them considerably. Similar embarrasment is felt by us when defining our attitude towards Afghan relations with Bolshevists and a clear statement from you of British policy towards them would be of great assistance to us.\\ 

\textsc{Notes in the Foreign and Political Department }\\

A meeting was held on the 27th January, 1920 to discuss certain matters in connection with the defensive measures against Bolshevik propaganda outlined in the Home Department Letter No. 2483 dated the 25th November, 1919. 

\begin{center}
\textsc{Present:} 
\begin{itemize}
    
    \item \textsc{Foreign Department:}
    \begin{itemize}
     \item The Hon'ble Mr. Dabbs. 
     \item Lt. Col. O'Connor.
     \item Mr. Cater. 
    \end{itemize}
    

    
   \item  Home Department:
   \begin{itemize}
     \item The Hon’ble Mr. McPherson.
     \item Lt. Col. Kaye.
     \item Mr. Corbett. 
    \end{itemize}
    
    \item General Staff Branch
    \begin{itemize}
     \item Lt. Col. Muspratt. 
     \item Maj. Lumby. 
     \item J- and P. (S.) 3716/20 
    \end{itemize}
    
    \end{itemize}

\end{center}

As a precaution, the part I have underlined [Italicised] should be treated as confidential. 
\begin{flushright}
(Intld.) H. M. 
\end{flushright}
    
\textsc{Internal}\\

\textsc{Pro-Bolshkvik Indians:} A youthful apostle of Bolshevism has recently come to notice in Bengal in the person of Durga 
Das Chatterji, a 4th year student of the Bangabasi College in Calcutta. This young man has been going about under the wing of the welLknown Liyaqat Hussein addressing meetings. Several times he has alluded to Bolshevism pointing out its advantages and asking his audience to accept it if Government failed to take immediate action in the matter of high prices and profiteering. England, he pointed out. in the present state of affairs, would never be able to save India. If the Bolsheviks attacked from within and without and the only course left for them was to accept Bolshevism which he recommended them to do. Durga Das is a well known protegee of the well known extremist Jitendra Lai Banerjee who sends him to meeting as his deputy when he cannot attend himself. 

\textsc{M. C. Rajagopai Achari:} A High Court Vakil of Madras, holding extreme views is reported to be an ardent pro-Bolshevist, his idea being to attain the revolution he desires to see by fostering labour unrest. In this programme he is said to be assisted by a certain Sukhini Narayan Iyer, a young barrister, now in Guntur, who recently returned from Ireland where he was associated with Sinn Feiners. 

These individual are being watched by the Madras Police. The former is touring the districts. 

Jethmal Parsanam (notorious Sindh agitator) recently made a speech on ‘Socialism’ at Karachi the whole trend of which, in the opinion of the reporting officer, was calculated to encourage industrial discontent, and dispose the audience favourably towards Bolshevism. Bolshevism, he said, was nothing else than hunger, seventy-five per cent of Indians were poor and must starve if the bureaucracy retained the reins of Government. 

The notorious Dr. Choitranr Gidvani supported him. 

S. P. Dave is now reported to have arrived in Bombay, unnoticed two months ago. He is stated to be living at Bhavnagar, Kathiawar. 

Chaman Lal(see list of Pro-Bolshevik Indians) has come into prominence this week. 

He is reported to have allied him.self with Miller, the ex-guard of the N. W. Railway (vide last week’s report) who IS now the head of a rapidly increasing Labour Association composed pricipally of railwaymen. To this association he has been appointed Legal Adviser. Chaman 
Lai has also allied himself in Lahore to a certain Swami Wichara, Nand, described as lecturer of the Poona branch of the Indian Home Rule League. This Swami has recently established in Lahore a branch of the League with Gawardhan Das, noticed last week for his pro-Bolshevik utterances, as President. Swami Wichara Hand’s scheme, it is said, is to obtain control over the labouring classes, form Unions, ally 
them when formed with trade unions of foreign countries and then to strike at imperialism. It is said that Chaman Lai has invited Swami Wichara Nand to Rawalpindi where it is proposed to start a branch of this League. 

Bepin Chandra Pal has renewed his anti-capitalist campaign. On 6th March speaking at the Surma Valley Conference at Sylhet on the subject of the rise in the cost of living he explained how India was being exploited by the foreign capitalist. His speech throughout can only be de- 
scribed as thinly veiled Bolshevism. 

No man. Pal holds, has a right to that which he does not produce with his own labour, be the product material or intellectual. The only hope in his opinion is to form an open alliance with British Labour, which looks upon capital as its natural enemy. 

\textsc{Known and Suspected Bolshevik Agents:} A Durani Pathan was recently found at Amritsar Station talking Bolshevism and praising the Bolsheviks, representing that if they came to India all wealth would be divided and there would be no more poor. He gave his name as Sardar Gholam Haider Khan and said he was a horse dealer and going to Bareilly. It appears that there is a man of this name resident at Kohat. Enquiries are in progress. 

\textsc{Muslim-Bolshevik Combine:} Information has reached the Allahabad C.I.D. that at the recent Bombay Khilafat Conference, Maulavi Mohammad Fakir, an Allahabad delegate, suggested to the subjects committee that owing to the recent comparison made in the British Press between Lenin and the prophet Muhammad, a resolution should be passed that it was not in the interests of Muhammadans to oppose Bolshevism in India or in any part of Asia. The resolution was disallowed but the information adds that most of the delegates present were in favour of using Bolshevism as a weapon against the British Government. 

In conversation with an officer of the Government recently, Mushier Hossein Kidwai showed that he had a very high opinion of Bolshevik strength and spoke of their “Great Citizen Army.” Bolshevik Russia, in his opinion is much more organic and therefore more powerful and dangerous than Imperial Russia. 


\textsc{An Indian Communist Manifesto:} In the issue of the Weekly Report of July 19th mention was made of a manifesto published in the Glasgow Socialist. A copy of this curious document has now been received. It is an appeal to the British to join hands with the coming proletarian revolution in India against both foreign imperialism and the sentimental nationalism which would create a bourgeois democracy of Indian exploiters. Omitting verbiage the appeal runs thus:-

The time has come for the Indian Revolutionists to make a statement of their principles in order to interest the European and American proletariat in the struggle of the Indian masses, which is rapidly becoming a fight for economic and social emancipation and the abolition of class rule. The appeal is made to the British proletariat because of their relation to revolutionary movements in countries dominated by British imperialism.

The nationalist movement in India has failed to appeal to the masses, because it strives for a bourgeoise democracy and cannot say how the masses will be benefited by the independent national existence. The emancipation of the working class lies in the social revolution and the foundation of a Communist State. Therefore the growing spirit of rebellion in the masses must be organised on the basis of the class struggle in close cooperation with the world proletarian movement.

But, because British domination deprives Indians of the elementary rights indispensable for the organisation of such a struggle, the revolutionary movement must emphasize in its programme the political liberation of the country. This does not make its final goal- a bourgeois democracy unless the native privileged class could rule and exploit the native workers in place of British Bureaucrats and Capitalist. All that the world is allowed to know of the Indian revolutionary movement is the agitation for political autonomy. This had naturally failed to enlist the sympathy of the working class in any country, which must always be indifferent to purely nationalist aspirations. 

The idea of class conscious rebellion against capitalist exploitation has been gaining ground in India, immensely stimulated by the war. The quickened industrial life, the rise in the cost of living, the employment of Indian troops overseas and the echoes of the Russian revolution, have fanned the discontent always existing in the masses. The nationalist revolutionary movement, recruited from educated youth of the middle class, tried to turn the discontent to an armed uprising against foreign rule. Since the beginning of the present century, terrorism, local insurrections, conspiracies and attempts at revolt have become more and more frequent until at least practically the whole country came under martial law. These activities did not inspire the masses with lasting enthusiasm; the leaders failed to prescribe remedies for the social and economic evils from which the workers suffer. By dynamic economic forces, which are destined to cause a proletarian revolt in every country, have grown acute in India and hence the spirit of rebellion has grown more and more mainfest among the people who are not moved by the nationalist doctrines presented by the revolutionaries. To-day there are two tendencies in the Indian movement, distinct in principles and aims. The Nationalists advocated an autonomous India and incite the masses to overthrow the foreign exploiter upon vague democratic programme or no programme at all. The real revolutionary movement stands for the economic emancipation of the workers and rests in the growing strength of a class conscious industrial proletariat and landless peasantry. This latter movement is too big for the bourgeois leaders and can only be satisfied with the Social Revolution. This manifesto is issued for those who fill the ranks of the second movement. We want the world to know that nationalism is confined to the bourgeois, but the masses are awakening to the call of the Social Revolution. 

The growth of class consciousness in the Indian proletariat was unknown to the outer world until last year, when one of the most powerful and best organised strikes in history was declared by the Indian revolutionaries. Though the Nationalists used it as a weapon against political oppression, it was really the spontaneous rebellion of the proletariat against unbearable economic exploitation. As the workers of the cotton mills owned by the native capitalists were the first to walk out it cannot be maintained that the strike was nothing more than a nationalist demonstration. 

It is known in England how this revolt of the famished workers was crushed by British imperialism. But the British working class were misled into believing that it was merely a nationalist demonstration and therefore abstained from taking definite action according to the principles of class solidarity. A simultaneous general strike would have dealt a vital blow to imperialtistic capitalism at home and abroad, but the British proletariat failed to rise to the occasion. 

The only stem taken was very weak and of a petty bourgeois nature-the progtest against the manner of crushing the revolt signed by William Lansbury and Thomas. This was not the voice of the revolutionary proletariat raised to defend the class interest.

The bourgeois nationalist movement cannot be significant to the world proletarian struggle of to the British working class, which is learning the worthlessness of mere political independence and sham representative government under capitalism. But the Indian proletarian movement is of vital interest. The tremendous strength which imperialistic capitalism derives from extensive colonial possession rich in natural resources and cheap human labour must no longer be ignored. So long as India and other subject countries remain helpless victim of capitalist exploitation and the British Capitalist is sure of his absolute mastery over millions and millions of human beasts of burden, he will be able to concede the demands of British Trade Unionists and delay the proletarian revolution which will overthrow him. In order to destroy it completely, world capitalism must be attacked simultaneously on every front, the British proletariat cannot march towards final victory unless he takes his comrades in the colonies along with him to fight the common enemy. 

The loss of the colonies might alarm orthodox trade union psychology with the threat of unemployment, by a class conscious revolutionary proletariat, aiming at the total destruction of capitalist ownership and the establishment of a Communist State cannot but welcome such a collapse of the present system since it would lead to the economic bankruptcy of capitalism-a condition necessary for its final overthrow. 

To all possible misgivings of British Comrades we declare that our aim is to prevent the establishment of a bourgeois nationalist government which would be another bulwark of capitalism. We wish to organize the growing rebelliousness of the Indian masses on the principles of class struggle, so that when the revolution comes it will be a social revolution. The idea of the proletarian revolution distinct from nationalism has come to India and is showing itself in unprecedented strikes. It is primitive and not clearly class conscious so that it sometimes is the victim of nationalist ideas. But those of the vanguards see the goal and the struggle and reject the idea of uniting the whole country under natioanalism for the sole purpose of expelling the foreigner, because they lealize that the native princes, landlords, factory owners, moneylenders, who would control the Government, would be not less oppressive than the foreigner. 'Land to the tiller’ will be our most powerful slogan, because India is an agricultural country and the majority of the population belongs to the landless peasantry. Our programme also calls for the organization ot the Indian proltariat on the basis of the class struggle for the foundation of a Communist State, based during the transition period on the dictatorship of the proletariat. 

We call upon the workers of all countries especially Great Britain to help us to realize our programme. The proletarian struggle in India as well as in other dependencies of Great Britain should be considered as vital factors in the International Proletarian Movement. Self-determination for India merely encourages the idea of bourgeois nationalism. Denounce the masked imperialists who claim it and who disgrace your name (of British workers). The fact that India is ruled by the mightiest imperialism known to history makes any kind of revolutionary organization among the working class almost impossible. The first step towards the social revolution must be to create a situation favourable for organizing the masses for final struggle. Such a situation can be created only by the overthrow or at least the weakening of the foreign imperialism which maintains itself by military power. 

"Cease to fall victims to the imperialist cry that the masses of the East are backward races and must go through the hell fires of a capitalists exploitation from which you are struggling to escape”-“we appeal to you to recognize the Indian revolutionary movement as a vital part of the world proletarian struggle against capitalism. Help us to raise the banner of social revolution in India and to free ourselves from Capitalistic Imperialism that we may help you in final struggle for the realization of the universal Communist State."


\begin{flushright}
Sd/- Manabendra Nath Roy 
Abani Mukherji 
Santi Devi
\end{flushright}


This appeal, with its orthodox Leninism and its misreading of Indian politics woven into an incitement to rebellion, is reminiscent of a letter addressed by Lenin to the British Labour Party just before the Scarborough Conference. That letter turned the Conference against Bolshevism and all its works and led to a descisive repudiation of Third Intemaional. This 
appeal may well have a similar effect if it comes to notice in India. Still the writers’ belief in indegenous Bolshevism in India is interesting, if not insignificant. 

\textsc{Indian Revolutionaries Abroad:} Some scraps of information are available regarding a few well-known persons, which indicate how they are working together. Mohendra (sic) Nath Ray was received in Europe by Birendra Nath Das Gupta, who forwarded him to Germany on his way to Reval. Birendra Nath Ghosh, recently released from the Andamans and now in Calcutta, is corresponding with Das Gupta, but with what object it is not known. Das Gupta himself wishes to return to India, A. A. Mirza, so long identified with Islamic and Pro-Bolshevik agitation in England, has at last made his way to Rome. Italy has become a most important centre of revolutionary intrigue. Benoy Kumar Sarkar, an old associate of Lajpat Rai in America where he still is, has applied for a passport to France. Mrs. Naidu has been travelling in Europe to the great interest of the revolutionaries, of whom Das Gupta writing to a friend in Italy strongly advised him to get an invitation to Italy extended to her and to see himself. 
This same letter described the printing of propaganda in Italy and their distribution through Germany and America. 

Chattopadhyaya remains in Stockholm. He is reputed to receive Bolshevik money, though he is often short of funds and is thought to supplement his own earnings with the help of Swedish friends. He receives anti-British literature from America and republishes it in Sweden. He, too, has a plan for a communist revolt in India and is confident of its success. He hopes to send it to India by hand in September or October. 

These details have been given because it is believed that the Indian revolutionaries abroad are beginning to show a new activity. They have found new Allies and, it would appear, new plans. They are quite unpractical enough to build on the hope of a Communist revolution and they are just as ready as they ever were to be exploited by unscrupulous associates. 

When the last mail left England the Third International was sitting in conference in Moscow. 

The delegates of Asiatic countries, India, China and Korea etc. attended the preparatory Session of the Congress of the Third International and was warmly received. Royde (? Roy) 
who represented India declared that the flames of the social revolution were spreading and that Oriental people would soon follow the example of Russia. This Royde may be the ubiquitous N. N. Bhattacharjya. He was followed by other European speakers whose addresses were received with applause and are to be printed for propaganda purposes. Lenin is said to have announced (Figaro 22nd July) that Russia had no intention of pursuing the campaign against the West after Poland had been conquered, but that the world revolution would then extend itself to India where Irish Soldiers were distributing arms and munitions to the Hindus. 


\textsc{Communist Party of India:} Some time ago it was reported that certain individuals in Calcutta had subscribed to and were receiving the Workers Dreadnought from England. The names of these individuals were given and enquiry was made about them in Calcutta. One of them only, Muhammad Yusha Knan, has been found to be receiving the paper; it could not be ascertained whether others were receiving or not. Mohammad Yusha Khan is a member of a big farm in Calcutta dealing in salted hides, he is Wahabi and a cousin of Mohammad Akram Khan, Khilatat agitator and editor of the Mohammadi. Yusha Khan helped Akram Khan with money to start this paper and supports him generally in political matters. This paper describes itself as published by the C. P. (British section of the Third International) editor Sylvia Pankhurst. 

Miss Pankhurst of course receives money from the Soviet Government and attended the recent conference of the Third Internatioanl at Moscow. 

Another Bolshevik production has recently been found in India. It IS called Soviet Russia and is published by Maartene Bureau in New York. This particular copy was sent gratis to the editor of an Indian paper. The Soviet subsidised Daily Herald also appears to be received by every mail. \\


\textsc{Indian Revolutionaries Abroad}\\

\textsc{B. N. Dasgupta:} The most interesting news of him is that he presented a petition to the Secretary of State praying that the terms of Royal amnesty may be applied to him. He was,
he says, a most loyal subject until the war broke out when by his youthful eagerness for democratic political progress and his then love for Turkey he was induced to help his 
Majesty’s former enemies. He makes the usual promises to amend and devote his full time and energy to further the industrial and commercial development of His Majesty’s Indian Hmpire. 

This merely means that, as reported from another source, he is home sick and anxious to return to India. He is said to believe, probably rightly, that the development of Indian industries is a fundamental step towards revolution. There is certainly no evidence of the sincerity of his repentence in the record of his recent activities........... He has great faith m the Bolsheviks and says an agreement has been reached between them and the Indian revolutionaries. The main centre of work are, he says Moscow, Kabul and New York, San Fransisco and small centre in England Among the Indians in Moscow are Mukherjee, M. N. Roy and Halfsri (?) and Rash Behari Bose is according to him in Afghanistan along with Mahendra Pratap and Acharya. 

It is perfectly true that M. N. Roy (N. N. Bhattacharyya) is in Russia and that Mahendra Pratap and Acharya are in Afghanistan. But nothing has been heard previously of Bose going to Afghanistan, a fact which would most probably have come to notice had it occurred. About Mukherjee there are excellent grounds for believing him to have stayed in Germany to watch the work there. M. N. Roy won a considerable reputation for himself among the Indians in America by his communism in Mexico, and since he has arrived in Europe he has set himself to work on Bolshevik rather than on nationalist line. His presence must tend to eclipse the old 
Indian Committee to whom by his Communist Manifesto he has declared himself antagonistic. But all the Berlin Indians are said to be anxious to join the Bolsheviks. 

\textsc{Diwanchand Varma:} This man claims for himself a considerable past as a revolutionary and to have been one of the first Indian “terrorists.” 

He is now a convinced Bolshevik and apparently in touch with some of the leaders, but he is rather out of touch with the Indian movement. \\


\textsc{Indian Communists}\\

Reports about the following individuals have been received and are summarized below ; 

\textsc{Dalip Singh Gil}, described as the son of a peasant in Patiala State and brother of a dacoit who was hanged, arrived in Switzerland from America during the war. He was suspected by the German Government of being a British spy and was enticed into Germany and arrested. He remained in Jail, where he made acquintance of Liebknecht, until the Revolution. He was set free with Leibknecht and was supported by him and his party, through whom he came to know 
German and Russian Communists, Radek being one of his intimate friends. From them he conceived the idea of trying to introduce Communism into India and himself became a Communist. His efforts to secure the support of other communists were failed by his ignorance of Geman. he therefore  associated Mansur(Dr. Mansur) with himself and thus made his own progress easy.

Chattopadhyay is still in Stockholm and states that he too has hopes of obraining Bolshevik money, with which he intends to start a paper called the "Indian Communist" to be distributed free all over the world. He has seen Kamenoff, who gives him a sham contract for purchasing chemicals in order to blind the police. He corresponds with Germany and Russia through Bolshevik couriers, is anxious to get B. N. Dutta from Berlin to help him and accuses Har Dayal of having been bought by the British Government.......... His faith is entirely fixed on the Bolsheviks, who are said to be preparing for an Indian Revolution in March next year, and whose Bureau of Eastern Propaganda is working harder than ever....... Chatto also hopes to make Bolsheviks of all Indians and intends to start with Rabindranath Tagore, whom 
he expects in Stockholm in September and October and whose recent utterance have been such as to encourage Chattos’s hopes. 

Saklatvala has been in communication with Roy (N. N. Bhattacharji) whilst the latter was in Moscow through a delegate who attended the conference from Glasgow who has now returned. Roy wants Saklatvala to establish an Indian Communist group associated with the British Communist
Party. He states he has been seeking to influence in the direction of improving conditions of Indian workmen, and is in thorough agreement with Saklatvala in despising the Indian National Congress, which he regards as “an illegal assembly of a few aristocratic gentlemen” called together in order to dominate the mass of the people. 

Moscow Conference-Reliable information gives names of delegate who represented various sections of British India as: 
(1) Mahendra Pratap  
(2) Suhrawardy.  
(3) Martin.
(4) Roy.
(5) Mantu.
(6) Barkatullah. 
(7) Unknown. 

Ail these men are well-known ; Martin and Roy are two aliases of N. N. Bhattacharji. As far as their succeeding movments are concerned Suhrawardy is at present under examination at Constantinople where he went via Tiflis, which place he communicated with his family in Bengal asking for money and stating he was “quite well.” 

N. N. Bhattacharji is reported to be with Jamal Pasha’s mission to Afghanistan and there is some reason to believe that he may attempt to enter India. 

B. N. Das Gupta is going to Stockholm as soon as his brother arrives, but he expects to return in about a month. Rabidra Nath Tagore, Mrs. Sarojini Naidu and B.N. Dutta from Berlin are also off to Stockholm. There is to be a meeting of the members of the Executive Committee of the Indian National Society as soon as everyone is assembled. Dutt has sent a wire in Cipher to Das Gupta to proceed to Stockholm at once. 

\textsc{Indian Revolutionaries Abroad}

The Berlin Hindu Group : B. N. Dutt’s correspondence is still the main source of news of this dwindling body of the irreconcileables. He recently wrote that the “Traitors” had left Germany for London a few days previously; that some of them were approvers like Dr. C. Chakravarty and among them was one likely to keep his word and work furtively in India. Dutt remarked that he was delighted to have got rid of these useless persons and to be left with a clean sheet, though there were still some who would have to be removed. Now was the time, he said, to procure fresh blood from India to assist in the accomplishment of their heavy task. Accordingly he asked Das Gupta at least to induce Jatin Sette (?) and Fazlul Hak Hasrat Mohavi (an Aligarh graduate) to join him as soon as possible ; he added that he had addressed a similar request to Chattopadhyaya. 

It is believed that Hasrat Mohavi (or Mohani) is identical with the individual interned in India for complicity in the silk letter case ; in short the individual now so prominent in the Khilafat agitation. Regarding Jatin Sette (?) Das Gupta remarked in conversation that he was an extreme revolutionary who had been interned but was now free. He is an M. A. of Calcutta whose real name may be Jatindranath Sen or Seth. 

\textsc{Soviet Designs on India:} That the Bolshevik Government is thoroughly earnest in its hope to provoke revolution in India, as the best means of wrecking the British empire, as I think, been so proved as to leave not the slightest doubt in anybody who is open to conviction. Bolshevik speakers and writers have openly proclaimed their intentions and spread the announcements over the world. From every direction have come secret reports of plans and intrigues undertaken to give effect to these designs. Every revolutionary party or society seems secure in its hope of financial and other assistance from Lenin and his friends. The distinction, therefore, which is made for clearness in this report between revolutionary bodies and Bolshevik agenices is a false distinction, because now-a-days every revolutionary 
organisation whatever its origin seeks alliance with Bolshevism. 

The important question then is by what methods the Bolsheviks can hope to execute their plans in India. They can rely either on an invasion from Central Asia of forces raised by themselves, or on indigenous agencies in India, or on a combination of the two. Indigenous agencies are certainly hard at work to promote disaffection against government. Their methods are certainly skilful and as such are likely to rot the core of Government’s strength by disaffecting its servants, military and civil, and by destroying the influence of the more conservative elements of Indian Society through the promotion of a government of dictatorship of the proletariat. That their methods are disguised as Khilafat agitation or election campaign need not affect their result. As regards the likelihood of invasion this seems more remote. 

\textsc{Indian Bolsheviks:} A report from Geneva of the 18th February declares that Bravin, the Bolshevik emissary has made his way into India with three Indian assistants and that he is working round about Peshawar where a secret conference was to be held in February. This Conference was to have been of the greatest importance as it was to have united the islamic and non-islamic parties for the war against England; and one Nafis was anxious to attend at 
all costs. 

Enquiry is made about Nafis who is said to be a native 
of Trans-Caspia, who was in Calcutta in 1902-05 and visited Persia, Moscow, Switzerland and Berlin where he was associated with Chempakaraman Pillai. He may possibly be identical with the notorious Abdul Hafiz of the Zurich Bomb case. But the report, so far as the object of the Peshawar Conference is concerned, is given with the greatest reserve. 

Another report states that there are now 150 Indians in Moscow and Petrograd who are undergoing instructions in the art of propaganda. When qualified in these school Indians 
return to their native country. A German named Preetz or Praetz, long engaged in India as merchant or manufacturer in the textile trade and now in Berlin is declared to have stated that he had received from Lenin the enormous sum of 50 million United States gold dollars and I Milliard of Czarist paper roubles for the purpose of propagating the Bolshevik gospel in India. \\

\textsc{Imported Bolshevism in India}

The letters printed below have a peculiar interest as to whether they are explained as emanating from real Bolshevik emissaries or from Indians aiming Bolshevism. There is no foundation in fact for the widely .spread rumour that Bravin has succeeded in entering India with two of his assistants. In fact he was superseded in Afghanistan by Suritz and is believed now to be in Caucasia. But this name may be a cloak for the emissaries who actually are in India. 

\textsc{Mahendra Nath Roy:} This Indian revolutionary escaped arrest in the United States by fleeing to Mexico with his American wife. There he continued the production of pamphlets and literature attacking the British Indian Government. On one occasion, as reported at the time, he offered the fruits of his labours to the German embassy for any purpose for which they could be employed. It was there too that he was 
converted to the Communist Creed and associated himself with Lynn A. E. Gale and other Bolsheviks and eventually became the leader of the Mexican Communists. But for a brief appearance as a labour agitator at Tampico his Bolshevism found only a literary expression, so far as is at present known. It is now reported that he left Mexico on January 15th last and that he is believed to be on his way to Russia via Spain. 
Since his departure ‘EL Communists ’ the organ of the Communist Party has not appeared and it is thought that lack of funds and lack of a suitable person to take direction of it will prevent its reappearance in future. 


\textsc{"Lenin The Strategist”} — “Lenin has very good reason for the Indian, Egyptian, Persian and other Nationalist intrigues which he is promoting against Great Britain. He regards it as impossible to exercise and direct influence on the English workmen which lead them along the paths of communism. Consciously or unconsciously, the English working of which Centre of an Empire, the prosperity of which depends on its colonies. He is thus too well off, and too deeply imbued with the idea of property and self-interest to be influenced by communist propaganda. The utmost of which he is capable is a progressive series of bargains with Capitalism and by that route communism will never be reached But, if England were deprived of her colonies, then her industrial condition would be no better than that of the countries of the European mainland and her exchange would fall as there has done. The English workmen would then cease to be prosperous or contended, and England could be made as ripe for communism as France or Italy. Therefore, in so far as England is con- cerned, Lenin is devoting himself ardently to the destruction of her Empire and the liberation of her colonies." \\

\textsc{Bolshehvik Propaganda in India }\\

\textsc{Bepin Chandra Pal}, who had been on tour in East Bengal, and Sylhet, along with Srish Chatterjee, pleader of well-known revolutionary tendencies, had returned to Calcutta. Detailed report of his speeches during his recent tour show that they were of a more than usually objectionable nature. At Sylhet on 23rd September he delivered a speech obviouly intended to excite the people of that district most of whom belong to the Baisnab sect. Universal brotherhood and self-reliance, he pointed out are the keynotes of the lives of both the Bolsheviks and the Baisnab, the only difference being in respect of violence to which the Bolsheviks are accustomed. Just as the Baisnab goes to Sri Brindaban, so the Bolsheviks, are also coming to India. 

\textsc{Rash Behari Bose:} A Report was received sometime ago that Bose was probably in Afghanistan in touch with the Bolseheviks. This has to some extent been corroborated by a confessing revolutionary in Bengal who reports that another absconder and associate of Rash Behari named Amarendra Chatterjee is in touch with the Bolshevik, through Rash Behari, while a different Calcutta Police source reports that Amarendra has recently been in Afghanistan. 

A report has been received that Khalil Makdour, a member of the Egyptian party in Geneva, has been asked to join a Bolsheviks Party which left Berlin in March to stir up trouble on the Indian frontier. 

It is noticeable that a good many rumours of imaginary Bolshevik successes are current in Northern India. These chiefly concern the relation between Bolshevism and Afghanistan, the intentions to the Amir to outwardly profess friendship and to suddenly descend on India at a favourable opportunity, and the spread of unrest among the frontier tribes etc. The return from Afghanistan of large numbers of ignorant Muhajirin is sufficient explanation in itself of the source from which these rumours originate, and there is no reason, on present evidence, to suppose that they are the work of Bolshevik agents who have penetrated India. 

In one of the recent numbers of the India News Service issued by the Friends of Freedom for India an account is given of the part played by Roy (N. N. Bhattacharya) at the Moscow Conference. According to this he showed himsef "plus royaliste que le roi" in opposing Lenin, who wished to support existing Nationalist agitation in India as a means of overthrowing the present administration preparatory to the establishment of Bolshevism. Roy held that agitation in India was confined to the middle classes, and that the purity of Bolshevik ideals should not be sullied by any cooperation with the “bourgeois”. He ultimately allowed himself to be convinced by Lenin-the whole affair was probably a move to gain notoriety. 

\textsc{The Berlin Group:} It is reported from Berlin that Achariya who is now in Moscow, has written to Chattaopadhyaya in Stockholm informing him that the Russians are now concentrating their energies on rendering assistance to the Pan-Islamic Movement, as such, and as outside other political movement. This attitude, Achariya points out, must be strongly protested against. 

Upon receipt of this letter Chatto decided to go at once in person to Moscow, it being felt that should this line be taken up and persisted in by Russia, it would be highly detrimental 
to the interest of Indian independence. Further it appears that Dutta has already sent an ultimatum on this subject to Lenin by the hand of a lady who has recently gone to Moscow, named Clara Szetky (jic). Das Gupta went to Berlin and from there to Stockholm, in order to join Chatto. Dutta was also to go, but had not, at the date of the report, obtained a passport. The three of them intend holding a conference in Stockholm and Chotto will then proceed to Moscow. 

It is understood that should the negotiations with the Russian Government turn out unsati.sfactorily, a violent anti-Revolutionary Propaganda will be started by the leaders of the Indian Revolutionary movement in Europe. At the moment they are at a loss to know how to act. Das Gupta (who is travelling under the name of Haider) will return from Stockholm to Berlin, in about a fortnight’s time. Dr. Ghose, his wife and nephew have met Chatto but it is not definitely known where the meeting took place. 

Chatto’ s intention of visiting Moscow has been confirmed from another source. It appears that he has received Bolshevik funds through Hellberg who is a prominent member of the Central Bolshevik Committee of Stockholm, and that he intends to accompany Litvinoff on his journey to Russia via Reval. 

At the Baku Conference in September Enver Pasha proclaimed his agreement with the views of the Third International in the name of Algeria, Tunis, Tripoli, Egypt, Arabia and India. 

Roy, representative of India, was apparently responsible for the statement that there were over 37 million landless peasants in India and that the entire land was in the possession of some six or seven hundred princely families. He regretted however to admit that the national India movement was being carried on by the middle classes. It is difficult to believe that even Roy would make the preposterous statement that all the lands of India in the hands of some six or seven hundred princely families, but his regret that the national India movement is being carried on by the middle classes is entirely in the strain of a letter he wrote from Reval at the end of May to a friend in America. In the course of that letter Roy said, 
“If the Nationalists leaders don’t see our point of view we are determined to part with them and even fight them if necessary, and it is inevitable that we must fight the Nationalists either now or later. Since we are convinced that the establishment of Nationalists Government would not emancipate the masses.” 

Roy is out for notoriety, and means to impress the Bolsheviks with his importance. It is difficult to estimate what influence he carries, probably it is not very great. From the 
latest report it appears that he has decided to remain in Tashkent for a few months and has abandoned his intention of proceeding to Kabul. 

\textsc{Sneevliets,} who was recently reported to be en route for the Far East, where he was to carry on Bolshevik propaganda, has suddenly returned to Holland. It is strongly suspected 
that his change of plans was due to direct orders from Moscow, in connection with Rabindra Nath Tagore’s visit to Holland. The Soviet Government sometime back invited Tagore and 
Sir Jagadish Bose to a congress to be held in Moscow to discuss Orientalism and Internationalism, and Sneevliet’s mission was apparently to prepare Tagore’s mind for the proposal which would be made to him at Moscow. The Communists in Holland watched Tagore very closely during his stay, and as a result an adverse report concerning him is said to have been sent to Moscow, as Tagore did not associate with communists, neither were his lectures appreciated by them. 

\textsc{Propaganda in India;} Many references have of late been made to Bolshevik plans for flooding India with agents and literature, and that such is their desire no doubt. But there is little evidence in India to show that these plans have ever been carried into execution. 

It is possible that men have entered the country who have been supplied with money from Russian sources, on the understanding that they would carry on Bolshevik propaganda; but once in India their connection with Bolshevism, has gone no further than taking Soviet money. Probably most of these socalled agents had no intention of carrying out their contracts, they desired to return to India and had no objection to return with money obtained at the cost of promises which they knew it would be impossible to enforce. 

So far as indirect methods are concerned, such as subsiding existing agitation, it is not easy to appreciate the situation. The labour unrest in large industrial centres is an obvious instance where Bolshevik influence might be suspected. Of the prominent labour leaders, Lajpat Rai has Bolshevisk leanings, Chaman Lai is in close touch with English Communists through Saklatvala in London. It is therefore not difficult to show a certain connection with Russian ideas, but up to the present no proof has been obtained of any Russian money behind the labour agitation. The rise in prices and economic causes generally are sufficient in themselves to explain the present epidemic of strikes. 

%letter here%

\textsc{Berlin to Switzerland:} Recent reports confirm the information given in previous weekly Report that the Berlin Committee are in communication with Ghadr Party in San Fransisco and the Friends of Freedom for India in New York. 

All communications between the Berlin Committee and the outside would appear to pass through the hands of Das Gupta in Zurich where he is known as M. A. Haider. This man has been recently described as the most active and dangerous of the Indian conspirators and has recently replaced Prabhakar as the leader of these men. From his retreat at Zurich he is in touch with many phases of the great anti-British conspiracy. 

A very optimistic letter reached him from B. N. Dutt in Berlin at the beginning of November. The writer asserted that the fight for India was about to begin and that Afghanistan would resume hostilities in six month’s time. He alluded in cryptic terms to a most fortunate event which has just occurred and which made him feel sure that the days of British domination in India were numbered. He pleaded the attention of British spies as an excuse of not saying more about it, but promised full detail when he met Das Gupta at the Socialist Conference in Switzerland in January. He added that owing to the great responsibility of his work he had summoned Chattopadyaya from Stockholm and was anxiouly awaiting his arrival. In a letter Dutt gave Das Gupta news received 'at last and after great difficulty' from Indian Committee in 
Kabul. He described their activities in Afghanistan and their efforts to utilise Kashmir as a secret jumping-off place for work in India. He wrote hopefully of the progress made, but excused himself from giving details. (It is noteworthy that Har Dayal sometimes ago urged the desireability of making use of Kashmir, particularly for importing arm into India , and said he had a friend in Srinagar ready to help. In a third letter Dutt informed Das Gupta that Chattopadhayayahas sentTarak Nath Das 5,000 Kronen through his attorney Gilbert E Roe (Roe is defending various Indians in deportatic and other cases and was recently elected a president of Friend of Freedom for India). 

Das Gupta has also received letters from Tarak Nath Das and Sailendra Nath Ghosh in America both appealing for funds from the Berlin Committee; Das on the ground that the work, he is doing benefits the Germans as well as Indians, and Ghosh on the ground that his work must be carried out on a far larger scale. Ghosh also said that he had received a very important message from India together with a letter from a certain Satu which is to be delivered personally either to B. N. Dutt or Das Gupta. 

\textsc{Activity in Central Asia:} Fifty-four schools have been opened at Tashkent, mainly for propaganda purposes where oriental languages are taught and some Indians are engaged as instructors. Propaganda literature is also being prepared. As agents become proficient they are to be sent to India, China and all other countries having a Muhammadan population. Those for India will enter mostly by Afghanistan under Afghan auspices or by sea under various disguises. 

The decision to concentrate all efforts on India was recently re-affirmed by the Tashkent Soviet, because it is hoped to decide there the destruction of the British empire and the future of the world proletariat. It is understood that centres where propaganda will be partly prepared are to be opened in India.\\ 


\textsc{Reciprocal Movements between Russia and India}\\

The following was issued by wireless on December 12th: 
“Russian papers report that an Indian Bolshevik Commission is said to have arrived at Samara in order to enter into relations with Soviet Russia”. 
The \textit{Svenska Dagbladet} of Helsingfors gave the following details about the same time; 

“Indian Bolshevik Commission is actually in Samara. The chiefs have declared that 300 millions of Indians are awaiting a favourable occasion for rejecting the British slavery and that they want to join with Russia”. 

From Sweden too comes the report that about 100 agitators have been sent to India from the schools in Moscow. I have received detailed reports about 3 such agitators have been sent to India from schools in Moscow. 

Two are Finns who were expected at the beginning of January to leave Stockholm for India as propagandists. Their names are given as Issenivs and Karl Harrin; but they were expected to travel with forged or stolen passports as Harry Bennet American and Thomas Grieg, British. Issenius is believed to be identical with Allan Usenius an extremely is known for certain of Karl Harrin, but he is possibly Hurmev Aara, a Finnish Bolshevik of some importance in Stockholm.

The third man is Dr. Max Fischer. It may be mentioned here though he is not known as a Bolshevik, but he is employed by the German Ministry of Foreign Affairs. He was reported in December to be on his way to Tricste for India for anti-British propaganda. He possesses 5 passports, three in his own name as commercial agent, consular agent, and tourist and two under an assumed name. This man is perhaps identical with a man of the same name who was working with the Chinese revolutionaries in Shanghai in 1916. 

\textsc{A Bengali Bolshevik:} The Intelligence Branch, C.I.D., Bengal has received the result of enquiries made with regard to Shaheed Suhrawardy, who was one of the first Indians to throw in his lot with the Soviet Government in Moscow. This man was a well-known revolutionist. His father Zahid Suhrawardy, a judge of the Calcutta Small Causes Court, has not heard of his son for a long time and is unwilling to talk about him. He believes him to be in touch with the Bolsheviks. An officer who knows the family recently ascertained from one of their relations that one of the members have gone to Russia via Afghanistan and been detained there. On inquiry whether any information could be obtained in Calcutta about Suhrawardy it was said that a Russian named Ivan- how could give information. This Russian was said to be a great Arabic and Persian scholar who had come to India to prosecute his studies and had been on friendly terms with Suhrawardy on account of the latter’s knowledge of Arabic. No Russian named Ivanhow is known in Calcutta, but inquiries are made to trace him.\\

\textsc{Indian Revolutionaries Abroad}\\ 

\textsc{The Berlin Indian Committee:} Information has come from Berlin regarding many Indian renegades in Europe. This shows that the community is considerably exercised in mind regarding the ultimate fate of its members, and that considerable friction exists between the Hindu and Muhammadan members. Details are given to illustrate the case with which individual renegades, even those who have not worked whole-heartedly with Germany, can obtain grants of money 
from the German Foreign Office. The active organisations of the Indians at the moment are: 

1. The Orient Institute. 
2. The Indische Gesellschaft 
3. The Hindustan Sabha 

It is not very clear to what end these different organisation are directed. The first seems to be working to maintain the pro-German and anti-British feelings of the various orientals in the lately belligerent countries. The second of which B. N. Dutt is the head is reported to be directly under the German Foreign Office, and all the Hindus in Germany belong to it. The third is the Indo-German Bolshevik Society founded about 6 months ago by Dr. Mansur, Verma and Dalip Singh with the object of spreading Bolshevism in India. It is reported to be most flourishing. According to Dalip Singh it has members in Sweden, Russia, Austria, Egypt, Turkey and America and is much helped by the local communist party in Berlin. \\

\textsc{Press and Platform Bolshevism:} The attitude of the Press towards Bolshevism is still following the lines indicated in a recent weekly Report. The opinion to be formed from a study of articles and speeches on the subject is that the extremist politician is becoming more and more inclined to dabble in extreme socialism. Whatever may be the individual view of socialism there can be little doubt that its doctrines combined with existing conditions in India form a dangerously inflammable mixture. 
 
The chief exponents of Bolshevism in the press at the moment are the Hindu of Hyderabad (Sindh) and the small group of extremist journalists at Cawnpore who are connected with Pratap and the Prabhu. This group has been noticed in recent issues of this report. Now the Maryada has been inoculated with the virus. This paper belongs to Madan Mohan Malaviya whose nephew Krishna Kanto Malaviya is the editor. Its circulation is about 1 ,600 and it is printed at 
the same press as the Abhyudaya which has a circulation of 3,500. The United Provinces CID which reports these facts, has drawn attention to these articles in the Maryada for February. Of these two are written by Rama Shankar Avasthi, Assistant editor of the Pratap. 

The first article asserts that people now realise that no amount of villification or denunciation of Bolshevism can check its onward progress. No one can be sure that it will keep with the boundaries of Russia. It has propaganda in most European countries and in America and is a great menace to capitalism and imperialism. Lenin and Trotsky are true patriots; they have crushed their enemies, improved the economic conditions of Russia and are carrying on the
internal administration smoothly. Mr. Llyod George is in favour of concluding Peace with them. They have had to go through a very difficult ordeal but have come out successful. 

The Hindu of Hyderabad (Sindh) published on January 30th an article on the “Bolshevik danger to India” from which the following is taken : 

“In our opinion whether the Bolshevik attack or preach their propaganda or not, if the grievances of the public against the Government continue and the quarrels between labourers and the wealthy go on, men themselves under similar provocations as have the Russians will be affected with Bolshevism; and subsequently, if similar bloodshed and disturbances occur, it will not be surprising. If this danger is not attended with risk today, it will be to-morrow. Therefore, it appears to be our duty to oppose this danger and for this we should prepare now. We should improve the condition of our backward brethern, and having shown our sympathy to them, we should strengthen the nation. Otherwise if, like the rich people of Russia or England, we oppress the backward classes, their sorrows and grievances will re-act on this 
nation.” 

On February 2nd Jethmal Parsaram, a notorious Sindhi agitator, lectured on socialism to an audience of 300. The speaker’s argument was difficult to follow, coloured as it was with facts distorted to suit his argument. He concludes by saying;

“We should get more rights of Home Rule. Officers, you are only a few. It is our country and you should give us the reins. India is not yours, seventy five per cent of Indians are poor, and if you have their reins they will starve. These poor men you should care for. When they get the votes they will trouble you very much. What you call Bolshevism is really hunger.” 

In reporting the lecture the Bombay Special Branch remarked that the lecture was significant for two reasons: “it shows how the extremists are deliberately fostering industrial discontent; and secondly how they or at any rate a section of them are prepared to welcome Bolshevism for the furtherance of their own ends. The reporting sub-inspector noted that the whole trend of the lecture was to dispose the audience favourably towards Bolshevism.” 

It is perhaps worth noting here the Gale’s Magazine of Revolutionary Communism has been advertised for sale and actually obtained in both Karachi and Bombay. The magazine is published in Mexico by a disreputable individual named Lynn E. Gale who fled from the United States during the war to avoid the draft. His magazine ii openly Bolshevik and advocates “New Thot”. Gale himself is an associate of Narendra Bhattacharya alias C. Martin in Mexico.\\

\textsc{Specimens of Pro-Bolshevism and other speeches in India }\\

Speaking at a Khilafat meeting at Lahore on February 8th last, Gobardhan Das, ex-convict is reported to have praised the Bolsheviks and said that the rich had no right to live so 
comfortably when the poor were in trouble. He described Bolshevik principles as quite natural and praiseworthy and advocated them as worth following. He wished, he said, to see Bolshevism preached and acted upon in India. 

\textsc{Indian Revolutionaries Abroad:} Reliable information has been received to the effect that Mahendra Pratap, Abdur Rabb and Acharyya are in Kabul at the beginning of January last. 
They had previously spent two months in Moscow, and had passed through Turkestan or their way to Afghanistan. In January too, Barkatullah was in Moscow, but was shortly to have gone to Turkestan. Chattopadhyay in Stockholm was asked to communicate with him through the Moscow Foreign Office. The last named was also instructed that it was desirable that he, Hardayal and other Indians in Europe should get in touch with representatives of the Russian Republic in different places. Relations were also to be established between Indian communities in all parts of the world and the Russian Government. Men of integrity and principles were to be sent to Russia for propaganda work. 

From Christiana it is reported that the European Indian Committee is beginning an intensive propaganda in order to undermine the reputation of the British Govt. The general ignorance of Norwegians regarding India and their sentimental character are believed to favour the revolutionists. Otherwise Chattopadhyay is reported to be following a law-abiding life, and his sister Mrs. Sarojini Naidu is said to be lecturing on Indian subjects without reference to politics. \\

\textsc{The Soviet’s Interest in India}\\

The Gazette de Lausanne of February 12th has contained a remarkable article by one Sergy Persky entitled “Lenin et les Indes Britanniques.’’ This has been a stock subject with the French Press for sometime past, but most of the articles have been merely copied from English papers. That in the Gazette was of different calibre. The writer described the disillusionment which has overtaken the American politicians who in 1918 denied that Bolshevism has any interest for them; and the disillusionment which awaits the British premier if he imagines he can confide in Lenin’s promises or trust him to abandon Bolshevik propaganda when the blockade is raised. While Litvinoff exchanged sweet words at Stockholm, Moscow worked hard to Bolshevise Afghanistan and the British colonies and awaited the moment for effective work in England... But it is India specially that they (Bolsheviks) attack.’’ 

In September 1918 the Council of Workmen and, Soldiers at Moscow received five Hindus “Messengers of Indian People”, really creatures of Lenin, who picturesquely described the sufferings of their compatriots and the oppression of the English. “All our hopes”, they concluded, “are based on you, our brothers”. “Come and deliver us and we shall bless you.” 

The writer then described two copies of a curious book which he had received from Moscow several months before. One copy was in Russian, the other in Hindi, and it was called: 

“India for the Indians, Blue Book; collection of secret documents. Edition of the Commissariat of Foreign Affairs, Moscow, 1st edition.” 

On the front page in large characters was printed: 
\begin{center}
\textsc{India for the Indians.} \\
\textsc{Down with the imperialists.} \\
\textsc{Long live the international.}\\
\end{center}

The volume purports to be a collection of consular reports and letters from India received during the Tsarist regime and taken from the Archives of the Imperial Foreign Office. 

Describing the periodical failure of crops, famines etc. the preface declares that it is the worst of the errors to attribute these entirely to natural causes. The only rational remedy is a complete change of the agrarian laws and the formation of a grain reserve. But England will not allow India to reserve the grain which she requires for herself, since she lives by the exploitation of her colonies. English policy towards India, both economic and administrative, is despotic in a degree equal to that of the old Tsarist regime. Neither the divine will nor the Indian workmen — so hard working and so well endowed for work, is to blame for the famines; the guilty one is the Englishman, egotistical and ferocious, who for more than a century has sucked the blood of his unfortunate victim. "This abominable policy of England" is illustrated by descriptions of the army, the police and the system of taxation.

England this will be a terrible blow. England without India is of no account : for this reason she has always refused to lighten her yoke. It is fair to say that England only entered the World War for the sake of India and the routes to India. Seeing danger from Germany and Austria she did not hesitate to throw one-half of Europe upon the other and finally to drag the whole civilised world into the bloody conflict. 

The importance of India to England is thus enormous; and the freedom of India is thus of vital moment, and every possible means of affecting it must be employed. An Indian revolution would cause a world-wide shock, and without an independent India there can be no general peace. We must therefore not only acclaim an Indian revolution, but with every means at our disposal we must work for it directly or indirectly. Let our Indian friends take this expression of our sentiments as a formal engagement to help them. In the not distant future we shall have the joy of seeing our two revolutionary roads meet and join, not only on the ground of national enfranchisement, but also on the yet more burning soil of the struggle of the classes and of the reconstruction of a new social edifice and order. \\


\textsc{Lenin and Bengal}\\

A report has been received which states that Lenin intends to form in Bengal an organisation based on the old Bande Mataram movement which is still vivid in the recollection of the natives. It also states that Lenin is the prime mover in the fabrication of paper money.\\


\textbf{1921}\\


\textsc{Indian Revolutionaries and the Bolsheviks}


A large amount of space has been given in recent weekly reports to account of the revival of revolutionary activity amongst Indians in Europe which has resulted under Chattopadhyay's leadership. A report' has been since received from a source entirely independent from that on which previous accounts have been based. It is interesting both as largely corroborating previous information and as giving certain new facts. There are naturally discrepancies, but the main outlines of the story are in agreement with what is already known. Chatto, it is stated was in Moscow, towards the end of 1920 , and whilst there succeeded in obtaining a promise of assistance from the Soviet Government. One of 
the conditions imposed, however, was that Chatto should show proof that he actually represented the Indian Revolutionary and Communist Parties, and the proof asked for was a “Mandate” signed by the leading revolutionaries and Communists. Chatto was not able to produce any such mandate, but is now engaged in drawing all the well know revolutionaries into his net so that he may satisfy the Soviet Government and obtain their assistance as soon as possible. This strengthens the surmise made in a previous Weekly Report that Chatto's reorganisation scheme has been designed chiefly to attract Russian financial assistance. 

Of the fact which are new, the following are the most interesting. Chatto whilst in Berlin met the leaders of the Egyptian, Persian, and Turkish Committees, as well as several Germans and Americans, and on all sides received promises of assistance and cooperation. He has been reproached for going too slow, and for holding meetings which result in nothing but talk, but he pointed out in reply that they failed 
badly once before, even with the greatest power at their back and their failure was to be ascribed to the fact that they did their work without due care and consideration. 

Chatto has opened the old bureau of the Berlin/Indian Committee and has appointed Heramba Lai Gupta as Sec- 
retary for the time being. B. N. Dutt has been made General Secretary for Europe of the Indian Committee. The objects 
of the Indian Committee agree with the information already received, but a new suggestion is the smuggling of Communist propaganda literature into India by means of (Indian ?) sailors who visit various Indian ports. The Indian Committee do not consider it safe to send “trained” Indians back to India, and it is proposed to utilise the services of European Socialists and Communists for propaganda work in India. 

With regard to Communication with India, the Committee have come to the conclusion that the only way this can be done are: \\
(1) Either through English Socialists, or \\
(2) Indian students in London. 

In the case of the latter, only those are to be employed who are entirely above suspicion. 

Interest now centres on Chatto’s visit to Moscow, and if he can succeed in persuading the Soviet Government to give 
him satisfactory financial backing we may see interesting developments. \\

\textsc{Chattohadhyay's Group}\\

Further progress has been made in Chatto’s scheme as far as commercial side is concerned. His proposal for propaganda and political activities, will probably be left in abeyance until it is known what assistance Soviet Government will give. 

B. N. Das Gupta has decided to leave Switzerland and to establish the H. Q. of the Indo-European Trading Company 
in Germany where it is thought that most of the work will be conducted. As a further step he has secured accommodation in Berlin and Leipzig and has left Kasim to manage the branch in Zurich. 

Chatto and Heramba Lai Gupta is passing under the false name of Lopez and Chatto is an absconder in a Swiss political case. The share-holders of the concern will be shown as : \\
B. N. Das Gupta \\
S. N. Das Gupta \\
S. K. Ray, \\
Abdul Wahid, and \\
Ismail Kamil, who is described as a Barrister-at-law and a member of the Legislative Council of the Government of Ceylon. 

With regard to the recruitment of Indians for training in Europe it is proposed to send S. N. Das Gupta and S. K. Ray back to India to search for suitable men. The Soviet Government are reported to be prepared to pay all the expenses incurred in bringing Indians to Europe for this 
work. Some of them will to direct to Russia for training in Press Propaganda and others will be sent to the various branches of the Indo-European Trading Company for industrial training. 

Chatto’s Group is much exercised as to how he can get the mandate required by the Soviet Government. He had been considering the possibility of utilising his sister (Mrs. Sarojini Naidu) for the purpose of approaching revolutioanary leaders of India, but was not certain if she would be given permission to return (Note — she has already salied for India). 

\textsc{Baku Conference:} Details from a reliable source has been received concerning the Indian Delegates who attended the 
Baku Conference held last September. These are said to have been seven in number, all residents of Peshwar, and with the 
exception of one Abdul Kadir were by profession petty traders. Abdul Kadir who acted as leader of the delegation, 
is described as the only man among them who apparently had any education. The party as a whole did not take a very 
active part in the actual Conference though it is stated that they were well received and much appreciated. 

Four other Indians are mentioned who attended the Conference, but not direct from India. These are Roy, Mukherji, 
Maqbal Hussain and Misri Khan. The part played by Roy at Baku, and his subsequent activity at Tashkent, are well- known. It appears he is a personal friend of Lenin — who places the greatest confidence in him. His object is not only to win “freedom” for India but also to revolutionise it into adopting Bolshevism. In a conversation with Quelch, the delegate to Baku from England, Roy gave him to understand 
that there were hopes of Communism being established at least in Bengal, if nowhere else. He based this statement on the ground that Bengal was the brain of India, and religious differences which work all the parts of India are less apparent in Bengal than elsewhere. 

Mukherji is stated to be working under Roy’s supervision in Tashkent in connection with the preparation of propaganda for India. He is described as an energetic worker well-trusted by the Bolsheviks, and he is probably identical with Abani Mukherji. 

As regards the methods to be adopted for propaganda in India it seems that a great point is being made of training Russian Muhammadans with fair complexion which resemble Europeans as far as possible. It is hoped to introduce these men into India where their European appearance would have great effect with the ordinary villager. \\

\textsc{Proposed Transfer of Propaganda Headquarters}\\

Proposals are on foot for the transfer of the Indian revolutionaries in Bolshevik hands from Tashkent to Kabul. Roy is believed to be behind this proposed transfer, which finds favour with the Bolsheviks, as they require a base nearer India. Roy, Abdur Rab, and Acharyya were reported to have left Tashkent about January 20th for Moscow to attend a Conference of Indians (Note — Possibly a conference to discuss Chatto’s proposals). It was considered uncertain whether Roy would himself return to Tashkent, his presence among the propagandists not conducive to peaceful and united effort. He is not in good odour with the Indians, who are reported to be disaffected towards their present employers. Friction has also occurred between Roy and Suritz, the Bolshevik representative at Kabul. A proposal has been made that Roy should be put in-charge of the advanced centre of propaganda at Kabul, and owing to Suritz’s unwillingness to work with or under Roy, it is believed that Suritz may shortly be replaced by Legrand, the head Bolshevik representative in America. 

\textsc{Chatto’s Group:} It is not yet clear how far negotiations for financial assistance from Soviet Government has progressed. According to Das Gupta a definite sum has been mentioned, £50,000. This amount is however conditional on Chatto being able to satisfy Lenin, both as to his position as leader of the Indian revolutionaries and also as regards the feasibility of his schemes. Chatto has not yet been able to obtain the mandate required as a preliminary step. The mandate is to be signed by well-known persons such as Gandhi and other leaders in India and Chatto though quite able to obtain signatures of Indian revolutionaries in Europe has not yet hit upon a scheme for approaching Indian leaders in India. The position is further complicated by the attitude taken up by Roy in Russia. Roy is also claiming to be the leader of the Indian Revolutionary party. He has considerable influence with Lenin and has done his best to discredit Chatto. The latter on his part has, during recent months, attempted to expose Roy to Lenin but apparently with little success. The matter would be simplified if Roy and Chatto would agree to work together but as far as Chatto is concerned he is not prepared to do this for the following reasons: 
\indent(1) Roy is not considered sufficiently clever or stable. \\
\indent(2) He is believed to have embezzled party funds. \\
[Note:— This last accusation was brought against him by the Indians in America with whom he worked before his flight to Mexico.] 

It is probable that Chatto will utilise the visit of his sister, Mrs. Sarojini Naidu, to India, to approach Gandhi and other extremist leaders on his behalf; until the result of her efforts is known matters will probably remain at a stand still. 

\textsc{Indian Activity in Europe}\\

\textsc{Chatto’ s Group:} The question of obtaining financial assistance from the Soviet is still unsettled and remains the chief anxiety of Chatto and his followers. At the date of latest information (15th March) almost all the Indian revolutionary leaders were in Berlin, most probably in connection with the final settlement of the matter. At that time it was considered quite certain that the Soviets were prepared to contribute cash under certain safeguards, and Das Gupta had received a promise to that effect conveyed through the Soviet representative in Berlin. It remains to be seen what effect the signing of the Trade Agreement between Great Britain and Russia, with its stipulation against anti-British propaganda will have on these plans. 

Chatto continues to direct his attention towards linking up Indian revolutionary movements in various parts of the world in addition of Agnes Smedley of the Friends of Freedom for India, who as previously noted, has already arrived in Europe and is working as his Secretary, a proposal has been put forward that S. N. Kar should also be sent from America. If this proposal is carried out Kar will replace B. 
N. Dutt as head of the local Indian Committee in Berlin.

The American link has further strengthened by the arrival in Berlin of an American journalist named Lockmann. Lockmann was during the war, a financial intermediary between the organization in America and the German Embassy and has always been in close touch with the Indian movement.
He is a personal friend of Agnes Smedley and is now being used as a Bolshevist progaandist and it is reported that he will shortly be sent to London with messages from Chatto to Indians in London. 

With regard to Chatto’s suggestion that openly revolutionary branches of his organisation should be established in the important European Capitals, an Indian Deputation recently approached the German Government on the subject and were given to understand that the title selected “The Indian Revolutionary Society” was an objectionable one and might give rise to trouble with England. The Government proposed that the Society should camouflage itself under the title of the “Indian News Service and Information Bureau, Limited”, it being understood that so far from interfering with any revolutionary activities the German Government would render the Bureau its moral and material support. The inclusion of Agnes Smedley in this Bureau is under consideration. 

Friction between M. N. Roy in Russia and Chatto still continues. It appears that Roy has induced the Soviet to despatch 40,000 dollars to the San Fransisco Ghadr Party. 

Chatto heard of this from the Soviet Representative in Berlin and tried to stop the funds being despatched but was too late. He was particularly anxious to suspend payment until all the European groups were united, under his own control. A further report regarding Roy states that he has lost influence through an anti-Amirist speech he has recently made and it is reported that at Chatto’ s instance Roy was summoned and reprimanded. 

Chatto’s schemes for uniting all Indian revolutionaries in Europe under his own control have made further progress by the enlistment of the notorious Dr. Hafiz, who has agreed to become a member of the Central Executive Committee. It has further been settled that Hafiz should go to Afghanistan and open amunition factory that at the expense of the Committee, with funds which (it is anticipated) will be received from 
the Soviet. Hafiz is an expert chemist and is at present employed in Austria in amunitions factory. 

Dai.ip StNGH Gii.l : In Weekly Report of the 2nd of May, paragraph 5, it was noted that Dalip Singh Gill has been imprisoned by the Soviet Government in Moscow as a spy, at the request of the Berlin-Indian Committee. This news has been confirmed by a letter from Dalip Singh Gill addressed to the Latvian Consul General, Moscow, headed Buturskaya Pri.son Cell No. 30, Moscow, March 30th, 1921. A copy of 
this letter has come to our hands. It runs as follows: 

“There is no British representative in Moscow. I send you this petition and beg you to afford me help as you do to American .subjects, as in accordance with the text of the trade agreement between Soviet Russia and England, published in the Pravda of the 22nd March, British subjects are to be immediately released. Please clear up this matter as regards my case and obtain information regarding the possibility of returning to my native land.” 

This is Gill’s third visit to Russia. After Gill had become a Communist in the winter of 1919 he went by Aeroplane 
from Berlin to Moscow, where he met various Russian leaders, He collected a considerable amount of money and on his return to Berlin, began to work on Communist lines. In the early summer of 1920 Gill again started for Russia, supplied with funds provided by Gen. Hoffman and Talat Pasha, the object of his journey being to obtain further help from Bolsheviks. This time the aeroplane was shot down by the Poles and Gill was imprisoned for some time. On his release he returned to Berlin and trained himself in a scheme to send literature from Berlin to Russia by air and thence to India. Gill again went to Moscow in January of this year and while there was imprisoned as a British spy. His protest 
to the Latvian Consul, that as a British subject he should be released according to the terms of the trade agreement with England, contrasts strangely with his previous record. \\


\textsc{Chatto's Group and Negotiations with The Soviet}\\

It is now possible to state further developments in Chatto’s scheme for uniting revolutionary work under one head. Discussions with the Soviet Government have been going in for some time with a view to arranging a meeting in Moscow of all prominent Indian revolutionaries to settle future plans and the part which Russia would play in them. This meeting was to be held in Moscow on May 25th. After a conference 
lasting for two or three days it was hoped to place a complete scheme before the Soviet Government and before the Third International on the 1st of June. The subject of discussion was to be briefly the “best method for inaugurating a revolution in India”. The following individuals were expected to be present in Moscow: 

\noindent(1) M. N. Roy \\
(2) M.P.T. Acharya \\
(3) Abdul Rab \\
(4) Shafiq Ahmad (Recently arrived from Afghanistan and one of the members of Provisional Government in Kabul.) \\
(5) Amin Faruqui (Secretary of the Indian Revolutionary Party in Tashkent.) \\
(6) V. Chattopadhyay \\
(7) Dr. Ahmad Mansur \\
(8) B. N. Dutt. \\
(9) P. S. Khankhoji alias Aga Khan.\\ 
(10) G.A.K. Lohani (now definitely admitted as a member of the Com-mittee but not on the executive) \\
(11) Nalini Gupta \\
(12) Das Gupta \\
(13) Agnes Smedley \\
(14) Abdul Wahid \\
(15) Dr. Abdul Hafiz \\

In addition to the above, others were expected. There are 35 members on the Tashkent Indian Committee and it is probable that representatives of this Committee were to attend. From Paris Madame Cama and Rao have been invited. They were reported to be willing to go provided they 
could obtain permission from the French authorities. Srinivasha Vishwamitra (an Indian recently expelled from Denmark as an alleged Communist) and Chatto’s brother has also been mentioned in connection with the Moscow meeting, while it is said that some Indians have been invited 
from England. These selections have not been made without a certain amount of opposition from the Soviet Government and its representative in Berlin, chiefly owing to the fact that some of the names in the list are those of well known nationalists. Mrs. M. N Roy. who arrived in Berlin about the 27th of April in connection with final negotiations, in particular objected to the inclusion of Heramba Lai Gupta 
and stated that if he, and others like him, appeared in the Committee, the support of the Russian Soviet would be refused. The Soviet representative in Berlin confirmed Roy’s statement. Chatto then wired to the Russian Government that he refused to be dictated to and threatended to break off negotiations. The result was a telegram from Chicherin himself to the effect that the Berlin Indian Committee might 
bring anyone they wished to the meeting at Moscow. 

Incidentally, it is reported, there is no longer any doubt, that the Soviet are already financing the Indian Revolutionary movement. Every member who goes to Moscow meeting was to have his expen.ses paid, and was to receive a minimum 5,000 marks. Chatto has had all his debts paid (15,000 Swedish kronens). Dr. Hafiz has been given 10.000 kronens as expenses for his wife and children. These payments have been made through the Soviet representative in Berlin. 

While these difficulties were being overcome a series of preliminary meetings was held in Berlin in order to define the precise attitude the Committee should adopt, and to present a united programme to the Russian Government. This, however, after much discussion proved an impossible task. “The Friends of Freedom for India” in America, for instance, are ready to accept Russian help for a political revolution in India but they will not tolerate any Communist propaganda. The Ghadr Party of America, on the other hand, are in favour of Communism. In Berlin H. L. Gupta is against Communist propaganda, whereas B. N. Dutt is in favour of it. Barkatulla and the “Provisional Government of India” party are in favour of political revolution. M. N. Roy and his associates are out for purely Communistic work. These differences proved impossible to reconcile and it was decided that the first step should be to present a statement before the Soviet Government and the Executive of the Third International of the real position in India and thereafter as far as possible to follow their advice. \\

\textsc{Indians in Europe}

Chatto’s Group : The members of the Indian Committee were reported about the second week of May to be busy in securing passports and arranging details for their journey to Russia in view of the meeting of Indian revolutionaries in Moscow. A copy of a letter given by Chatto to fellow conspirator shows how matters were being arranged. 

%letter here%

N.B. Victor Kopp is the Soviet representative in Berlin. Vorovsky was at one time the representative of the Soviet Government in Stockholm. 

It appeals that H. L. Gupta is to remain in Berlin, where he will be in-charge of affairs during Chatto’s absence. It has been suggested that the ultimate destination of all the Indians who are going to Mo.scow will be Afghanistan. This is possible, but it is clear that their movements after the Moscow Conference will depend on the decision arrived at by the Soviet Government and up to the present no information on this point has been received.

\textsc{Indians in Berlin:} The Berlin Indian Committee has recently been suffering from an epidemic of “Spy Mania”. At their request Dalip Singh Gill has been imprisoned by the Soviet Government in Moscow. Dr. Mansur, who was formerly working with Dalip Singh Gill in Berlin is now 
being carefully watched though he is in other respects free. At present he is giving Hindustani lessons in Berlin. 

Ram Bhattacharji in Berlin is also regarded as a spy and was brought to task by the Committee, and asked to clear himself. This he did by giving certain references and by showing that he had some of his teeth knocked out by the Indian Police. Pending further enquiries the Committee will have nothing to do with him. S. K. Roy in Switzerland and Mookerji who attended the Baku Conference as an Indian delegate are 
also suspected and according to B. N. Dutta, Mookherji is to be shot at sight. Of lesser known individuals Varma and Kaul (not identical with P. N. Kaul) are also regarded as British spies. In Paris A. Ghosh is under strong suspicion. 

The arrival of Agnes Smedley (of the F.F.I.) in Berlin has already been mentioned. It has now been ascertained that she made the journey from America by enlisting as a stewardess on an American Ship under the name of Miss Bird. On landing at Danzig the only document she could produce was a paper showing her as a stewardess. She then wired to B. N. Dutta who arranged with the Berlin Foreign Office that this paper should be visaed authorizing her to proceed to Berlin. Chatto has now managed to get her a German pass- port, and it is rumoured he proposes to marry her. \\


\textsc{Chatto’s Negotiations with The Soviet}\\

It is now possible to indicate the general trend of the negotiations which have taken place between the Berlin Indian Committee and Moscow. Chatto has unsuccesfully carried out his schemes for linking up the revolutionary centres in Europe. He is in touch with American groups, such as the Friends of Freedom for Indian, and has enlisted under his banner most of the prominent Indian seditionists in Europe, and he has formulated an ambitious plan of work which only requires money to be put into execution. Having reached this stage, his next step was to approach the Soviet Government, as the leader of the Indian Communist Party, for financial assistance. To investigate his claims and the representative nature of his Society, the Soviet Government sent an agent to Berlin early in March with a view to test Chatto’s state- 
ment that his Society fully represented Indian opinion. The agent demanded a mandate signed by the well-known Indian leaders such as Gandhi, etc. Chatto could not produce the required mandate but the promised it would be shortly forthcoming and further argued, "If you could believe one man, M. N. Roy, who has no mandate from Indians why should not you believe us — a Society?” \\

\textbf{[\textsc{Note} — Chatto is reported to have hit on the idea of utilizing his sister Mrs. Naidu to approach Indian leaders on her return to India, in order to obtain their signatures. Mahamed Ali in a recent speech, in which he alluded to the intention of Government to arrest him on some “absurd pretexts”and said that such a pretext might be a charge of conspiring with Bolshevists through Mrs. Naidu.] }\\

Ultimately it appears the agent was satisfied and promised to recognize Chatto’s organization and returned to Moscow having promised that he would send them 100,000 roubles. It was at this stage that Chatto wrote to one of his confederates explaining the situation, and the statement that Chatto had at last arranged for financial support from the Soviet, which appeared in a recent Weekly Report, was based on this letter. Later information, however, shows that negotiations have not yet been concluded. 

In the third week of March a letter and telegram from Moscow was received by Chatto stating that it would be impossible to send the promised assistance as M. N. Roy stood in the way. Roy, it seems, advised the Soviet that Chatto s group is a Nationalist Party and not a Communist 
Party. “They are the very people who were Nationalists at the time of the war for German money, now they have found Russian money and are Communists.” The Soviet agent stated that it was against communist principles to help Nationalists. After receiving this news the Indian Committee in Berlin were at a loss how to proceed, discussed innumerable plans which included a scheme for the murder of Roy. They were considering the advisability of going in a body to Russia and of making a final appeal for support when the agent of the Soviet Government returned to Berlin at the beginning of April. 

At the date of latest information negotiations had been resumed. Chatto had wired to Abdul Wahid, B. N. Das Gupta and Dr. Hafiz, probably with a view to calling them to Berlin and possibly with the intention of taking them with him When he goes to Moscow. 

The opposition of Roy is clearly responsible for the set back which Chatto has experienced in his negotiations with the Soviet. Information which we recently received stated that Roy’s attitude has now been made clear, as he has made an offer of cooperation with the Indians in Berlin provided they accept his terms. The terms include strict adherence to communism and the acceptance or Roy as leader. It is the 
last condition which is responsible for hesitation to close with Roy’s offer. Heramba Lai Gupta, in particular, who has 
much influence in the party, is absolutely opppsed to working under Roy. 


\begin{center}
    \textbf{\textsc{D. Experience of Dr. Bhupenoranath Dutta In Moscow} }
\end{center}

\textit{Dr. Bhupendranath Dutta. youngest brother of Swami Vivekananda, was one of the most active Indian revolutionaries outside India. When Berlin Committee was formed during the World War I to co-ordinate the activities of Indian Revolutionaries outside India and to secure German arms and assistance to organise armed resistance against British Government in India in order to attain India 's independence, Dr. Bhupendranath Dutta was elected as Secretary of the Berlin Committee. As Secretary. Dr. Dutta was the Chief Executive of the Berlin Committee 
which was given the status of an Emigre Government by the Government of Germany. Virendranath Chattopadhayya, younger brother of Shrimati Sarojini Naidu, an Indian revolutionary of outstanding merit and reputation, was a pillar of the Berlin Committee and due to his efforts Dr. 
Bhupendranath Dutta could become the Secretary of the Berlin Committee. }

\textit{The October Revolution in Russia in 1917 fired the imagination of the leaders of the Berlin Committee and they were gradually attracted towards Marxist ideology and became eager to make contacts with the leaders of the October Revolution. But their sole aim was ‘independence ’ of India. After Germany’s defeat in World War I there was a turmoil in the Governmental set-up in Germany and consequently the leaders of the Berlin Committee faced great uncertainties. In the subsequent period Dr. Bhupendranath Dutta, Virendranath Chattopadhayya and other leaders of the Berlin Committee established contacts with Bolshevik Party and the Communist International and also paid a visit to Moscow. }

\textit{Dr. Bhupendranath Dutta gave an account of their experience in Moscow in his book (in Bengali), entiled: "Aprakasita Rajnitik Itihash. " Some relevant excerpts from this book are given here (in English Translation) for the readers to make their own assessment of these historical 
facts. }\\

\textbf{Aprakasita Rajnitik Itihash*}

“The Russian revolution had stirred the minds of Indian revolutionaries living in Europe. Some of them were aligned to Left-Socialist ideas before that. Virendranath Chattopadhayya and Trimul Acharya were members of the Communist-Anarchist Party in Paris. The writer (i.e.. Dr. 
Bhupendranath dutta) in his student days became a member of the Bronxpark Socialist Club in New York. Madame Cama was left-minded. I heard that she was sympathetic to Russian Bolshevik ideology. In 1925, when the writer (i.e.. Dr. Bhupendranath Dutta) departed from her in Paris, 
she said to this writer (i.e. Dr. Dutta) mixed in English and French : “keep your flag high like Admiral Togo and organise the ouvriers et paysans of India." The French Socialist leader Jaures and Longuet, the grandson of Karl Marx, were their friends. The Indian revolutionaries while staying abroad received support only from the leftists of Europe. The Socialist leader, Hyndman, of England, the Russian Anarchist leader Peter Krapatkin, the Bolshevik leader Lenin - all of them wanted independence of India. 

\footnote{*Excerpts from ‘Aprakasita Rajnitik /fi'AasA' (Unpublished Political History) }


So, when one of them established a new state order through revolution, was it not natural that all left-minded persons would go there ? For this reason the vision of all types of revolutionaries was fixed on Moscow. Then Moscow was known as “New Mecca.” 

“The new Revolution of Russia had greatly influenced the minds of left-minded Indian revolutionaries. Of the Indian revolutionaries, who were aligned to Moscow, had convened a conference in Stockholm in 1920 and decided their course of action. The writer (i.e.. Dr. Dutta), Pandurang Khankhoje- who came from Iran, Birendranath Das Gupta and Biswamitra-an Indian student staying in Denmark, assembled there. After discussion, it was decided that: those who wanted to remain Nationalist-minded they should form an organisation and start work: and, those who were left-minded i.e., those who believed in the Communist ideology-would form a separate organisation to start work but both the groups would struggle for India’s independence. This programme of action was intimated to Gadar Party in America. The expenses for this conference were 
borne by the Sweedish Communist leader, Strome. We also met the Communist leaders of Sweden. They said that they could not do anything from there and we should go to Moscow for necessary arrangement. So, it was decided to send Virendranath Chattopadhayya to Moscow. Before this conference M.N. Roy had also invited Virendranath to Moscow." 

“In the winter of 1 920 Chattopadhayya went to Moscow and returned after discussions with the leaders there. The leaders of the Communist International told him there; “You bring other (Indian) revolutionaries here, form a Committee and start work.” He (Chattopadhayya) also agreed to bring other (Indian) revolutionaries to Moscow and on this understanding returned to Berlin.” 

“During this time Borodin returned to Berlin. The writer (i.e.. Dr. Dutta) introduced him to Chattopadhayya. He (i.e., Boradin) said : “In the meantime you start work in Berlin after forming a Committee and arrange to establish the Communist International’s contact with India.” 

“Before this development i.e., before Chattopadhayya’s departure for Moscow the writer (i.e.. Dr. Dutta) received a letter from him from Stockholm containing information that one young man named Ghulam Ambia Lohani was on his way to Moscow via Berlin and arrangement for his stay 
in Berlin had to be made. As far as I remember Chattopadhayy a got the name of this youngman from his sister Shrimati 
Sarojini Naidu A few days after receipt of this letter one lame young man reached the writer’s residence at Ansbro-kher Street (in Berlin ). After acquaintance he said that he hailed from Pabna district (now in Bangladesh-Ed.); he came to study law in London and was on his way to Moscow. He secured his passage money from the Soviet Embassy in London. Later on, after staying with him for a few years the 
writer came to know ail about him. In London he lived a fast life and all along lived like that. He had earlier studied at Aligarh (University) for some time before. He had married a French woman in London but at that time he had no con- nection with her. He was a very intelligent person and could speak English well.” 

“The Indian Revolutionary committee was set up in Berlin as proposed by Borodin. Lohani reached Berlin after this and Chattopadhayy a also returned from Moscow. Chattopadhayy a wanted to include Lohani in the group that would visit Moscow...” 

“During this time one evening a young woman came to the writer’s (i. e.. Dr. Dutta’s) place and asked him : ‘Are you Mr. Dutta? Thereafter she introduced herself as Agnes Smedley. She was engaged in Indian revolutionary work in New Y ork in association with Taraknath Das and Sailendranath Ghosh and faced imprisonment for four years along with them. When Taraknath Das and others were penniliess she 
supported them from her earnings. 

After the First World War she was a front-ranking worker of the organisation, “Friends of Indian Freedom.” She was acquainted with all the Indian revolutionaries living in U. S. A. The Indians did not see any woman so devoted to India like her except Sister Nivedita. She was born in a poor worker’s family in Pennsylvania in U.S.A. She was a stenographer and also a journalist. Her articles were published in the Modern Review, the monthly journal published from Calcutta ” 

“At last all of us started for Moscow. Chattopadhyya, Agnes Smedley and Khankoje went together. Nalini Gupta was also taken to Mo.scow. The writer (i.e.. Dr. Dutta) and Birendranath Das Gupta went separately. Except Nalini Gupta we all were members of the “Indian Revlutionary 
Committee.” 

“On reaching Moscow the writer (i.e. Dr. Dutta) again met Borodin. Borodin told Chattopadhayya : “I suggested formation of Indian Revolutionary Committee (in Berlin) so as to delay your visit to Moscow and advised you to start work in Berlin.” Its meaning was understood by us sub.se- 
quently. The writer (i.e.. Dr. Dutta) again met Borodin. Borodin said : “You have come ; stay here for six months; you will meet everybody. This is how we work here.” Subsequently it was learnt that he (i.e., Borodin) advised some one of our group : “ Y ou hold a conference with Mahendra Pratap and others and form a Committee.” 

“On reaching Moscow the members of the Berlin Committee met Acharya and Peshwari. Roy was then in Tashkent. Subsequently he (Roy) and Abani returned to Moscow with their wives and a few Muzahareens. In the mean time the writer (i. e. Dr. Dutta) met Sabitsky, Office Secretary of the 
Communist International, and told him ; “We are waiting here so long but nothing is progressing.” He (Sabitsky) replied : “Let Roy return, then a Commission will be set up and programme of work will be decided.” Before this discussion we had the impression that we came to Russia as 
Members of the Indian Revolutionary Committee of Berlin. In this regard he (Sabitsky) said : “We attach no value to this Committee.” During this time Chattopadhayya and Agnes Smedley declared themselves as husband and wife according to the social practice of the Communists. 

At last a Commission started its work to finalise the programme of work. Borodin, Koeltch, Rutgers and all Indians assembled here. Rutgers was the President of this Commission. He asked all members of the Indian Team individually to give his opinion. 

But Chattopadhayya said ; “We belong to one party. One member representing the whole party will give the opinion.” In reply Borodin said : “ we do not know any party; we will select the appropriate person after scrutiny.” Its implication was that they would select their ‘own person’ 
according to their choice. The members of the Berlin Committee said : “Then we will boycott this Commission” 

Thereafter, three months elapsed. No new Commission was set up. Subsequently Rakosi became the Secretary of a Commision and started work promptly. He convened the meeting of the new Commission to decide the programme of action for the Indian Revolutionaries. Rakosi was well- 
known to Nalini Gupta. On his way to Moscow Nalini was a co-passenger with Rakosi in a Steamer. Nalini informed the writer (i.e. Dr. Dutta) that Rakosi said : “I am not concerned either with Roy’s group or with Dutta’s group : I will assemble both the groups and form a Party.” 

The members of Chattopadhayya ’s group started propagating ; “There is no scope to organise workers movement in India, only nationalist movement should be organised.” They propagated this theory to the delegates of all countries. This created division in the party. Subsequently when a Commission was set up under the Chairmanship of James Beles all Indian Revolutionaries assembled there. On finding Borodin in this meeting chattopadhayya asked the President, “In what capacity he (Borodin) was there.” The President replied : “He is present here as a member of. the Commission.” Chattopadhayya then said: “If it is so, I will boycott this Commission.” The President ignored this threat. Chattopadhayya left the meeting. At that time Borodin asked one of the Indians: “ In what capacity you have joined this Commission-as an individual or as a member of the party?” 

He replied: “As an individual.” In this Commission a thesis of Roy which he had already printed and sent to the International, was distributed. The thesis of this writer (i.e., Dr. Dutta) and of his companions was also placed in the meeting along with a thesis of Chattopadhayya.” Chattopadhayya sent his thesis to Lenin. Lenin gave him a reply. \\



“Dear Comrade Chattopadhayya, \\
\indent I have read your Thesis. I am in agreement with your views. We will have to destroy British imperialism. When I can meet you will be conveyed to you by my Secretary. 
\begin{flushright}
Sd/- V. Ulianov (Lenin)”*
\end{flushright} 
\footnote{* (Translation from Bengali version) }

The writer (i.e.. Dr. Dutta) sent his thesis to Lenin through Rakosi. Lenin gave the following reply: \\


\indent“To \\
\indent Comrade Bhupendranath Datta, \\
\indent Dear Comrade Datta, \\
\indent I have read your Thesis. We should not discuss about the Social 
classes. I think we should abide by my Thesis on colonial 
question. Gather statistical facts about Peasants’ League if they 
exist in India. 

\begin{flushright}
Yours, 
V. Ulianov (Lenin)” 
\end{flushright}

The contention of the writer’s thesis was: So long as the foreign power was ruling over us. We should unite all classes of people and try to complete the Political Revolution. In this connection Marx’s “Civil War in France" was referred to and views of Marx were quoted. But from the very beginning Communist Party should be organised which, after the Political Revolution, should strive for Socialist Revolution to establish Socialism in India. 

“The members of the different Indian groups were in this Commission. James Beles was the President of this Committee and he was representing the Communist International. Besides, Borodin, Trionosky, Dr. Thalheimer and Rakosy, Secretary of the Communist International were in this Commission. Dr. Thalheimer was the Editor of the German Communist Party’s organ, “FREIHEIT.” 

“The Commission sat for two days. On the first day Lohani read his thesis. Khankhoje said “If you want to form a Communist Party, do it ; if you want to organise working class movement, do it.” On the second day the members of our group were to read the thesis. The writer (i.e. Dr. Dutta) said, “I have sent a copy of my thesis to Lenin. This thesis is quite lengthy, for this reason I am reading the gist of it.” While 
reading the thesis the writer referred to the writings of Karl Marx a number of times. On noticing it Dr. Thalheimer commented in jest ; “Our Indian Comrades have read too much of Karl Marx.” During this time Borodin asked me, “Where is the difference between your’s and Roy’s views?” The writer (Dr. Dutta) said in reply : “Roy is unwilling to work with the nationalists. In the revolutionary movement in India whom one will find except the nationalists?” On hearing it Borodin said : “It is correct.” At last Agnes Smedley placed her brief comments on India to the President of the Commission instead of reading it. On reading it Beles (Chairman of the Commission) asked her : “Comrade, being a member of IWW how could you be such an English-hater?” Finally Roy stood up and said ; “There is a proposal to form a new Communist Party (for India). But Communist Party of India has already been formed. Why not these Comrades join this Party ?” As soon as Roy said it. the members of the writer’s (Dr. Dutta) group protested against it and put up a typewritten protest letter to Rakosi. It was written in this protest letter : “We all Communist-minded people want to form a Communist Party of India. We have sent message to our country accordingly and preparations are going on in this direction. But suddenly without informing us a Communist Party of India was formed in Moscow.* We do not recognise this Party and we are not willing to give any co-operation to it.” 


\footnote{•It is actually "Tashkent”.}

On hearing this report everyone who had come from Germany became unwilling to stay in Moscow any longer. All of them wanted to return back. 

Finally the Time came to leave Moscow. The writer (Dr. Dutta) went to meet Roy and his wife. Roy told him (Dr. Dutta) : “You stay here and take responsibility for all work. Do not. feel sorry for my victory.” The writer (Dr. Dutta) said in reply : “ Roy, it is not true. Neither have you won, nor am I defeated. It is the mandate of Malibureau that a small bureau will be set up in Moscow. Now you make your career here. I make my career elsewhere.” Then Roy said: “The world is big enough for everybody.” 
