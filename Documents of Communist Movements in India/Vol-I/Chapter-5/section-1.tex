\noindent(In Ahmedabad Congress, 1921)\\
\\

It is on record that Moulana Hasrat Mohani placed a resolution in Ahmedabad Congress in 1921 for complete independence of India. 

It is already stated in this volume (Page 123) that M. N. Roy had sent a detailed appeal in the form of a memorandum to Ahmedabad Congress in 1921 which was to be presided over by Desh Bandhu Chittaranjan Das. M. N. Roy had also sent this memorandum to several individuals ill India by post. M. N. Roy had .said in his memoir (Page 547, 1964 Edition) “But ultimately it did come before the Congress legally. Two delegates from Ajmer reprinted the document on their signatures and submitted a resolution that the Congress should discuss it. It was further .said that the appeal gave Maulana Hazrat\footnote{ In Dr. Pattabhi Sitaramayya’s “ The History of the Indian National Congress”, the name appears as Maulana Hasrat Mohani (vide page 228. vol. 1. 1946 Edition). } Mohani the idea to move for the first time in a Congress Session the resolution that complete independence was the goal of the Indian National Congress.” M. N. Roy further stated in his memoirs; “That was the first item in the programme of national revolution outlined in the appeal. The Congress Session rejected Hazrat Mohani’s re.solution and the mover was arrested soon afterwards. But the ideal of complete independence gained ground in the ranks of the Congress, although still for some time the latter officially waged war against the “bureaucracy”.

That Maulana Hasrat Mohani's resolution placed in Ahmedabad Congress in 1921 for complete independence of India was a very significant event before the whole Ahmedabad Congress is borne out by what Dr. Pattabhi Sitaramayya had said in his “ The History of the Indian National Congress”, Volume I, page 228, quoted below: 

“We must now refer to a debate initiated by Maulana Hasrat Mohani, who proposed to define ‘Swaraj” in the creed as “Complete Independence, free from all foreign control.” At this distance of time, one is apt to look upon it as the most natural- sequence of all that had happened, and may even wonder why it should have been resisted at all by the Congress or by Gandhi. But, at the time, Gandhi was obliged to speak out frankly: 

“The levity with which the proposition has been taken by some of you has grieved me. It has grieved me because it shows lack of responsibility. As responsible men and women, we should go back to the days of Nagpur and Calcutta.” 
