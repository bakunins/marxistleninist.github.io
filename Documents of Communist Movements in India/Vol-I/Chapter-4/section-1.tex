\textbf{Sent by M. N. Roy from abroad}\\
\\

\noindent Fellow Countrymen, 

You have met in a very critical moment of the history of our country to decide various questions affecting gravely the future of the national life and progress. The Indian nation today stands on the eve of a great revolution, not only political but economic and .social as well. The vast mass of humanity, which inhabits the great peninsula, has begun to move towards a certain goal; it is awakening after centuries of social stagnation resulting from economic and political oppression. The National Congress has placed itself at the head of this movement. Yours is a very difficult task, and the way before you is beset with obstacles almost insuperable and pitfalls treacherous and troublesome. The mission of leading the people of India onward to the goal of national liberation is great and you have made this great mission your own. The National Congress is no longer a holiday gathering engaged in idle debates and futile resolution making; it has become a political body — the leader of the movement of 
national liberation.

The newly acquired political importance obliges the Congress to change its philosophical background; it must cease to be a subjective body; its deliberations and decisions should be determined by the objective conditions prevailing and not according to the notions, desires and prejudices of its leaders. It was so when the Congress, national in name only, was the political organ which expressed the opinions and aspiration of a small group of men who ruled over it. If the old Congress dominated by the Mehta-Gokhale-Bose-Banerji combination is dead and discarded from the field of pragmatic politics, it is because these men wanted to build a nation after their own image; they did not and could not take into consideration the material they had to work with; they failed to feel the pulse of the people; they knew what they thought and wanted, but they did not know, neither did they care to know what the people — the people which constituted that nation which their Congress also pretended to represent — needed for its welfare, for its progress. The old Congress landed in political bankruptcy because it could not make the necessities of the common people its own; it took for granted that its demands for administrative and fiscal reforms reflected the interest of the man in the street; the “grand old men” of the Moderate Party believed that intellect and learning were their inviolable mandates for the leadership of the nation. This lamentable subjectivism, originating from defective, or total absence of understanding of the social forces that underlie and give strength to all movements, made the venerable fathers of Indian Nationalism betray their own child; and it led them to their own ruin, disgrace and political death. You, leaders of the new congress, should be careful not to make the same mistake; because the same mistake will lead to the same disaster. 

The programme of the Congress under the leadership of the Non-cooperation Party, is to attain Swaraj within the shortest possible time. It has discarded the old impotent tactics of securing petty reforms by means of constitutional agitation. Proudly and determinedly, the Congress has raised the standard with “Swaraj within a Year” written on it. Under this banner, the people of India are invited to unite; holding 
this banner high you exhort them to march forward till the goal is reached. This is indeed a noble cause. It is but natural that the people of India should fight for the right of ruling itself. But the function of the Congress, as leader of the nation, is not only to point out the goal, but to lead the people step by step towards that goal. From its activities of the last year, it is apparent that the Congress understands its task and is trying to find the best way of executing it. The people must be infused with enthusiasm to fight for Swaraj; they must be united in this struggle, because without union the goal will not be attained. 

Why was the old Congress discredited? Because it could not make the national question a vital problem for the people. Under the old leaders, the Congress was caught in the cesspool of political pedantry and petty reformism. Not much greater results can be expected if these are to be replaced by abstract idealism and political confusion. In order to deserve the name and to be able to execute the difficult task set before it, the National Congress must not permit itself to be carried away by the sentiment and idealism of a handful of individuals however great and patriotic they may be; it must take into consideration the cold material facts; it must survey with keenness the everyday life of the people — their wants and sufferings. Ours is not a mere political game; it is a great social struggle. 

The greatest problem before the 36th Congress is how to enlist the full-hearted support of the people in the national cause, how to make the ignorant masses follow the banner of Swaraj. In order to solve this problem, the first thing necessary is to know what is that ails the masses? What do they want? What is needed for improving the immediate environment of their material existence? Because only by including the redress of their immediate grievances in its programme will the Congress be able to assume the practical leadership of the masses of the people? 

Several thousand noisy, irresponsible students and a number of middle-class intellectuals followed by an ignorant mob momentarily incited be fanaticism, cannot be.the social basis of the political organ of a nation. The toiling masses in the cities, the dumb millions in the villages must be brought into the ranks of the movement if it is to be potential. How to realize this mass organization is the vital problem before the Congress. How can the man working in the factories or labouring of the fields be convinced that national independence will put an end to his sufferings? It is not a fact that hundreds of thousands of workers employed in the mills and factories owned by rich Indians, not a few of whom are leaders of the national movement, live in a condition unbearable and are treated in a manner revolting? Of course by prudent people such discomforting questions would be hushed in the name of the national cause. The argument of these politicians is “let us get rid of the foreign domination first.” Such cautious political acumen may be flattering to the upper classes; but the poor workers and peasants are hungry. If they are to be led on to fight, it must be for the betterment of their material condition. The slogan which will correspond to the interest of the majority of the population and consequently will electrify them with enthusiasm to fight consciously, is "Land to the peasant and bread to the worker”. The abstract doctrine of national self determination leaves them passive; personal charms create enthusiasm loose and passing. 

How can the Congress expect to arouse lasting popular enthusiasm in the name of the Khilafat and by demanding the revision of the treaty of Sevres? The high politics behind such slogans may be easy for the learned intellectuals to understand; but it is beyond the comprehension of the masses of Indian people who have been steeped in ignorance not only by the foreign ruler, but by our own religious and social institutions. Such propaganda based on the questionable doctrine of utilizing the ignorance of the masses in order to make them do the bidding of the Congress, cannot be expected to produce the desired result. If the masses of the Indian people are to be drawn into the struggle for 
national freedom, it will not be done by exploiting their ignorance. Their consciousness must be aroused first of all. They must know what they are fighting for. And the cause for which they fight must include their immediate needs. What does the man in the street need? The only aspiration of his life is to get two meals a day, which he hardly achieves. And such are the people who constitute 90\% of the nation. Therefore, it is evident that any movement not based on the interests of these masses cannot be of any lasting importance or of formidable power. 

The programme of the Congress has to be denuded of all sentimental trimmings; it should be dragged down from the height of abstract idealism; it must talk of the things indispensable for mortal life of the common human being; it must echo the modest aspirations of the toiling masses; 
the object for which the Indian people will fight should not be looked for somewhere in the unknown regions of Mesopotamia or Arabia or Constantinople; it should be found in their immediate surroundings, in their huts, on the land, in the factory. Hungry mortals cannot be expected to fight for an abstract ideal. The Congress must not always urge the people, which can be called the classical example of suffering and sacrifice personified, to suffer and sacrifice only. The first signs of the end of their age-long suffering should be brought within their vision. They should be helped in their economic fights. The Congress can no longer defer the formulation of a definite programme of economic 
and social reconstruction. The formulation of such constructive programme advocating the redress of the immediate grievances of the suffering masses, demanding the improvement of their present miserable condition, is the principal task of the 36th Congress. 

Mr. Gandhi was right in declaring that “the Congress must cease to be a debating society of talented lawyers”, but if it is to be, as he prescribes in the same breath, an organ of the “merchants and manufacturers”, no change will have been made in its character, in so far as the interest of the majority of the people are concerned. It will not be any more national than its predecessor. It will not meet any more dignified end. If it is to represent and defend the interest of one class viz. the merchants and manufacturers, it cannot but fail to take care of the common people. The inevitable consequence of this failure will be the divorce of the Congress from the majority of the nation. The merchants and manufacturers alone cannot lead the national struggle to a successful end; neither will the intellectuals and petty shop- 
keepers add any appreciable strength to the movement. What is indispensable is the mass energy ; the country can be free, Swaraj can be realized, only with the conscious action of the masses of the people. In order to be able to execute its task, the Congress must know how to awaken the mass energy, how to lead the masses to the field of resolute action. But the tactics of the Congress betray its lamentable indifference to and lack of understanding of the popular interest. The Congress proposes to exploit the ignorance of the people and expects them to follow its lead blindly. This cannot happen. If the leader remains indifferent to the interest of the follower, the two will soon fall asunder. The masses are awakening; they are showing signs of vigour; they are signifying their readiness to fight for their own interest; the 
programme of using them as mere instrument, which are to be kept in their proper place, will soon prove ineffective. If the Congress makes the mistake of becoming the political apparatus of the propertied class, it must forfeit the title to the leadership of the nation. Unfailing social forces are constantly at work; they will make the workers and peasants conscious of their economic and social interests, and ere long 
the latter will develop their won political party which will refuse to be led astray by the upper class politicians. 

Non-cooperation cannot unify the nation. If we dare to look the facts in the face, it has failed. It is bound to fail because it does not take the economic laws into consideration. The only social class in whose hand non-cooperation can prove to be a powerful weapon, that is the working class, has not only been left out of the programme, but the prophet of non-cooperation himself declared “it is dangerous to make political use of the factory workers”. So the only element, which on account of its social-economic position, could make non-cooperation a success is left out. 

The reason is not hard to find; the defenders of the interests of the merchants and manufacturers betray unconsciously their apprehension lest wage-earners should be encouraged to question the right of exploitation conceded to the propertied class by all respectable society. The other 
classes which are called upon to non-cooperate, being dependent economically on the present system, cannot separate themselves from it, even it is damned as “satanic” by the highest authorities. 

Non-cooperation may prove to be a suitable weapon to fight, or better said, to embarrass the foreign bureaucracy, but at best, it is merely destructive. The possible end of foreign domination, in itself, is not sufficient inducement for the people at large. They should be told in clear terms what benefit would accrue to them from the establishment of Swaraj. They should be convinced that national autonomy will help them solve the problem of physical existence. Neither will empty phrases and vague promises serve the purpose; it has to be demonstrated by the acts of the Congress that proposes to achieve the amelioration of the people’s suffering, and that it will not neglect the immediate needs of the poor in quest of abstract freedom to be realized at some future date. 

For the defence and furtherance of the interests of the native manufacturers, the programme of Swadeshi and boycott i.s plausible. It may succeed in harming the British capitalist class and thus bring an indirect pressure on the British Government, though being based on wrong economics, the chances of its ultimate success are very problematical. But as a slogan for uniting the people under the banner of the 
Congress, the boycott is doomed to failure; because it does not correspond, nay it is positively contrary, to the economic condition of the vast majority of the population. If the Congress chooses to base itself on the frantic enthusiasm for burning foreign cloth, it will be building castles on a bed of quick sand. Such enthusiasm cannot last; the time will soon come when people will feel the scarcity of cloth and as long as there will be cheap foreign cloth in the market there can be no possibility of inducing the poor to go naked rather than to buy it. The Charka has been relegated to its well-deserved place in the museum; to expect that in those days of machinery it can be revived and made to supply the need of 320 millions of human beings, is purely visionary. The boycott will enlist the support of the manufacturers, but it will never receive a dependable response from the consumers. Then, all the doctrines of purifying the soul may be good for the opulent intellectuals, but their charm for the starving millions cannot be permanent. Physical needs know no bounds, and a political movement cannot be sublimated beyond material reasons and necessities. They are mistaken who say that Indian civilization is purely spiritual, and that the Indian people are not subject to the same material laws that determine the destinies of the rest of humanity. 

While for any serious or lasting purposes, the Non-cooperation programme cannot be said to have achieved a small part of what was expected, the 36th Congress intends to go a step further on the road of Non-cooperation. To their great discomfiture the leaders of the Congress observe the popular enthusiasm evoked by Khilafat agitation, and Non-cooperation subsiding day by day. The enlisting of several lakhs of members and the raising of the Tilak-Swaraj Fund cannot be accepted as a clear reflection of the popular support behind the Congress. Pessimism about the solidity of ranks and tenacity of purpose of the Non-cooperation demonstration has of late been repeatedly expressed by responsible Congress leaders both from the press and platform. To enlist his name in the Congress register and to contribute a rupee to the Swaraj Fund does not necessarily imply that a member will be ready to take active part in the struggle. In order to keep the artificially fomented popular enthusiasm alive, the leaders of the Congress have been looking for new diversions of an exciting character. But either consciously or unconsciously, they would not lay their hand on the real cause of popular discontent and develop their discontent by helping the masses acquire consciousness. Instead, another irresponsible step has been taken. Without waiting for the annual Congress, the All-India Committee has sanctioned Civil Disobedience. But the very language of the resolution shows that its authors themselves are in doubt as to whether it can be carried into practice any better than the other aspects of Non-cooperation. The resolution asks “those who could support themselves to leave the government services”. Considering the fact that the proportion of the government employees unable to make both ends meet one day without their miserable salary, is almost 90\%, it cannot be expected that the response to this ukase will be very imposing. 

Civil Disobedience when carried into practice, will be some sort of a national strike. If everybody stops working, the government will be paralysed. But is the Congress certain that everybody will readily respond to its call? If it is, then it betrays lamentable ignorance of the material condition of the people, as well as of the economic laws that determine all social forces and political actions. On leaving their civil and military occupations, thousands and thousands of people will be without any means of livelihood; is the Congress in a position to find work for them? And it should not be forgotten that the lower middle-class element employed in the government departments, will never stoop to manual work. The Congress leaders seem to appreciate the complexity of the situation; because, in the words of Mr. Gandhi, “they are not prepared to provide employment for those soldiers who would leave the army”. With the disastrous effects of the exodus of the Assam plantation workers still fresh in memory, how can it be expected that the same tactics would not be followed by the same result in the future? The political organ of a nation cannot execute its task only with popular demonstrations. Our object is not confined to bothering the government; we are struggling for freedom. It cannot be realized unless the activities of the Congress are determined by a constructive programme; unless the leadership of the Congress becomes more responsible and less demagogic. 

Taken light-heartedly, the resolution of Civil Disobedience will end in making the Congress ridiculous. Because, in spite of all optimism, all enthusiasm, the Congress does not represent the interests of all the sections and classes of which the nation is composed. Much less does it advocate the material welfare of the workers and peasants who form the overwhelming majority of the nation. What is the use of speaking in high-sounding language when the speeches are not backed up by action, determined and permanent? The spirit of the people cannot be raised by such impotent tactics; nor is the government terrorised. They only discredit the speaker, sooner or later. The threat of declaring Jehad unless the Khilafat is redeemed has become too hackneyed; the deferring of the establishment of Swaraj month by month fails to inspire confidence in thinking people. Why do these bombastic resolutions of the Congress never come out of the airy realm of words? Because the Congress does not determine its tactics in accordance with the play of social forces. 

It is simply deluding oneself to think that the great ferment of popular energy expressed by the strikes in the cities and agrarian riots in the country, is the result of the Congress or, better said, of the Non-cooperation agitation. No, it is neither the philippics against the “satanic western civilization”, nor the constant reiteration of the Punjab wrongs, nor the abstract formula of Khilafat that have awakened the 
discontent of the wretched masses, who appear to have once and for ail shaken off the spirit of passive resignation. The cause of this awakening, which is the only factor that has added real vigour and a show of majesty to the national struggle, is to be looked for in their age-long economic exploitation and social slavery. The mass revolt is directed against the propertied class, irrespective of nationality. This 
exploitation had become intense long since, but the economic crisis during the war-period accentuated it. The seething discontent among the masses, which broke out in open revolt after the war, was not, as the Congress would have it, because the Government betrayed all its promises, but because the abnormal trade boom in the aftermath of the war intensified the economic exploitation to such an extent that people were desperate, and all bonds of patience were broken. 

Newly developed industries brought hundreds and thousands of workers to the crowded cities where they were thrown into a condition altogether revolting. Sudden prosperity of the merchants and manufacturers brought in its train increased poverty and suffering for the workers. City life
opened new visions to the workers hitherto resigned to their miserable lot as ordained by Providence. The inequality of wealth and comfort became too glaring, the worker got over the lethargic resignation typical of the Indian peasantry, and rebelled. His revolt, under such circumstances could not have been against this or that government; it was against the brutal system that wanted to crush him to the dust. Mass revolt is alarmingly contagious. The spirit was soon carried to the villages by various channels, and resulted in agrarian riots, which today are spreading like wild-fire all over the country. These are the development of the social forces generated by objective conditions. The political movement must give up the pretention of having created these force: . but must bend its head before their majestic strides and adapt itself to their action and reaction. It is these social forces which lend potentiality and real strength to the political movement. In 
fact every political movement is the outcome of the development of certain social forces.   

What has the Congress done to lead the workers and peasants in their economic struggle? It has tried so far only to exploit the mass movement for its political ends. In every strike or peasant revolt the non-cooperators have sacrificed the economic interest of the strikers for a political demonstration. The Congress from its intellectual, ideological and material aloofness, demands Swaraj and expects the masses 
of population to follow it through thick and thin. It does not hesitate to call upon the poverty-stricken workers and peasants to make all kinds of sacrifices, sacrifices which are to be made in the name of national welfare, but which contribute more to the benefit of the native wealthy than to harm the foreign ruler. The Congress claims the political leadership of the nation, but every act betrays its ignorance of or 
indifference to the material interest of the majority of the people. So long the Congress does not show its capability and desire to make the everyday struggle of the masses its own, it will not to be able to secure their steady and conscious support. Of course, it should not be forgotten that with or without the leadership of the Congress, the workers and peasants will continue their own economic and social struggle and eventually conquer what they need. They don’t need so much the leadership of the Congress but the latter’s political success depends entirely on the conscious support of the masses. Let not the Congress believe that it has won the unconditional leadership of the masses without having done anything to defend their material interests. 

His personal character may lead the masses to worship the Mahatmaji; strikers engaged in a straggle for securing a few pice increase of wages may shout "Mahatmaji-ki-jai"; the first fury of rebellion may lead them to do many things without any conceivable connection with what they are really fighting for; their newly aroused enthusiasm; choked for ages by starvation, may make them burn their last pieces of loin cloth; but in their sober moment what do they ask for? It is not political autonomy, nor is it the redemption of the Khilafat. It is the petty, but imperative necessities of every day life that egg them on to the fight. The workers in the cities demand higher wages, shorter hours, better living conditions; and the poor peasantry fight for the possession of land, freedom from excessive rents and taxes, redress from the exorbitant exploitation by the landlord. They rebel against exploitation, social and economic; it does not make any difference to them to which nationality the exploiter belongs. Such are the nature of the forces that are really and objectively revolutionary; and any change in the political administration of the country will be effected by these forces. The sooner the Congress understands this, the better. 

If the Congress aspires to assume the leadership of the masses without founding itself upon the awakening mass energy, it will soon be relegated to the dead past in order to share the ignominy of its predecessor. To enlist the conscious support of the masses, it should approach them not with high politics and towering idealism, but with the readiness to help them secure their immediate wants, then gradually to lead them further ahead It is neither the Khilafat cry, nor the Boycott resolution, nor the absurd doctrine of “back to the Vedas with Charka in hand”, nor the scheme of making the middle-class intellectuals and petty shop-keepers declare a national strike that wilt unite the majority of the nation behind the Congress. Words cannot make people fight; they have to be impelled by irresistible objective forces. The oppressed, pauperized, miserable workers and peasants are bound to fight, because there is no hope left for them. The Congress must have the workers and peasants behind it; and it can win their lasting confidence only when it ceases to sacrifice them ostensibly for a higher cause, namely the so- called national interest, but really for the material prosperity of the merchants and manufacturers. If the Congress would lead the revolution which is shaking India to the very foundation, let it not put its faith in mere demonstrations and temporary wild enthusiasm. Let it make the immediate demands of the Trade Unions, as summarized by the Cawnpur workers, its own demands; let it make the programme of the Kisan Sabhas its own programme, and the time will soon come when the Congress will not stop before any obstacle; it will not have to lament that Swaraj cannot be declared on a fixed date because the people have not made enough sacrifice. It will be backed by the irresistible strength of the entire people consciously fighting for the material interest. Failing to do so, with all its zeal for Non-cooperation, for all its determination to have the Sevres treaty revised, despite its doctrine of Soul-Force, the Congress will have to give in to another organization which will grow out of ranks of the common people with the object of fighting for their interests. If the Congress wants to have the nation behind it, let it not be blinded by the interest of a small class; let it not be guided by the unseen hand of the “merchants and manufacturers" who have replaced the “talented lawyers” in the Congress, and when the present tactics seek to install in the place of Satanic British. 

While the Congress under the banner of Non-cooperation, has been dissipating the revolutionary forces, a counter-revolutionary element has appeared in the field to mislead the latter. Look out, the revolutionary zeal of the workers is subsiding, as shown by the slackening of the strike movement; the Trade Unions are falling in the hands of reformists, adventurers and government agents; the Aman Sabhas are 
capitivating the attention of the poor peasants by administering to their immediate grievances. The government knows where lies the strength of the movement; it is trying to divorce the masses from the Congress. This clever policy directed by master hands, cannot be counteracted by windy pharases and sentimental appeals. Equally £lever steps should be taken. The consciousness of the masses must be awakened; that is the only way of keeping them steady in the fight. 

Fellow Countrymen, a few words about Hindu-Moslem unity, which has been given such a prominent place in the Congress programme. The people of India are divided by vertical lines, into innumerable sects, religions, creeds and castes. To seek to cement these cleavages by artificial and 
sentimental propaganda is a hopeless task. But fortunately, and perhaps to the great discomfiture of some orthodox patriots, who believe that India is a special creation of Providence, there is one mighty force that spontaneously divides all these innumerable sections horizontally into two homogeneous parts. This is the economic force : the exploitation of the disinherited by the propertied class. This force is in operation in India, and is effecting the innumerable vertical lines of social cleavage, while divorcing the two great classes further apart. The inexorable working of this force is drawing the Hindu workers and peasants closer and closer to their Moslem comrades. This is the only agency of Hindu-Moslem unity. Whoever will be bold enough to depend on the ruthless march of this force of social-economics, will not have to search frantically for pleas by which the Musulman can be induced to respect the cow, nor to make the ignorant Hindu peasants believe that the salvation of his soul and the end of his earthly misery lies in the redemption of the Khilafat or the subjugation of the Armenians by the Turks. Hindu-Moslem unity is not to be cemented by sentimentality; it is being realised practically by the development of economic forces. Let us concentrate and depend on the objective. 

Fellow Countrymen, let the Congress reflect the needs of the nation and not the ambition of a small class. Let the Congress cease to engage in political gambling and vibrate in response to the social forces developing in the country. Let it prove by deeds that it wants to end foreign exploitation not to secure the monopoly to the native propertied class, hut to liberate the Indian people from all exploitation, political, 
economic and social. Let it show that it really represents the people and can lead them in their struggle in every stage of it. Then the Congress will secure the leadership of the nation, and Swaraj will be won not on a particular day; elected according to the caprice of some individuals, but by the conscious and concerted action of the masses. 

\begin{flushright}
Manabendra Nath Roy\\
Abony Mukherjee
\end{flushright}


