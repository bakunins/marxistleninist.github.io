


\textbf{For The Second Congress of The Communist International}

In submitting for discussion by the Second Congress of the Communist International the following draft theses on the national and the colonial questions I would request all comrades, especially those who possess concrete information on any of these very complex problems, to let me have their opinions, amendments, addenda and concrete remarks in the most concise form (no more than two or three pages), particularly on the following points:\\
\indent Austrian experience; \\
\indent Polish-Jewish and Ukrainian experience; \\
\indent Alsace-Lorraine and Belgium ; \\
\indent Ireland; \\
\indent Danish-German, Italo-French and Italo-Slav relations; \\
\indent Balkan experience; \\
\indent Eastern peoples; \\
\indent The struggle against Pan-Islamism; \\
\indent Relations in the Caucasus; \\
\indent The Bashkir and Tatar Republics; \\
\indent Kirghizia; \\
\indent Turkestan, its experience; \\
\indent Negroes in America; \\
\indent Colonies; \\
\indent China-Korea-Japan. \\

June 5, 1920 
\begin{flushright}
N. Lenin 
\end{flushright}


1) An abstract or formal posing of the problem of equality in general and national equality in particular is in the very nature of bourgeois democracy. Under the guise of the equality of the individual in general, bourgeois democracy proclaims the formal or legal equality of the property-owner and the proletarian, the exploiter and the exploited, thereby grossly deceiving the oppressed classes. On the plea that all men are absolutely equal, the bourgeoisie is transforming the idea of equality, which is itself a reflection of relations in commodity production, into a weapon in its struggle against the abolition of classes. The real meaning of the demand for equality consists in its being a demand for the abolition of classes. 

2) In conformity with its fundamental task of combating bourgeois democracy and exposing its falseness and hypocrisy, the Communist Party, as the avowed champion of the proletarian struggle to overthrow the bourgeois yoke, must base its policy, in the national question too, not on abstract and formal principles but, first, on a precise appraisal of the specific historical situation and, primarily, of economic conditions; second, on a clear distinction between the interests of the oppressed classes, of working and exploited people, and the general concept of national interests as a whole, which implies the interests of the ruling class; third, on an equally clear distinction between the oppressed dependent and subject nations and the oppressing, exploiting and sovereign nations, in order to counter the bourgeois-democratic lies that play down this colonial and financial enslavement of the vast majority of the world’s population by an insignificant minority of the richest and advanced capitalist countries, a feature characteristic of the era of finance capital and imperialism. 

3) The imperialist war of 1914-18 has very clearly revealed to all nations and to the oppressed classes of the whole world the falseness of bourgeois-democratic phrases, by practically demonstrating that the Treaty of Versailles of the celebrated “Western democracies” is an even more brutal and foul act of violence against weak nations than was the Treatly of Brest-Litovsk of the German Junkers and the Kaiser. The League of Nations and the entire post-war policy of the Entente reveal this truth with even greater clarity and distinctness. They are everywhere intensifying the revolutionary struggle both of the proletariat in the advanced countries and of the toiling masses in the colonial and dependent countries. They are hastening the collapse of the petty-bourgeois nationalist illusions that nations can live together in peace and equality under capitalism. 

4) From these fundamental premises it follows that the Communist International’s entire policy on the national and the colonial questions should rest primarily on a closer union of the ''roletarians and the working masses of all nations and countries for a joint revolutionary struggle to overthrow the landowners and the bourgeoisie. This union alone will guarantee victory over capitalism, without which the abolition of national oppression and inequality is impossible. 

5) The world political situation has now placed the dictatorship of the proletariat on the order of the day. World political developments are of necessity concentrated on a single focus — the struggle of the world bourgeoisie against the Soviet Russian Republic, around which are inevitably grouped, on the one hand, the Soviet movements of the advanced workers in all countries, and, on the other, all the national liberation movements in the colonies and among the oppressed nationalities, who are learning from bitter experience that their only salvation lies in the Soviet system’s victory over world imperialism. 

6) Consequently, one cannot at present confine oneself to a bare recognition or proclamation of the need for closer union between the working people of the various nations; a policy must be pursued that will achieve the closest alliance, with Soviet Russia, of all the national and colonial liberation movements. The form of this alliance should be determined by the degree of development of the communist movement in the proletariat of each country, or of the bourgeois-democratic liberation movement of the workers and peasants in backward countries or among backward nationalities. 

7) Federation is a transitional form to the complete unity of the working people of different nations. The feasibility of federation has already been demonstrated in practice both by the relations between the R.S.F.S.R. and other Soviet Republics (the Hungarian, Finnish and Latvian in the past, and the Azerbaijan and Ukrainian at present), and by the relations within the R.S.F.S.R. in respect of nationalities which formerly enjoyed neither statehood nor autonomy (e.g., the Bashkir and Tatar autonomous republics in the R.S.F.S.R., founded in 1919 and 1920 respectively). 

8) In this respect, it is the task of the Communist International to further develop and also to study and test by experience these new federations, which are arising on the basis of the Soviet system and the Soviet movement. In recognising that federation is a transitional form to complete unity, it is necessary to strive for ever closer federal unity, bearing in mind, first, that the Soviet republics, surrounded 
as they are by the imperialist powers of the whole world — which from the military standpoint are immeasurably stronger — cannot possibly continue to exist without the closest alliance; second, that a close economic alliance between the Soviet republics is necessary, otherwise the productive forces which have been ruined by imperialism cannot be restored and the well-being of the working people cannot be ensured; third, that there is a tendency towards the creation of a single world economy, regulated by the proletariat of all nations as an integral whole and according to a common plan. This tendency has already revealed itself quite clearly under capitalism and is bound to be further developed and consummated under socialism. 

9) The Communist International's national policy in the sphere of relations within the state cannot be restricted to the bare, formal, purely declaratory and actually non-committal recognition of the equality of nations to which the bourgeois democrats confine themselves — both those 
who frankly admit being such, and those who assume the name of socialists (such as the socialists of the Second International). 

In all their propaganda and agitation — both within parliament and outside it — the Communist parties must consistently expose that constant violation of the equality of nations and of the guaranteed rights of national minorities which is to be seen in all capitalist countries, despite their "democratic” constitutions. It is also necessary, first, constantly to explain that only the Soviet system is capable of ensuring genuine equality of nations, by uniting first the proletarians and then the whole mass of the working population in the struggle against the bourgeoisie; and, second, that all Communist parties should render direct aid to the revolutionary movements among the dependent and underprivileged nations (for example, Ireland, the American Negroes, etc.) and in the colonies. 

Without the latter condition, which is particularly important, the struggle against the oppression of dependent nations and colonies, as well as recognition of their right to secede, are but a false signboard, as is evidenced by the parties of the Second International. 

10) Recognition of internationalism in word, and its replacement in deed by petty-bourgeois nationalism and pacifism, in all propaganda, agitation and practical work, is very common, not only among the parties of the Second International, but also among those which have withdrawn from it, and often even among parties which now call themselves communist. The urgency of the struggle against this evil, against the most deep-rooted petty-bourgeois national prejudices, looms ever larger with the mounting exigency of the task of converting the dictatorship of the proletariat from a national dictatorship (i.e. existing in a single country and incapable of determining world politics) into an international one (i.e. a dictatorship of the proletariat involving at least several advanced countries, and capable 
of exercising a decisive influence upon world politics as a whole). Petty-bourgeois nationalism proclaims as internationalism the mere recognition of the equality of nations, and nothing more. Quite apart from the fact that this recognition is purely verbal, petty-bourgeois nationalism preserves national self-interest intact, whereas proletarian internationalism demands, first, that the interests of the proletarian struggle in any one country should be subordinated to the interests of that struggle on a world-wide scale, and second, that a nation which is achieving victory over the bourgeoisie should be able and willing to make the greatest national sacrifices for the overthrow of international 
capital. 

Thus, in countries that are already fully capitalist and have workers’ parties that really act as the vanguard of the proletariat, the struggle against opportunist and petty-bourgeois pacifist distortions of the concept and policy of internationalism is a primary and cardinal task. 

11) With regard to the more backward states and nations, in which feudal or patriarchal and patriarchal-peasant relations predominate, it is particularly important to bear in mind: 

first, that all Communist parties must assist the bourgeois-democratic liberation movement in these countries, and that the duty of rendering the most active assistance rests primarily with the workers of the country the backward nation is colonially or financially dependent on; 

second, the need for a struggle against the clergy and other influential reactionary and medieval elements in backward countries; 

third, the need to combat Pan-Islamism and similar trends, which strive to combine the liberation movement against European and American imperialism with an attempt to strengthen the positions of the khans, landowners, mullahs, etc.;* 
\footnote{In the proofs Lenin inserted a brace opposite points 2 and 3 and wrote “2 and 3 to be united”. — Ed. }

fourth, the need, in backward countries, to give special support to the peasant movement against the landowners, against landed proprietorship, and against all manifestations or survivals of feudalism, and to strive to lend the peasant movement the mo.st revolutionary character by establishing the closest possible alliance between the West-European communist proletariat and the revolutionary peasant movement in the East, in the colonies, and in the backward countries generally. It is particularly necessary to exert every effort to apply the basic principles of the Soviet system in countries where pre-capitalist relations predominate — by setting up “working people’s Soviets”, etc.; 

fifth, the need for a determined struggle against attempts to give a communist colouring to bourgeois-democratic liberation trends in the backward countries; the Communist International should support bourgeois-democratic national movements in colonial and backward countries only on condition that, in these countries, the elements of future proletarian parties, which will be communist not only in name, are brought together and trained to understand their special tasks, i.e., those of the struggle against the bourgeois-democratic movements within their own nations. The Communist International must enter into a temporary alliance with bourgeois democracy in the colonial and backward countries, but should not merge with it, and should under all circumstances uphold the independence of the proletarian movement even if it is in its most embryonic form; 

sixth, the need constantly to explain and expose among the broadest working masses of all countries, and particularly of the backward countries, the deception systematically practised by the imperialist powers, which, under the guise of politically independent states, set up states that are wholly dependent upon them economically, financially and militarily. Under present-day international conditions there is no salvation for dependent and weak nations except in a union of Soviet republics. 

12) The age-old oppression of colonial and Weak nationalities by the imperialist powers has not only filled the working masses of the oppressed countries with animosity towards the oppressor nations, but has also aroused distrust in these nations in general, even in their proletariat. 
The despicable betrayal of socialism by the majority of 'he official leaders of this proletariat in 1914-19, when “ ‘defence of country ” was used as a social-chauvinist cloak to conceal the defence of the “right” of their “own” bourgeoisie to oppress colonies and fleece financially 
dependent countries, was certain to enhance this perfectly legitimate distrust. On the other hand, the more backward the country, the stronger is the hold of small-scale agricultural production, patriarchy and isolation, which inevitably lend particular strength and tenacity to the deepest of petty-bourgeois prejudices, i.e., to national egoism and national narrow-mindedness. These prejudices are bound to die out very slowly, for they can disappear only after imperialism and capitalism have disappeared in the advanced countries, and after the entire foundation of the backward countries’ economic life has radically changed. It is therefore the duty of the class-conscious communist proletariat of all countries to regard with particular caution and attention the survivals of national sentiments in the countries and among nationalities which have been oppressed the longest; it is equally necessary to make certain concessions with a view to more rapidly overcoming this distrust and these prejudices. Complete victory over capitalism cannot be won unless the proletariat and, following it, the mass of working people in all countries and nations throughout the world voluntarily strive for alliance and unity. 

\begin{flushright}
Published in June 1920, in Lenin’s Collected \\
Works, Volume No. 31 according to the text of the \\
proof-sheet, as amended by V.I. Lenin \\
\end{flushright}

\textbf{3. Report of the Commission on the National and The Colonial Questions, July 26}
\footnote{The commission on the national and the colonial questions, formed by the Second Congress of the Communist International, included representatives of the Communist parties of Russia, Bulgaria, France, Holland, Germany, Hungary, the U.S.A., India, Persia, China, Korea and Britain. The work of the commission was guided by Lenin, whose theses on the national and the colonial questions were discussed at the fourth and fifth sessions of -he Congress, and were adopted on July 28. }

Comrades, I shall confine myself to a brief introduction, after which Comrade Maring, who has been secretary to our com- mission, will give you a detailed account of the changes we have made in the theses. He will be followed by Comrade Roy, who has formulated the supplementary theses. Our commission have unanimously adopted both the preliminary theses, as amended, and the supplementary theses. We have thus reached complete unanimity on all major issues. I shall now make a few brief remarks.

First, what is the cardinal idea underlying our theses? It is the distinction between oppressed and oppressor nations. Unlike the Second International and bourgeois democracy, we emphasise this distinction. In this age of imperialism, it is particularly important for the proletariat and the Communist International to establish the concrete economic facts and to proceed from concrete realities, not from abstract postulates, in all colonial and national problems. 

The characteristic feature of imperialism consists in the whole world, as we now see, being divided into a large number of oppressed nations and an insignificant number of oppressor nations, the latter possessing colossal wealth and powerful armed forces. The vast majority of the world’s population, over a thousand million, perhaps even 1,250 million people, if we take the total population of the world as 1,750 million, 
in other words, about 70 per cent of the world’s population, belong to the oppressed nations, which are either in a state of direct colonial dependence or are semi-colonies, as, for example, Persia, Turkey and China, or else, conquered by some big imperialist power, have become greatly dependent on that power by virtue of peace treaties. This idea of distinction, of dividing the nations into oppressor and op- 
pressed, runs through the theses, not only the first theses published earlier over my signature, but also those submitted by Comrade Roy. The latter were framed chiefly from the standpoint of the situation in India and other big Asian countries oppressed by Britain. Herein lies their great importance to us. 

The second basic idea in our theses is that, in the present world situation following the imperialist war, reciprocal relations between peoples and the world political system as a whole are determined by the struggle waged by a small group of imperialist nations against the Soviet movement and the Soviet states headed by Soviet Russia. Unless We hear that in mind, we shall not be able to pose a single national or colonial problem correctly, even if it concerns a most outlying part of the world. The Communist parties, in civilised and backward countries alike, can pose and solve political problems correctly only if they make this postulate their starting-point. 

Third, I should like especially to emphasise the question of the bourgeois-democratic movement in backward countries. This is a question that has given rise to certain differences. We have discussed whether it would be right or wrong, in principle and in theory, to state that the Communist International and the Communist parties must support the bourgeois-democratic movement in backward countries. As a result of our discussion, we have arrived at the unanimous decision to speak of the national-revolutionary movement rather than of the “bourgeois-democratic” movement. It is beyond doubt that any national movement can only be a bourgeios-democratic movement, since the overwhelming mass of the population in the backward countries consists of peasants who represent bourgeois-capitalist relationships. It would be utopian to believe that proletarian parties in these backward countries, if indeed they can emerge in them, can pursue communist tactics and a communist policy, without establishing definite relations with the peasant movement and without giving it effective support. However, the objections have been raised that, if we speak of the bourgeois-democratic movement, we shall be obliterating all distinctions between the reformist 
and the revolutionary movements. Yet that distinction has been very clearly revealed of late in the backward and colonial countries, since the imperialist bourgeoisie is doing everything in its power to implant a reformist movement among the oppressed nations too. There has been a certain rapprochement between the bourgeoisie of the exploiting countries and that of the colonies, so that very often — perhaps even in most 
cases — the bourgeoisie, of the oppressed countries, while it does support the national movement, is in full accord with the imperialist bourgeoisie, i.e., joins forces with it against all revolutionary movement, and revolutionary classes. This was irrefutably proved in the commission, and we decided that the only correct attitude was to take this distinction into account and, in nearly all cases, substitute the term “national- revolutionary” for the term “bourgeois-democratic”. The significance of this change is that we, as Communists, should 
and will support bourgeois-liberation movements in the colonies only when they are genuinely revolutionary, and when their exponents do not hinder our work of educating and organising in a revolutionary spirit the peasantry and the masses of the exploited. If these conditions do not exist, the Communists in these countries must combat the reformist bourgeoisie, to whom the heroes of the Second International also belong. Reformist parties already exist in the colonial countries, and in some cases their spokesmen call themselves Social-Democrats and socialists. The distinction I have referred to has been made in all the theses with the result, I think, that our view is now formulated much more precisely. 

Next, I would like to make a remark on the subject of peasants’ Soviets. The Russian Communists’ practical activities in the former Tsarist colonies, in such backward countries as Turkestan, etc., have confronted us with the question of how to apply the communist tactics and policy 
in pre-capitalist conditions. The preponderance of pre-capitalist relationships is still the main determining feature in these countries, so that there can be no question of a purely proletarian movement in them. There is practically no industrial proletariat in these countries. Nevertheless, we have assumed, we must assume, the role of leader even there. Experience has shown us that tremendous difficulties have 
to be surmounted in these countries. However, the practical results of our work have also shown that despite these difficulties we are in a position to inspire in the masses an urge tor independent political thinking and independent political action, even where a proletariat is practically non-existent. This work has been more difficult for us than it will be for comrades in the West-European countries, because in Russia the proletariat is engrossed in the work of state administration. It will readily be understood that peasants' living in 
conditions of semi-feudal dependence can easily assimilate and give effect to the idea of Soviet organisation. It is also clear that the oppressed masses, those who are exploited, not only by merchant capital but also by the feudalist, and by a state based on feudalism, can apply this weapon, this type of organisation in their condition too. The idea of Soviet organisation is a simple one, and is applicable, not only to 
proletarian, but also to peasant feudal and semi-feudal relations. Our experience in this respect is not as yet very considerable. However, the debate in the commission, in which several representatives from colonial countries participated, demonstrated convincingly that the Communist International’s theses should point out that peasant’s Soviets, Soviets of the exploited, are a weapon which can be employed, not only in capitalist countries but also in countries with precapitalist relations, and that it is the absolute duty of Communist parties and of elements prepared to form Communist parties, everywhere to conduct propaganda in favour of peasants’ Soviets or of working people’s Soviets, this to include backward and colonial countries. Wherever conditions permit, they should at once make attempts to set up Soviets of the working people. 

This opens up a very interesting and very important field for our practical work. So far our joint experience in this respect has not been extensive, but more and more data will gradually accumulate. It is unquestionable that the proletariat of the advanced countries can and should give help to the working masses of the backward countries, and that the backward countries can emerge from their present stage of developments when the victorious proletariat of the Soviet Republics extends a helping hand to these masses and is in a position to give them support. 

There was quite a lively debate on this question in the commission, not only in connection with the theses I signed, but still more in connection with Comrade Roy’s theses, which he will defend here, and certain amendments to which were unanimously adopted. 

The question was posed as follows: are we to consider as correct the assertion that the capitalist stage of economic development is inevitable for backward nations now on the road to emancipation and among whom a certain advance towards progress is to be seen since the war? We replied in the negative. If the victorious revolutionary proletariat conducts systematic propaganda among them, and the Soviet governments come to their aid with all the means at their disposal — in that event it will be mistaken to assume that the backward peoples must inevitably go through the capitalist stage of development. Not only should we create independent contingents of fighters and party organisation in the colonies and the backward countries, not only at once launch propaganda for the organisation of peasants' Soviet and strive to adapt them to the pre-capitalist conditions, but the Communist International should advance the proposition, with the appropriate theoretical grounding, that with the aid of the proletariat of the advanced countries, backward countries can go over to the Soviet system and, through certain stages of development, to communism, without having to pass through the capitalist stage. 

The necessary means for this cannot be indicated in advance. These will be prompted by practical experience. It has, however, been definitely established that the idea of the Soviets is understood by the mass of the working people in even the most remote nations, that the Soviets should be adapted to the conditions of a pre-capitalist social system, and that the Communist parties should immediately begin work in 
this direction in all parts of the world. 

I would also like to emphasise the importance of revolutionary work by the Communist parties, not only in their own, but also in the colonial countries, and particularly among the troops employed by the exploiting nations to keep the colonial peoples in subjection. 

Comrade Quelch of the British Socialist Party spoke of this in our commission. He said that the rank-and-file British worker would consider it treasonable to help the enslaved nations in their uprising against British rule. True, the jingoist and chauvinist-minded labour aristocrats of Britain and America present a very great danger to socialism, and are a bulwark of the Second International. Here we are confronted with 
the greatest treachery on the part of leaders and workers belonging to this bourgeois International. The colonial question has been discussed in the Second International as well. The Basle Manifesto is quite clear on this point, too. The parties of the Second International have pledged themselves to revolutionary action, but they have given no sign of genuine revolutionary work or of assistance to the exploited and dependent nations in their revolt against the oppressor nations. This, I think, applies also to most of the parties that have withdrawn from the Second International and wish to join the Third International. We must proclaim this publicly for all to hear, and it is irrefutable. We shall see if any attempt is made to deny it. 

All these considerations have formed the basis of our resolutions, which undoubtedly are too length but will nevertheless, I am sure, prove of use and will promote the development and organisation of genuine revolutionary work in connection with the national and the colonial questions. And that is our principal task. 

(From V.I. Lenin’s Collected Works, Volume No. 31) 

