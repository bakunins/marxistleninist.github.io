
\begin{center}
\subsection{M N Roy's Version on the Formation OF Communist Party of India in Tashkent}
\footnote{Excerpt from ; Memoirs of M. N. Roy, 1964 edition. Pages : 459-467}
\end{center}

The destitute Indian religious emigres were brought to Tashkent. It was not easy to house, feed and clothe them there. Everything was scarce in the capital of the new-founded Turkestan Republic. The first difficulty was to find a house which could accommodate more than fifty people. Having received the news of the coming of the Indian "comrades," the Turk-Bureau of the Communist International had found a suitable house 
with the help of the Turkestan Government. In the modern part of the city, there was no large enough house. The house for the Indian emigres was between the modern part and the old part of the city. It was a one-storey building with a large number of rooms of all shapes and sizes. It was still winter. Already at Bokhara the Indian emigres had been provided with warm clothing. Each was given a uniform of the Red Army 
soldier. The long brown coat was quite warm. Yet, some arrangement had to be made for heating the house, at least for another month or so. Fuel was the scarcest commodity of all, and to'heat a big house required a considerable quantity. To obtain that, a special permit of the Government had to be secured, and no lesser official than the Commissar of Supplies could issue the permit. The now famous Lazar Kaganovitch 
was the official to be approached. I thought of seeing him personally so as to make it sure that the permit for the precious commodity would be obtained soon. But my secretary interpreter — a Russian jew who had lived long in America— felt that it would not be quite proper. In the official hierarchy I held a position higher up than the Commissar of Supplies. The Secretary said that a letter from me would be enough, and he 
would go with it. I asked him to draft the letter in Russian, and he produced a tearful document describing all the suffering of the Indian comrades, and how they would freeze if the house was not heated properly. I asked if it was necessary to write all that. He replied, "Yes, some agitation is necessary to move the Commissar." 

My Secretary went off with the letter as drafted by him, and within a couple of hours returned triumphantly with the permit, not only for fuel, but also for a pair of "valniki" (thick high felt boots worn in Russia during the winter) for each Indian comrade. He also brought a message from the Commissar, that the Indian comrades would be provided with everything necessary to make them comfortable, and that I would only have to ring him up, and the required commodities would be delivered promptly. 

Before long, two truckloads of wood arrived, and several tall white porcelain tile stoves in the house were lit. There followed a scramble for the warmest room. Anticipating that many similar difficulties would follow, I suggested that three from amongst the emigres should be chosen to constitute a House Committee to supervise the allotment of rooms, and also the provision of food for them all. There were several 
fairly educated young men in the crowd. I selected three of them and recommended them for the Committee. They readily agreed with my suggestion that, in the allotment of rooms, preference should be given to those not in good health and also to the aged. There were actually several gray beards who had joined the crusade for the defence of the Khilafat. It was a miracle that they had survived the hardships of their journey 
over the snow-peaked Hindukush. It was also agreed that cots should be provided only to the aged and those in indifferent health. It was impossible to secure cots for all. anyhow. But large woollen carpets were secured to cover the floors, and on the whole the guests were fairly comfortable. 

Feeding arrangements were also quite satisfactory-one meal consisted of what was called pilaf, no worse than eaten in my residence or in that of any other high party or State official. The evening meal was composed of lepioshka (thick but well-baked unleavened bread known in Northern India as nan) and some meat preparation, usually shashlik (Shik Kabab). In between, in the morning and in the afternoon, apple tea with raisins (in lieu of sugar) was also supplied. It was not a very luxurious board, as expected from oriental hospitality. But none in those days got anything better, and most of the guests were not accustomed to anything better at home. Nevertheless, before long, there was grumbling, and articulate complaints. Except for the minority of the educated, the rest of the lot felt that they were obliging us by accepting our 
hospitality. The House Committee spared me the trouble of listening to unreasonable complaints and trying to do the impossible. As a matter of fact, about a dozen young men of the company proved to be very helpful. 

Presently, the time came when we had to broach the question of the purpose of bringing the Indian emigrants to Tashkent. At Bokhara most of them were attracted by the hope of receiving military training. They believed that after a short sojourn in Tashkent, they would all be sent home with plenty of arms and money to fight the British. But we had to decide the nature of training according to the competence of the recruits. It would be easy enough to teach all of them how to shoot a gun or even to handle some more complicated weapons. It would also not be difficult to satisfy their expectation of going back to India with arms. But the question was: what would most of them do with their guns, and whom would they fight, and for what ideal? 

They had left India with the purpose of fighting for the Khilafat. Most of them were not even nationalists. They were anti-British, but had no idea of what would happen when the British were driven out of India. So it was felt that a measure of elementary political training was necessary before the majority of the emigrants could be armed. The plan was not to convert them to Communism, but to awaken in them the minimum measure of political consciousness. They might be easily persuaded to abandon the slogan "Khilafat Zindabad" for the slogan "Inquilab Zindabad." But they should have some idea of revolution, and how it would be brought about. And for that purpose, first of all they should become loyal soldiers of the revolution. Then we were thinking in terms of a national democratic revolution But if most of the emigrants were not nationalists, none of them had any idea of democracy. 

I discussed those difficulties with the educated minority of the group. After some persuasion, they agreed with my proposal that the emigrants should receive a course of political training before a military school for training the soldiers of the Indian Revolution could be founded. Arms were easily available. As a matter of fact, I had come to Tashkent with a large quantity of them. But there was no possibility of 
sending them farther on nearer to India. We did not want the Indian emigrants to be Communists. But if they were to be armed, we wanted to be sure that the arms would not be turned against us. 

It was quite clear that, before we could proceed to do anything fruitful with the emigrants, the educated minority should be differentiated from the fanatical mass. So, to begin with, I set myself to the task of politically educating the educated few. Most of them responded quite satisfactorily, although a few turned out to be very refractory. I was very much surprised to find that a few of the educated young men were more fanatical than the emigrant mass. Curiouly enough, one of them eventually became an equally fanatical Communist. He is still living somewhere in Pakistan, although it is reported that he has left, or has been expelled from, the Communist Party. He was a pathological case-distrustful of everything and fanatically religious. Another, a somewhat more elderly person who claimed to have been closely associated with Mohammad Ali, was a more deliberate trouble-maker. On the strength, of his supposed association with the leader of 4he Indian Khilafat 
Movement, he commanded the confidence of the group and could sway them as he liked. He is also alive, and it is reported that he holds a high position in Pakistan. 

My closer relation with the educated few aroused suspicion and resentment among the rest. The latter of the two mentioned before took advantage of that situation, and carried on a whispering campaign that I was trying to convert them to Communism. An educated few among them had actually been so converted. There was a terrible uproar, but the trouble-maker was a very skilful intriguer, and could not be detected easily. The ten.se situation came to a head when one day the majority of the group demanded that the meat given to them should be from sheep slaughtered in their presence. They suspected that they might be given meat which Mussalmans were not allowed to eat. It was a groundless suspicion, because in those days no pigs were easily available in Turkestan. But the excited fanatics would not listen. When the House Committee, assisted by the other educated few, tried to explain the position, they were denounced as Kafirs who would be driven out of the house if they continued 
their objectionable activity. It was an embarrassing position. I would not have minded throwing the majority of the lot out into the streets because no good was ever likely to come out of them anyway. But the Russians were very sensitive about the "Indian comrades" and advised me to be patient and persistent. On my replying that my patience was exhausted by the unruly fanatics and that I did not believe that further perseverance would produce any result, the Russian comrades suggested that the Presi-dent of the Turkestan Republic, who was a Muslim, should visit the boarding house and talk to the Indian comrades. The President himself was a young man, who wore European clothes and kept no beard. 
However, his visit did make an impression; after all, he was the head of the government! 

After the visit, he sympathised with my difficulties and offered to help. He was also of the opinion that nothing could be done with most of the Indian emigrants, and suggested that we should select only the hopeful few and tell the rest to go out and earn a living, on the ground that in a Communist State none could have bread unless he worked. He offered to provide employment for those who would be willing to work. Of course, his advice could not be accepted, and it had to be kept a strictly guarded secret. 

Meanwhile, I went ahead with the political training of the educated few, barring the couple of mischievous ones. To pacify them, the ignorant majority was allowed to go out and roam in the bazar. That was a risky procedure, because even Tashkent was full of enemy spies in those days. They could take advantage of the disaffection of the Indian emigrants, and with a little money purchase their services. Before long, it was discovered that the apprehension was not unfounded, and I was placed in the delicate situation of dealing with suspected enemy agents as they were dealt with in those days in Russia. My position was delicate because the Russians would not do anything against the "Indian comrades." 

My preliminary efforts with the educated minority produced greater results than I expected and wanted. Most of them transferred their fanatical allegiance from Islam to Communism. I had not spoken to them at all about Communism. I only told them that driving the British out of India would be no revolution, if it was succeeded by replacing foreign exploiters by native ones. I had to explain the social significance of a revolution: that, to be worthwhile, a revolution should liberate the toiling masses of India from their present economic position. Instinctively idealists, they readily agreed with my opinion and jumped to the conclusion that, if the revolution was to liberate the toiling masses, it would have to be a Communist revolution. I was surprised when some of them approached me with the proposal that they wanted to 
join the Communist Party. Others enquired why we should not found the Communist Party of India there and then. Their enthusiasm was very well meant. Although some of them had a utilitarian motive, I could not discourage them. 

Presently, they were reinforced by the arrival of a small group which called itself Communists already at Kabul. It was led by an old gray-bearded Maulana, Abdur Rub, and a South Indian Hindu named Acharya. On their arrival, they were accommodated in the emigrants' house and expected to have special attention and privileges owing to their professed political faith. I would have welcomed the advent 
of even a few clever and convinced Communists to help me deal with a rather difficult situation. But after some conversation I discovered that Abdur Rub was an impostor, and Acharya was an anarchist, if he was anything. But the educated minority of the earlier emigrants were easily 
influenced by Abdur Rub and Acharya, who fanned their Communist fanaticism. 

The result of a new crisis in the emigrants' house was that some of the inmates began talking about communism openly and went to the extent of making disparaging remarks about their fanatical past, which was still present with most of the other. Occasionally, it came to fierce altercations and even exchange of blows. To maintain order and to protect the minority, we had to post some armed guards near the house. On the other hand, the minority, which proposed the formation of an Indian Communist Party, was reinforced by the Abdur Rab-Acharya group and, on the latter's instigation, sent a delegation to the Turk-Bureau of the Communist International to plead their case. I tried to argue with them that there was no hurry. They should wait until they returned to India. There was no sense in a few emigrant individuals calling themselves the Communist Party. They were evidently disappointed, and I apprehended that the experience might dishearten them. I needed their help to manage the refractory majority of the emigrants. The idea of turning them out with the offer of employment was not practical. So I agreed with the proposal of the formation of a Communist Party, knowing fully well that it would be a nominal thing, although it could function as the nucleus of a real Communist Party to be organised eventually. An intelligent and fairly educated young man named Mohammad Safiq, who had come from Kabul with the Acharya group, was elected secretary of the party .(*) 

\footnote{* M. N. Roy did not give further details about formation of Communist 
Party of India in Tashkent. }


The party was formed. But what should be its activity? A Communist Party must work among the masses. India was far away. The Indian masses could not be reached from where we were. But I pointed out that in the emigrant group we had a cross-section of the Indian masses. On ruturn to India, the pioneering Communists would have to face the political backwardness, general ignorance and religious fanaticism of the masses. So they had better serve their apprenticeship by endeavouring to influence the cross-section of the Indian masses within our reach. They agreed and it was decided that they should try to persuade the rest of the emigrants to attend a series of general political talks preparatory to their admission to the proposed military school. The members of the communist group were to deliver those talks. Apart from Mohammad Safiq, two others proved to be quite efficient. One was Shaukat Usmani. He was a graduate of some Indian University and quite intelligent. But he was the most fanatical of all and stuck to his guns till the bitter end. As he was known among the emigrants to be a devout Mussulman, his talks were readily attended and began to have influence. The other was Abdullah Safder. In India he used to teach Urdu to British army officers on the Frontier, but he himself hardly new any English. But being a professional teacher, he was also quite successful in his talks. Usmani returned to India several years later and became an important member of the Communist Party. He was an accused in the Meerut Conspiracy Case. Safder went to Moscow and graduated from the Communist University for the Toilers of the East. Then he received higher education in Marxist Theory at the Institute of Red Professors. He also came hack to India eventually and lived underground. In the earlier part of the Second World War he left India with the object of returning to Russia. I have had no news of what happened to him thereafter. 

The central theme of the talks delivered in the emigrants' house was that they must soon return to India to fight for her liberation. But a few of them, even if supplied with arms, would not be able to make a revolution. For that purpose, they must win over thousands and thousands of others like them- selves. Therefore, before returning to India they must learn what they would tell them to win them over for the revolution. 
Once that preliminary training was imparted, they would be admitted to the Military Schools to receive training in the use of all sorts of arms and when, on return to India, they would have enlisted the support of many others for the revolution, plenty of arms would be sent to them. 

The talks, delivered by several of their own fellow-religionists, did not mention the word "Communism" nor made any disrespectful reference to religion, which pacified the recalcitrant lot, and the atmosphere in the India House (that was how the emigrants' house was called) became quiet. The idea of being trained in the use of arms seemed to have been attractive. 

Shortly thereafter, a Military School was founded. The group of Russian officers who had accompanied me from Moscow was still in Tashkent. To them was entrusted the organisation of the school. John, the American Wobbly, was appointed the Commandant of the School. He was to look after 
the discipline. Having looked over his wards, he sarcastically remarked: "We are going to train not an army of revolution, but an army of God." 

\subsection{Formation of Communist Party of India in 1920}
\footnote{*Excerpts from Muzaffar Ahmad's Memoirs : “Myself and the Communist Party of India”, pages : 45-56}
The foregoing extract is from M. N. Roy’s original English text. The extract shows that the muhajir (self-exiled) youths who had emigrated to Tashkent in 1920 compelled M. N. Roy to lay there the foundation of the Communist Party of India. This idea created by his Memoirs would have remained firmly planted in our minds, had not new facts come to light meanwhile. Most of the muhajirs who became members of the Communist Party of India in Tashkent and in Moscow are dead. Of those who are still alive, we may, if we try, get in touch with only Comrade Rafiq Ahmad of Bhopal. He also is seventy (1967). Others who are still living live in Pakistan. To us Pakistan today is not only an alien country but a remote one too.

I will never say that M. N. Roy was not asked some time or other by the muhajir youths in Tashkent to form the Communist Party of India. But we now possess evidence that far from doing what he claims in his writing to have done in founding the Party in Tashkent, Roy actually did the 
reverse. It was M. N. Roy himself, not the muhajir youths, who took the real initiative in founding the Communist Party of India in Tashkent. \\

\textbf{Dr. Devendra Kaushik’s Discovery in Tashkent}\\

Shri Devendra Kaushik, M. A., Ph. D., is a Reader in History in the University of Kurukshetra. After obtaining a doctorate in History from the University of Agra, he became a Ph.D. also of the Lenin University of Uzbekistan in the Soviet Union for his researches on The Leninist Nationalist Policy in Central Asia. He had to stay in Tashkent for three years for the purpose. He has told me that it was in Tashkent that he first read my book The Communist Party of India and Its Formation Abroad and felt a desire to see the places where the muhajirs lived. He also enquired whether any documents of the period of the muhajirs’ stay in Taskent were available or not. As a result of his investigations, he found a file (F. 60, ed. No. 724, L. 1-14) in the Archives of the Communist Party of Uzebekistan. In fact, the Keeper of the Archives himself had searched out this file for Dr Kaushik. Dr Kaushik discovered some valuable documents relating to the Communist Party of India in the file. Dr Kaushik, I believe, could little expect that he would come by several documents of such value. The file under reference contained the following documents: 
(i) very brief minutes of the meeting in Tashkent in which the Indian Communist Party was first formed; 

(ii) brief minutes of a subsequent meeting of the newly-formed Indian Communist Party; 

(iii) a letter to the Communist Party of Turkestan communicating the news of the formation of the Indian Communist Party in Tashkent. 

[In 1920 this Party was called the Communist Party of Turkestan. Turkestan had not yet been reconstituted into a number of republics under different names.] 

I am reproducing below the text of all three documents found by Dr Kaushik: 

\begin{center}
    \textbf{(1)}
\end{center} 

“Formed the Indian Communist Party in Tashkent on Oct. 17, 1920, with the following members 

1. M. N. Roy 

2. Evelina Trent Roy 

3. A. Mukherjee 

4. Rosa Fitingof 

5. Mohd Ali (Ahmed Hasan) 

6. Mohd Shafiq Siddiqui and 

7. Acharya, M. Prativadi Bayankar 

The period for probation for candidate members would be three months. 

Mohd Shafiq is elected Secretary. 

The Indian Communist Party adopts the principles proclaimed 
by the Third International and undertakes to work out a 
programme adopted to the Indian condition. 

\indent (Sd.) President : M. Acharya \\
\indent (Sd.) M. N. Roy, Secretary 

\begin{center}
\textbf{(2)}    
\end{center}


“Further, in the Party Archives, Tashkent, are given the minutes of a subsequent meeting held on December 15, 1920. It reads as follows ; 

Resolved to admit Abdul Qadir Sehrai, Masud Ali Shah Kazi and Akbar Shah as candidate members. 

An Executive Committee of Roy, Shafiq and Acharya is elected. 

\begin{center}
\textbf{(3)}    
\end{center}


"Also preserved among the documents is a communication sent by Roy to the Central Committee of the Communist Party of Turkestan. It has been signed by Roy as the Secretary concerned. (The Russian term is ‘otvestvenny secretar’). "here is another signature also thereupon which is illegible. Roy signed in bold letters in Russian in red ink. It reads as under: 

“This is to state that the Communist Party of India has been organized here. It is working in conformity with the principles of the Third International under the political guidance of the Turkestan Bureau of the Comintern.” 

None of the seven members with whom the Communist Party of India was first formed in Tashkent on October 17, 1920, belonged to the muhajir youths who had emigrated from India to Turkestan that year; yet Manabendranath writes that it was because of their obstinacy that he proceeded to build the Communist Party. Evelina Trent Roy, one of the seven members, was Manabendranath ’s first and American wife. Rosa Fitingof was Abani Mukherjee’s Russian wife. Muhammad Ali had left India in 1915, while a student in the Medical College in Lahore, he had also declared himself 10 be a Communist much earlier. Mohammad Shafiq also was not one of the muhajir emigrants of 1920. He had left India in 1919, for political reasons. Acharya had left India in 1908. Narendranath Bhattacharyya alias Manabendranath Roy, and Abani Mukherjee had left India in 1915. 

At the time of writing his Memoirs, Roy could not even dream that the brief minutes relating to the formation of 
the Communist Party of India in Tashkent would not remain hidden in the Archives of the Communist Party of Uzbekistan and would one day be discovered by a person named Dr Devendra Kaushik. Now necessity knows no law, and M. N. Roy was compelled by necessity to take the initiative in founding the Communist Party of India in Tashkent. He attended the Second Congress of the Communist International as a delegate of the Communist Party of Mexico, but the affection and appreciation he received in Congress was due to his being an Indian. Even after the Congress he was put in charge of work in the Eastern countries, especially India. At that time there was not the slightest possibility of a 
Communist Party being formed in India in the near future, and Roy had left Mexico once for all. At various places Roy had written that he was his own representative in the Communist International and went to Moscow at the invitation of Lenin. All this claim of his is just empty self-glorification. It is true that Lenin invited Roy to Moscow on the basis of Borodin’s report, but Roy did not go there with empty hands. He went there on the advice of Borodin, his patron, as delegate of the Mexican Communist Party just formed. All this is recorded in Roy’s Memoirs. It was in Mexico that Borodin had impressed upon Roy that representatives of Communist Parties only (although the name Communist Party was not essential) could hold any office of high responsibility in the Communist International. With the Third Congress of the Communist International drawing near, Roy had to found the Communist Party of India in Tashkent; otherwise, what could have been his \textit{locus standi} there? Whom could he represent?

But what I fail to understand altogether is why among the seven founding members, there was not even a single muhajir emigrant of 1920. It was these muhajirs who were likely to become members of the Communist Party of India in Tashkent or in Moscow afterwards. It will not be an exaggeration to say that the presence of the muhajirs in Tashkent at the time was a windfall for Roy. Three of the muhajirs of 1920 were 
admitted as candidate members at a Party meeting on December 15, 1920; but two of them, Abdul Qadir Sehrai (Khan) and Masood Ali Shah, were questionable characters.\\

\textbf{Who were Acharya and Abdur Rab?}\\

It was in Tashkent that M. N. Roy’s relations with Maulavi Abdur Rab and M. Prativadi Bayankaf Acharya became strained. In his Memoirs Roy dismisses Maulavi Abdur Rab as an ‘imposter’ and Acharya as an ‘anarchist’. Some information about them should be given here. Maulavi Abdur Rab be longed probably to the district of Peshawar, for he is referred to also as. Abdur Rab Peshawari. As far as I have been able to gather, he was a linguist, a good speaker and a scholar. He was a high official of the British Government, probably in the diplomatic service. The Makran Gazetteer v/ as compiled from material gathered by Abdur Rab. Makran was situated in the Kalat section of Beluchistan; it was not an easily 
accessible place Once a man had been there, the terrible experience of the place would haunt his memory like a constant nightmare. In the introduction to the Gazetteer, it is written:

"In the present work an endeavour has been made to collate whatever published information is available and to supplement it with material gleaned from the country itself. For this purpose one of our Gazetteer assistants, Maulavi Abdur Rab, was deputed to Makran, where he spent 14 months in investigating actual conditions in Situ during 1903-4, and I am indebted to him for the local material included in this work." (Baluchistan District Gazetteer Series, vol. vii Makran, by R. Hughes-Buller, 1. C. S., Bombay, 1906, p. iv.) 

Mr R. Hughes-Buller, I. C. S., editor of the Gazetteer, has acknowledged his debt to Maulavi Abdur Rab for spending fourteen months in Makran during 1903-4 and collecting local material for the Makran Gazetteer.   

I do not know in which places and on what sort of assignments Maulavi Abdur Rab was employed in the following few years. It is found in Dr Bhupendranath Datta’s Aprakasita Rajnaitik Itihas (1953) (Unpublished Political History) that in 1914— at the beginning of World War I— he 
was a high official in the British Embassy at Baghdad. The British closed down the embassy and left Baghdad after the Turks had sided with Germany and against the British. But they left Mualavi Abdur Rab behind in the hope that he would take advantage of his being a Muslim to smuggle information to the British. But Abdur Rab did nothing of the sort : he sided straight with Turkey, and, consequently, with Germany. I believe that this was the first time Abdur Rab embroiled himself in anti-British politics. After the war he was seen in Angora, Afghanistan and in 1919 in the Soviet Union also. 

Now I shall tell something about Acharya of South India. His name is recorded as Mandayam Parthasarathi Tirumalai Acharya in the police report in our country. But he signed his name as M. Prativadi Bayankar Acharya in the Party meeting on October 17, 1920. Originally, the Acharyas had been inhabitants of Mysore, but for many years members of the family had been domiciled in Madras. I do not know whether Mr Acharya came in contact or not with any revolutionary party in his student days in India. He became involved in politics in 1908, when he went to London probably for studies. He became a fellow-worker of Virendranath Chattopadhyaya’s at that time. In France also, he and Chattopadhyaya both together came in contact with the Anarcho-Communist Party. He had toured Europe and America. As far as it can be surmised, the first time he made the acquaintance of Abdur Rab and became also his friend was during World War I. They were seen together in Afghanistan. They went to Russia in 1919, long before M. N. Roy visited the country. They also met Lenin in 1919. 

It is easy to see why M. N. Roy dubbed Acharya an ‘anarchist’. In 1920, during the Second Congress of the Communist International, Acharya might have himself told Roy that he had come into contact with the anarchists in Europe ; or Roy might have received this information from 
some other person. But Roy has not written a single word about why he has dismissed Abdur Rab as an impostor. Before giving him this bad name Roy should have given,- at least, some hints as to where and how he had practised imposture. I have heard that disgraced as he was in the Soviet Union, Acharya returned with a Russian wife to Germany. There he turned again towards his former anarchism and described himself then as an ‘Anarcho-Syndicalist’ and ‘a member of the Fourth International’ . Is it because of Acharya then that the name of the Fourth International is heard in Mysore and some parts of South India? Towards the end of his life, Acharya returned to India and lived in Bombay. I 
heard (in 1967) that he had died.

In Tashkent in 1920, and later possibly in Moscow also, the names of Acharya and Abdur Rab were spoken of together by the muhajirs. It appears that the two friends became estranged. Manabendranath had dismissed Acharya as an ‘anarchist’, but it is found that when he founded the Com- 
munist Party of India in Tashkent, he included Acharya, but left out Abdur Rab. This shows that Roy .succeeded, at least temporarily, in freeing Acharya of his ties with Abdur Rab. Roy's character is indeed worth studying. 

It is necessary to study the character of Abdur Rab also. What I have gathered from documents shows that Abdtir Rab was not a religious fanatic. He was also not a staunch Khilafatist. In my book \textit{The Communist Party of India and Its Formation} Abroad, I have given an account of the journey of Rafiq Ahmad of Bhopal, as told by him to me. It was not possible for me to take down all that he said; besides, the account in that case would have become much longer. Rafiq Ahmad told me a lot about Abdur Rab also at that time, but I have written very little, rather nothing, about all that. Moreover, there were many things about Abdur Rab which Rafiq Ahmad did not know. Recently (in 1967), Rafiq Ahmad wrote to me from Bhopal again about Abdur Rab. now I feel that it is necessary to give here .some extracts from the account given 
by Rafiq Ahmad about Abdur Rab. Rafiq Ahmad and his comrades reached Kabul on May 1, 1920, or thereabouts. Maulavi Abdur Rab and Acharya also were in Kabul at the time. Rafiq Ahmad writes : 

“We met Abdur Rab and Acharya of South India in a hall in the Russian Embassy. After having tea and cakes, we started talking. Maulavi Abdur Rab said that they had come from Russia to Kabul three or four months back. The situation in Kabul was that Badshah Amanullah wanted to have peace with the British at any price. Disappointed at this development, he was going back to Russia with his comrades. There were some Indians among these comrades. He said that there had been a revolution in Russia and the Revolutionary Government would give them help and facilities in all their endeavours. Abdur Rab also said that for those who wanted it the Soviet Government would even arrange the journey from Russia to Anatolia. Subsequent events showed that the Soviet Government had extended this facility regularly to the muhajirs; but the muhajirs had never been given permission to enter Anatolia. Abdur Rab wanted the names of those of us who wanted to go to Russia with him. He told us that 
he would secure necessary permission from the Badshah whom he was meeting the next day. A number of them, including Rafiq Ahmad, submitted their names to him on the spot. The next day when Muhammad Akbar Khan, Muhammad Akbar Shah, Sultan Muhammad and Gawhar Rahman met Abdur Rab, he informed them that Badshah Amanullah would not permit the muhajirs to go to Russia. He had probably met the Badshah meanwhile. Abdur Rab then advised the mujahirs to submit a joint petition to the Badshah, praying for permission to go to Russia. The muhajirs did accordingly. Abdur Rab said, “Take your own time and come. I shall wait for you in Tashkent.” Abdur Rab left for Tashkent and we went to Jablus Siraj. To some of the muhajirs he gave £3 per head so that they could get suits of clothes made for themselves.” 

This account shows clearly that Abdur Rab was not at all a staunch Khilafatist. Instead of encouraging the muhajirs to go to Anatolia, he encouraged them to go to Russia, the land of revolution. What views did he hold and what course of politics did he follow? He used to declare himself a Communist; then why did he not join the Communist Party of India? If he thought that the Communist Party of India formed in 
Tashkent was not worth its name, he could have joined it and built it into a real Communist Party. His following was not small. He had visited Russia long before Roy. He had met Lenin long before Roy came to Russia. What, however, I fail to understand at all is what effect his actions produced on Indian politics. 

The data that Dr. Devendra Kaushik collected from old local newspapers during his stay in Uzbekistan contain the information that a small party of Indian revolutionaries from Kabul arrived in Tashkent in March, 1919. \footnote{• Link (English weekly, Delhi) January 26, 1966, p. 72. } 
On the basis of the data collected by him. Dr Kaushik says that Abdur Rab was the leader of this party. But Col. F. M. Bailey, a British\ Intelligence officer, who was in Tashkent at that time, has written in his book Mission to Tashkent that Ba'rkatullah was the leader of the party. Dr. Kaushik has mentioned also this fact in his article. The First Congress of the Communist International was held from March 2 to March, 6, 1919. It is, therefore, evident that Barkatullah, Abdur Kab and others of the party would have then been guests not of the Communist 
International but of the Soviet Government. Abdur Rab was a very good speaker; besides, he could make speecnes in Turkish, which made him quite popular with the residents of Tashkent. It is learnt from Dr Kaushik’s article that the old residents of Tashkent still remember Abdur Rab. 

Considering the circumstances, we can easily realize that Abdur Rab had love of authority and leadership. He was, therefore disappointed to see Manabendranath Roy in Tashkent and realized at once that leadershif) had passed into Roy’s hands. The organizational theory that a revolutionary party was greater than one’s personal leadership did not appeal to Abdur Rab’s mind. He started building his own party with the 
muhajirs. Although he had declared himself a Communist, he had in his pocket another organization named ‘Association of the Indian Revolutionaries’. Dr Bhupendranath Datta has quite wrongly referred to this organization as ‘The Indian Nationalist Samiti’. Abdur Rab was in Moscow, organizing a separate party of his own with the muhajirs. Later some muhajir youths returned to Peshawar and made various state- 
ments to the police, divulging which of them had belonged to Abdur Rab’s party and which of them to Roy’s party, i.e. the Communist Party of India. Roy’s quarrel with Acharya of South India started during their stay in Tashkent. Roy had proposed that Acharya return to India, stay underground and organize the Communist Party of India. It was not possible for Acharya, who had left India in 1908 and joined the exiles 
abroad, to return to India and do this work. Acharya felt that Roy was making a proposal of this kind only to discredit him. Acharya made a counter propo.sal that Roy was not fit to remain in the Communist Party and that his name, therefore, should be struck off the Party rolls. The Communist Party of Turkestan tried to make up this quarrel but having failed, they advised both the parties to go to Moscow for a settlement. Moscow’s verdict went against Acharya. Acharya then rejoined Abdur Rab’s group. From the documents of the Moscow Conspiracy 
Case (1922-23), held in Peshawar, it is found that Abdur Rab was in the end working for the American Relief Mission. The Soviet Government and the Communist International did not approve of this action of his. Therefore, Abdur Rab too left the Soviet Union in disgrace and went to Germany. On August 30, 1967, Raja Mahendra Pratap wrote to me that “Maulavi Rab had a Turkish wife and he was, therefore, allowed to visit 
Stamboul at the request of some Turkiiih official. He must have died also in Turkey”. The Raja has heard only that Maulavi Rab is no longer alive. 



