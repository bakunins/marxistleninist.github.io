

It was in the early 1920s that the national revolutionaries of India who had come into existence before the First World War were converted into Communists. They were influenced by two factors which inter-acted with each other: 

Firstly, the experience of the struggle inside the National Movement, between the Moderate leaders and the Revolutionary ranks which came to the fore just before and during the First World War. 

Secondly, the Russian Proletarian Revolution of November 1917 which inspired the Indian revolutionaries as did the revolutionaries all over the world. Many of the Indian revolutionaries made almost a pilgrimage to the land of the first proletarian revolution in the world. Those who did not undertake such a difficult venture formed small communist groups in Bombay, Calcutta, Madras and the U.P.-Punjab region.\\ 

\subsection{CPI formed in Tashkent}

The Indian emigrate revolutionaries who had gone to several European countries, to the United States, Canada etc. were also influenced by the Russian Revolution. Some of them took the initiative in the Soviet City of Tashkent in organising what was called the Communist Party of India. Although formed outside India, this new organisation did a lot of work to educate (he young Indian communists at home. That organisation gave the first theoretical and practical education in Marxism-Leninism to the scattered groups of communists living and working in their homeland. The Tashkent Committee may therefore be considered the original foundation of what subsequently became the Communist Party of India. 

This however had a major drawback  although calling itself the Communist Party of India, it had its office outside the country. For this reason, a section of Indian communists did not consider the formation of the Tashkent group as the foundation of the Communist Party of India. They therefore decided to hold an open conference of Indian Communists in the City of Kanpur in U.P. ; a legally-working Communist Party of India was also formed in Kanpur. This, according to some, was the foundation of the real Communist Party of India, since it was formed and functioned inside the country. 

Neither the Tashkent organisation nor its Kanpur successor however could work legally due to intense repression resorted to by the British Rulers. The leading comrades of the Kanpur Communist Party of India, together with a large number of communist fellow travellers, were involved in the Meerut Conspiracy Case which lasted from early 1929 to the end of 1934. Only after the accused in the Meerut Conspiracy Case were released could a formally-constituted, though illegally functioning. Central Committee and Polit Bureau of the Communist Party of India started working. The Tashkent Committee and the Kanpur Committee were thus the fore-runners of the continuing leadership of the Communist Party of India. 

The work turned out by the Tashkent and the Kanpur Committees, as well as the countrywide organisation of the Workers' and peasants' Party of India(a broader set up than the Kanpur-formed Communist Party of India), were thus the organisational foundation in the 1930s for the All India Centre of the Communist Party of India formed in 1934.

It is thus over six decades since the continuing leadership of the Communist Party of India has been functioning. During the fairly long period, the Party has made enormous strides. Today, though divided into the CPI(M), the CPI and various other political groups considered Marxist-Leninist, the Communists are a force to reckon with in Indian politics. 

Three of the 25 States-Kerala, West Bengal and Tripura are having governments led by the CPI(M), the CPI and other Marxist groups like the RSP and the Forward Bloc. At the Centre too, a Government is in existence in which the liberal bourgeois and Marxist-Leninist groups are co-operating with each other with a view to keep the two major bourgeois formations, the Congress (1) and the BJP, out of power. 

Apart from the Centre, this combination is leading also the State Governments of Tamil Nadu, Andhra, Karnataka, Bihar, Assam and Jammu and Kashmir. Nine out of 25 States in India thus have governments in which Marxist-Leninists are participants. The general political influence of Indian Communists over the Central and State administration is therefore unmistakable. 

Although numerically the Marxist-Leninists are a small minority in the Indian political scene, its influence in Indian politics is far more than the electoral strength wielded by the Indian Marxist-Leninists. For. unlike other political groups, the Marxist-Leninists have made a distinct contribution to the theory and practice of Indian politics. It is attempted in this article to explain how this influence of Marxism-Leninism on 
the theory and practice of Indian politics has arisen.\\ 

\subsection{The great heritage of early ideological work}

As early as in the beginning of the 1920s, the leaders of the Tashkent group which called itself the Communist Party of India made big contributions to the study of the economy, polity, philosophy, etc. of India. In a series of articles written in the periodical journals brought out by the Tashkent group, current major political developments were subjected to Marxist-Leninist analysis. 

M. N. Roy, the former Indian national revolutionary who became a communist, wrote a series of articles in the journal of the Tashkent group, subjecting current political developments in India to Marxist-Leninist analysis. His contributions to the discussion of the communal problem and the analysis of the Mahatma Gandhi phenomenon were greatly rewarding 
to the Indian Communists. 

His analysis of the communal riots that were breaking out after the withdrawal of the Non-Cooperation-Khilafat Movement constituted a great contribution to the study of a major socio-political problem. His analysis contained in the book India in Transition was, in fact, the first beginning of Marxist-Leninist analysis of the Economy, Polity and Ideology of India under British Rule. 

Another foreign comrade who did the same work was Rajani Palme Dutt, the British Communist who wrote \textit{Notes of the Month} which were a month-by-month analysis of international developments as well as national political developments in India. His book under the title India Today was a sister publication to Roy’s India in Transition. The two together constituted text books applying Marxist-Leninist theory to the basic socio-economic, cultural and political problems of India. 

Such ideological work carried out by Roy and Dutt together with the day-to-day political and organisational work turned out by the comrades in India, gave an ideological basis on which 
the finally-organised and centralised Communist Party of India with its Central Committee and Polit Bureau came into being. 

This tradition of ideological work, applying the theory of Marxism-Leninism to Indian condition and enriching the theory of Marxism-Leninism with the experience of the political- 
organisational revolutionary work turned out by the Marxist-Leninists, was carried forward by Indian communists from 1934-35 (when a Central Committee and Polit Bureau came to be formed and were able to work continuously). This is the great heritage that we of the CPI(M), the CPI and other Marxist-Leninists proudly cherish and carry forward. 

It is proposed in the following paragraphs to explain what are the major contributions made by the Indian communists in this long period, stretching from the formation of the Tashkent group which called itself the Communist Party of India down to the present times. 

As early as in the beginning of the 1920s, M. N. Roy in his writings had drawn attention to the problems of communal riots, relating it to class struggle. He pointed out that the only antidote to communal division is class unity which means the bringing together of the working people belonging to all castes and communities in the struggle against imperialism and the rich belonging to all castes and communities. This principle is even relevant today, more than seven decades after Roy wrote his articles. Class unity through struggles against the oppressing classes is the only solution to the communal problem. 

As opposed to the Hindutva, Islamic Republic, Christian rule and so on, as well as to the rule of particular castes should be projected the democratic republic in which men and women born in every caste, believing in every religion and so on should be brought together in the struggle against the exploiting classes cutting across all caste and communal differences among the ruling classes. 

This was the basis on which, two decades later, the communists joined other secular forces including the Congress in opposing the two nation theory of Mohammed Ali Jinnah on 
which the demand for Pakistan was propagated. The Congress however projected, the communists pointed out, their idea of a single nation State of India ; they rejected the idea that India is inhabited by crores of people who are divided on the basis of language and culture and that the area inhabited by a single linguistic- cultural group is a \textit{nationality within the greater unity of the Indian Nation}. We on our part pointed out that the political unity of India can be preserved only if the linguistic cultural groups inhabiting a particular State is considered a distinct nationality within the indivisible Indian State. It was in this sense that the communists in the 1940s called India a multi-national State. Multi-national India defined by the communists is, in other words, supplementary rather than contradictory to the unity of India as a nation. 

That was why the Programme of the united Communist Party of India adopted in 1951 and the Programme of the CPI(M) and the CPI adopted in 1964 demanded maximum possible State autonomy for every cultural- linguistic group like the Malayalees, the Bengalis, the Tamils, the Andhras, the Kannadigas, the Punjabis etc. 

This was the distinct contribution made by the Indian communists to the nature and content of Indian politics. India’s unbreakable unity can be safeguarded only if the multiplicity of linguistically-culturally defined States are considered distinct nationalities within the united Indian State. No other political party has been able to put forward such a clear idea. This therefore is a valuable contribution made by the Marxists-Leninists to the theory and practice of India’s political thought. 

\subsection{Agrarian Revolution - Before and After Independence} 

The first Programmatic statement of the illegal Communist Party of India issued in 1930 drew attention to the inter-relationship between the National Revolution and the Agrarian Revolution. Indian freedom can be won only through the revolutionary means in which the multi-million peasantry are drawn into a militant movement headed by the working class. 

This was opposed to the ideas of the bourgeois leadership of the national movement led by Mahatma Gandhi, such as non-violence, the landlord-bourgeois classes being “trustees” of the people for their property etc. The Communist Party of India therefore was enjoined to carry on an ideological struggle against these Gandhian concepts, as well as socio-political struggles by way of organising the industrial and agricultural workers, working peasants and all other sections of the working people against bourgeois-landlord oppression. This was the dividing line between the bourgeois-sponsored right leadership of the Congress and the mass of Congressmen and women. 

Within the Congress organisation, a clear left group emerged in the mid- 1 930s. Its struggle against the right leadership reached its climax in the election to the Congress Presidentship in 1939. The left candidate fighting that election, Subhash Chandra Bose, issued a statement in which he alleged that the right-wing leadership of the Congress was trying to enter into negotiations with the British Rulers on transfer of power to India. This, he said, is contrary to the national interests. What is required is mass national struggle against British Imperialism for which policy he was fighting the election. 

The line advocated by the left candidate in the Congress Presidential election had the support of a majority of ordinary Congressmen. Bose therefore was elected President of the Congress. The bourgeois leadership however manoeuvred him out of the Congress Presidentship, going to the extent of personally expelling him from the Congress. 
This was the background against which two forms of anti-imperialist struggle were organised in the 1940s - the Quit India struggle led by the Congress headed by Mahatma Gandhi and the Indian National Army movement led by Subhash Chandra Bose. The former was apparently a militant struggle which drew millions of Indians into militant forms of action. The latter 
resulted in the formation of the Indian National Army headed by Subhash Chandra Bose which had an agreement with the military Rulers of Japan who extended support to the I. N. A. 

The Communist Party of India could not support either of them because the Quit India struggle organised by the Congress leadership was the preparation for a move to put pressure on the British Rulers to open negotiations with the bourgeois leaders for transfer of power. As for the INA, the Communist Party of India considered that, despite the genuinely anti-imperialist sentiments of its leaders, it was a movement directed by the Japanese imperialist power which had its contradictions with the British Rulers. 

The Communist Party of India therefore kept away from both the movements but, after the end of the war, it engaged itself in mighty militant mass struggles like the strike wave of the industrial workers and militant peasant movements like the Tebhaga movement in Bengal, the Punnapra Vayalar struggle in Travancore, the Telengana struggle in Hyderabad and peasant 
out-breaks in various parts of the country like the Telegu districts and Malabar in Madras, the district of Thane in Maharashtra, several districts in Punjab and so on. Together with the strike wave of industrial workers, middle class employees including government employees, students, youths, women etc. this was a mighty revolutionary upsurge in the post-war months. Despite its isolation from the anti-imperialist masses because of its opposition to Quit India and INA movements therefore, the Communist Party of India came out as the leader of revolutionary movements in the post-war years. 

However, the Communist Party of India was an extremely weak force in the anti-imperialist movement which by and large was led by the Congress. The latter therefore used the enormous prestige which it enjoyed thanks to its leadership of the Quit India and INA movements, to direct the immense mass upsurge as the basis on which negotiations could be opened between the Congress and the Government as well as between the Congress and the League. These 3 way-negotiations ultimately enabled the British rulers to put into practice the slogan given by the Muslim League to “divide and quit” India. The negotiations ended in the transfer of power to the Congress leaders in the Hindu-majority Indian Union and to 
the League leaders in the Muslim majority Pakistan. The militant mass actions, to some of which the Communist Party of India gave effective leadership like the Telengana struggle which lasted for 4 years and the Bombay revolt of the personnel of the Royal Indian Navy, were thus betrayed by the bourgeoisie because the proletariat and its party had not become powerful enough to develop the militant mass struggles into a revolutionary all-India upheaval. 

\subsection{Nationalism and Internationalism }

The Congress which led the Quit India struggle and the Subhash Chandra Bose leadership which gave leadership to the INA movement were two forms in which the bourgeois leadership tried to use the situation created by the Second World War for throwing the British out. The Congress leaders, used the Quit India struggle to initiate negotiations with the British Rulers, while the Subhash Chandra Bose leadership depended upon the anti-British faction of imperialist powers for winning freedom. Both were sincere in their desire to win freedom for India--the Congress leadership through negotiations with the British Rulers and the Bose leadership through direct military and financial assistance from the fascist powers to throw the British out of India. Both however were contrary to the interests of the Indian people who wanted to carry out an anti-imperialist and anti-feudal agrarian revolution. The CPI’s opposition to the Quit India and INA movements therefore was, in the last analysis, correct. However, the CPI was a minor force in the political scenario, unable to meet the onslaught of Gandhian ideology in the Quit India movement and the left bourgeois, pro-fascist ideology of the INA movement. 

The same difficulty had been faced during the years of the anti-fascist war by our comrades in several foreign countries, particularly in China, Vietnam, Laos and Cambodia as well as Korea. They too had to face the ideological and political offensive of the bourgeoisie. But, unlike us in India, they were far stronger, far more powerful in their respective national movements. They could organise a wide national revolutionary movement which was anti-fascist as well as directed against the ruling imperialist power in their respective countries. They therefore could come out of the war period much stronger than we in India. 

The Chinese comrades could throw the Kuomintang, reactionaries out of power, making China People’s Democratic Republic. In Vietnam too, at the end of the war, half of the country was liberated where a People s Democratic Government headed by Ho Chi Minh was established. The 
same thing happened in Korea where the Communists could establish themselves as the ruling party in the northern part of the country, while the south remained under the leadership 
of the reactionary bourgeoisie. That was how the end of the anti-fascist war saw India divided on communal lines, while China, North Vietnam and North Korea had People’s Democratic Revolutions. 

China, Vietnam and Korea were thus examples of the successful carrying out of the programme of anti-imperialist agrarian revolution, while in India the process of agrarian revolution was disrupted by the manoeuvres reported to by the British Rulers, together with the two (the Congress and the League) wings of the Indian bourgeois leadership. 

It however goes to the credit of the CPI that, during the two bourgeois-led movements of the Quit India and the INA, the Party could successfully steer clear of bourgeois ideologies and keep the flag of proletarian internationalism afloat. The Party stuck to the principle of proletarian internationalism, declaring that the anti-fascist war waged by the Soviet Union, China, Vietnam and Korea was as much a People’s War for India as it was for those particular countries. 

The CPI pointed out that active participation in the international People’s War against Fascism and the organisation of the anti-imperialist agrarian revolution were the two factors 
that would liberate India from British rule. Braving against the attacks from the misguided national bourgeoisie, the party held aloft the flag of resistance to fascism as an integral part of India’s freedom struggle. That was why, after a short period of complete isolation from the anti-imperialist masses (1942- 45), the Party could gather the patriotic forces and give effective militant leadership to the post-war revolutionary upsurge of 1946-47. 

However, because of the organisational weakness of the CPI, the Congress and the League leadership could come to a negotiated settlement with the British Rulers for transfer of power to the Congress in the Indian Union and to the Muslim League in Pakistan. But the very role played by the CPI in leading the post-war revolutionary upsurge made the party a significant force in Indian politics. 

The result was that, in the first general elections which took place in 1952, the party came out as the major party of opposition both in the Lok Sabha and the Rajya Sabha. The roles played by A.K.G. in the Lok Sabha and Sundarayya in the Rajya Sabha, together with the work of the Communist and allied groups in Travancore-Cochin, Madras, Hyderabad, West Bengal and Tripura enabled the Party to become a force to reckon within Indian politics, although, in electoral terms on the national scale it was a small minority. 

Mention may, in this context, be made also to the role played by the then left in the united CPI on the question of India-China dispute in 1959-1962. As in the days of the Quit India and INA movements, in the days of the India-China dispute too, the CPI as a whole, particularly the left in the CPI, was relatively isolated among the anti-imperialist masses. The stand adopted by the left in the CPI however made it clear that, as opposed to the line of war with China adopted by all the bourgeois parties and the right wing in the CPI, the left in the CPI advocated peaceful negotiations for settlement of the India-China dispute which was proved correct in subsequent years when the Government of India itself had practically to admit that it was foolish on its part to have gone to war with China on the border question. Not only the then right in the CPI but even such parties as the Congress and the subsequent BJP had to admit that the border dispute between India and China has to be settled through negotiations with China and not through war. 

The three decades that elapsed since the beginning of the organised work of the CPI’s leadership in 1934 to the split of the party in 1 964 was thus the period in which the united CPI made big advances but during which serious differences arose within the CPI — between the right and the left wings in the Party. 


\subsection{Right and "Left” Opportunism in the Pre-split CPI}

Although it witnessed big advances made by the Party from 1934 to 1964, its leadership was a victim of right and “left” opportunism. In the 1942-1946 period, the Party leadership was the 
victim of right opportunism. This was corrected at the Second Congress of the Party held in 1948, but the leadership switched over from right to “left” opportunism. 

The left opportunism which made its first appearance at the Second Congress of the Party, repeated itself in a new form at the 1950 May meeting of the Central Committee. Both however 
were corrected, after serious discussions with the leadership of the CPSU, at a Special Conference held in October 1951. 

The common mistake of the right and “left” opportunism was the failure to realise that, if the programme of carrying out the anti-imperialist agrarian revolution is to be implemented, the 
Party has to base itself on the militant struggles of the working people led by the working class, combined with the tactics of revolutionary use of elections and legislatures under the bourgeois constitution. Out of this common failure arose the two deviations of (a) the bourgeois right opportunism and (b) the petty bourgeois “left” opportunism. These were subjected to serious self-criticism which ended in the formulation of a new Party Programme and Statement of Policy. 

The Programme stated that the immediate objective of the Communist Party of India is not the introduction of socialism in India which is impracticable since the objective conditions are lacking and the necessary subjective forces are not developed. The Programme however pointed out that the crisis of the system in the post independence period had brought to the fore-front the possibility of bringing about a people’s democratic revolution out of which will arise a people’s democratic State and Government. It was on this basis that a Programme of people’s democracy was adopted at the Calcutta special conference of the party in October 1951. 

Having thus spelt out the objective of the people’s democratic revolution which was incorporated in the Party Programme, the Statement of Policy discus.sed the tactical line. Considering the Client objective situation and the maturity of the development of the subjective forces, the Statement of Policy, adopted along with the Party Programme, made two points:

Firstly, post-independence India having established a bourgeois democratic system with its elected legislatures to which the executives are responsible, it is necessary for the Party to go into the electoral battle, secure as many electoral victories as possible since all electoral battles are the concrete manifestations of class struggle.

Secondly, struggle on the parliamentary arena is only an important aspect of the struggle for People's Democracy. As important as the struggle on the electoral arena is the struggles of the working people outside the parliamentary arena. Any neglect of the struggle on the parliamentary arena will no doubt amount to failure to use the immense possibilities of gathering the revolutionary forces under the leadership of the working class.


There should however be no illusion, spread by the right wing Social Democrats, that merely through the struggles on the parliamentary arena, the struggle for people’s democracy will end in success. On the other hand, the very use of bourgeois parliament for preparing people’s democracy should be subordinated to the extra parliamentary struggle. Non-use of the revolutionary potentialities of the struggle on the parliamentary arena and subordinating the mass struggle to parliamentary work are the “left” and right forms of opportunism against which the Party should guard itself. 

The Statement of Policy, adopted along with the Party Programme, had visualised a situation where the Party will have to adopt the tactics of waging peasant guerilla war as was done in Telengana. Actual experience however proved that, if the bourgeoisie resorted to the method of sabotage against People’s Democracy, that situation has to be faced by mobilising more and more masses of working people led by the working class against the bourgeois-landlord Government. 

That was why the Party did not have any hesitation to form its own government in Kerala in 1957 and use that opportunity to mobilise the people around the programme of agrarian reforms and other measures which would bring more and more people to the cause of People’s Democracy. 

The major contribution made by the pre-split CPI in he 1950s was the successful use of the parliamentary arena-membership of the two houses of parliament at the Centre and State legislature everywhere as well as the Kerala Government without slipping into the right opportunist “Parliamentarism” of the social democratic type. 

The left-wing in the pre-split CPI which converted itself into the CPI(M) made a concrete analysis of the rise and overthrow of the Kerala Government on the basis of which the newly emerging CPI(M) wrote pargraph 112 in the 1964 Programme. While drawing attention to the possibilities of forming left- led governments in various States, that paragraph warned against the illusion that the governments formed on that basis would be able to solve the basic problems either of the country as a whole or of the State concerned. 

It was on the basis of this understanding, further strengthened by the formation of two left-led governments (Kerala and West Bengal) in 1967 that the CPI(M) formulated the idea that the left-led governments should be considered as “instruments of struggle for winning more and more popular masses, more and more political allies, to the cause of People’s Democracy". 

One consequence of this understanding of the relation between the parliamentary and extra-parliamentary struggles was the fact the CPl(M) came to the conclusion that, while it should take initiative to form and lead its own State Governments wherever possible, it would not join governments unless the Party is sure that it can influence the policies of the Governments. 

That was why, in 1967, the non-left governments in Punjab, U.P. and Bihar were supported by the Party, though it did not join them. 'The same policy was adopted by the CPl(M) in relation to the Janata Government in 1977, the National Front Government in 1989 and, finally, the United Front Government in 1996. 

\subsection{In the Post-split CPI and the CPl(M)}

The Struggle between the right and the left in the pre-split CPI ultimately led to the 1964 split. This was based on 3 ideological-political issues and one organisational issue. The ideological political issues were: 

(a) The attitude to the Congress and its Government: The right in the pre-split CPI and the post-split CPI looked upon the Congress as a “progressive national organi.sation" with which the Communists should collaborate. The CPl(M), on the other hand, stood for uncompromising struggle against the Congress, even though, from issue to issue, the Party was prepared to cooperate with the Congress. (At the Fourth Congress of the pre-split CPI, the right in the Party had called for such unity with the Congress as would eventually lead to the formation of a Congress-Communist coalition government.) Defeated at that particular Party Congress, they did, after the split, implement the line in practice in Kerala and West Bengal: the CPI leader in Kerala (Achutha Menon) headed the anti CPI(M) Government of which the Home Minister was the notorious Karunakaran; in West Bengal too, the post-split CPI extended full cooperation to the semi-fascist terror regime of the early 1970s and subsequently to Indira Gandhi’s Emergency regime. 

(b) On the India-China border dispute, the post-split CPI extended full support to the campaign for the anti-China War. The CPl(M). on the other hand, fought for the Government of 
India entering into negotiations with China and settling the border question through talks. The post-split CPI joined not only the Congress but even the Jan Sangh in carrying on a chauvinistic campaign against the Chinese and against the so-called “pro-China” CPl(M). It was from this campaign that the bourgeois politicians and the bourgeois media took courage to characterise the CPI(M) as “pro-China” and the post-split CPI as “Pro- 
Russia”. 

(c) On question of ideological dispute between the Soviet Party on the one hand and Chinese Party on the other, the CPI fully sided with the Soviet position, while the CPI(M) opposed the Soviet line, even while entertaining reservations on some aspects of the Chinese line. 

Together with these major ideological and political issues was the question of inner-party unity on which too pre-split CPI was sharply divided. The left in the pre-split CPI demanded that all ideological and political questions in dispute between the two sides should be discussed at a special party conference convened on the basis of party membership existing at the previous party Congress. This was not acceptable to the right in the pre-split CPI. Based on this accidental majority of members of the leading bodies of the pre-split CPI, they demanded that their political and organisational position should be accepted by the entire party. The accidental majority enjoyed by them in the leading bodies of the pre-split CPI emboldened them to start taking disciplinary actions against those who differed from them. All th§ appeals made by the left in the pre-split CPI for finding democratic solutions to the political and organisational problems were rejected. The left in the pre-split CPI therefore were forced to organise themselves separately, forming a new separate party called the CPI(M). 
The ideological, political and organisational struggle between the post-split CPI and the CPI(M) was thus a continuation of the struggle that went on in the pre-split CPI. Considering the fact that, on all the 3 issues on which the right and the left fought each other in the pre-split CPl-approach to the Congress and its Government, the India-China border dispute and the crisis in the international communist movement-subsequent history has proved that the post-split CPI had adopted a wrong line while the CPI (M) had a correct line. Any attempt at re-unification of the communist movement should therefore begin with a serious self-criticism on the part of the post-split CPI. 

Before closing this subject, it is necessary to note that the 
CPI(M) had to fight and defeat the "left” and right opportunism 
which raised its head in the Party, as for example: 

(a) There was a trend in the Party leadership in the 1970s 
that, in view of the semi-fascist terror in West Bengal and the Emergency regime at the national level, the Party should organise itself for a Telengana type struggle ;

(b) This found expression at the 10th Congress (Jalandhar) where a minority of delegates opposed the line of supporting the Janata Government ;

(c) There was a strong opposition at the Salkia Plenum for the proposition that the Party should develop itself as a "a mass revolutionary party of the working class". The argument was that a revolutionary party can never be a mass party.

(d) When the Party decided to break with the Janata Government for its links with the RSS, there was opposition to it from a minority.

(e) On the eve of the formation of the United Front, a view was expressed in the PB and CC that we should join the government, even leading it. 

It is thus obvious that the post-split CPIM) has to fight and defeat the manifestations of “left” and “right” opportunism. The post-split CPI, Naxalites and other groups that consider 
themselves communist will therefore have to undertake a serious self- criticism of their ideology, politics and organisation. 

\subsection{Re-Unification of Ideology and Politics before Organisational Re-Unification }

Three decades after the formal coming into existence of a regularly functioning CPI, the party got divided into the CPI(M) and CPI; from within the CPI(M) itself, the Naxalites and other “Left Communists” also came into existence. Furthermore, certain other -left parties like the Forward Bloc and the RSP are today in the field. 

After a decade and a half of the separate existence of the CPI(M) and the CPI, they started working together in struggle against the Congress on the one hand and certain other bourgeois parties (including caste and communal parties) on the other. These united actions graduall' developed into a more stable united front of left parties which allied itself with certain other secular democratic parties. 

That was how the two Left Front Governments of West Bengal and Tripura as well as the Left Democratic Front Government in Kerala have come into existence. The constituents of these left and democratic fronts have come together also at the national level to form the United Front 
Government. 

This United Front Government at the Centre and the left coalition governments in the 3 States have raised the question whether the time has not come for all parties and organisations which consider themselves Marxist-Leninists to merge themselves in a single party. It was the CPI which raised this question first. But the sentiment has spread in other left parties as well. 

The CPI(M) however considers it impractical and undesirable at present to have a single united Communist Party of India. The organisational coming together of various Marxist-Leninist groups should in its view follow, not precede, the unification of an ideological and political character. 

Unless the question of why the pre-split CPI was divided into the CPI(M) and the CPI and why other left groups calling themselves Marxist-Leninists have come into existence is addressed properly and answers given, it will be ridiculous to think of organisational unification of all existing Marxist-Leninist groups. That is why the CPI(M) suggests: 

(a) Unity in action to develop the emerging unity of left, 
secular and democratic forces ; 

(b) Continue with fraternal discussions of the ideological and political questions which divide various groups with a view to arriving at a common understanding on the international, national. political and organisational questions. As pointed out by Lenin, ideological and political unity is an essential pre-condition for organisational unity. 

The emergence of a united movement of left, democratic secular and federalist forces is a great achievement of the more than 7 decades of the existence of India’s Marxist-Leninists. In the course of this work of 7 decades, while the Indian communists have made tremendous advances, they have also committed some serious mistakes. These should be subjected to serious self-criticism (by all Marxist-Leninist groups) with a view to consolidate achievements and correct the mistakes. Only such a self-criticism, undertaken by all Marxist-Leninist groups, would prepare the ground for the organisational unification of all communist parties and groups into one single united Marxist-Leninist Party in India.

While the time has come for us to keep that objective, it is pre-mature to give the slogan of a merger of all Marxist-Leninist groups. The essential pre-condition for the subsequent re unification or merger of all Marxist-Leninist groups is the development of the emerging unity in action among all left groups on the basis of a critical and self-critical analysis of the 7 decades of the development of India's Marxist-Leninist movement.
